\section{Dimostrazione per la monade identità}
\label{dimostrazione-per-la-monade-identita}

Alla \Cref{la-monade-identita} è stata data una definizione di monade per \lstinline{Identity}. In seguito è riportata una dimostrazione del rispetto delle leggi monadiche elencate nella \Cref{cos-e-una-monade} per la definizione fornita\footnote{Nei passaggi della dimostrazione, così come in quelle successive, sono utilizzati i nomi canonici utilizzati per l'implementazione Scala; quindi, si utilizza \lstinline{flatMap} anziché \lstinline{>>=} e \lstinline{pure} anziché \lstinline{return}}.

Dimostrazione dell'identità sinistra, ovvero che \lstinline{pure(a).flatMap(f) = f(a)}:

\begin{tabularx}{\textwidth}{ll}
  \lstinline{pure(a).flatMap(f) =} & \emph{Definizione di \lstinline{pure}} \\
  \\
  \lstinline{a.flatMap(f) =} & \emph{Definizione di \lstinline{flatMap}}    \\
  \\
  \lstinline{f(a)}$\qed$ &
\end{tabularx}

Dimostrazione dell'identità destra, ovvero che \lstinline{m.flatMap(pure) = m}:

\begin{tabularx}{\textwidth}{ll}
  \lstinline{m.flatMap(pure) =} & \emph{Definizione di \lstinline{flatMap}} \\
  \\
  \lstinline{pure(m) = m}$\qed$ & \emph{Definizione di \lstinline{pure}}    \\
\end{tabularx}

Dimostrazione dell'associatività, ovvero che \lstinline{(m.flatMap(f)).flatMap(g) = m.flatMap(x => f(x).flatMap(g))}:

\begin{tabularx}{\textwidth}{ll}
  \lstinline{(m.flatMap(f)).flatMap(g) =} & \emph{Definizione di \lstinline{flatMap}} \\
  \\
  \lstinline{f(m).flatMap(g) =} & \emph{Definizione di \lstinline{flatMap}}           \\
  \\
  \lstinline{g(f(m)) =} & \emph{Composizione di funzione}                             \\
  \\
  \lstinline{(x => g(f(x)))(m)} & \emph{Definizione di \lstinline{flatMap}}           \\
  \\
  \lstinline{m.flatMap(x => g(f(x))) =} & \emph{Definizione di \lstinline{flatMap}}   \\
  \\
  \lstinline{m.flatMap(x => f(x).flatMap(g))}$\qed$ &
\end{tabularx}

