\chapter{Basi del linguaggio Haskell}
\label{app:haskell}

Haskell è un linguaggio di programmazione funzionale puro, basato su un modello di valutazione \term{lazy} e con inferenza di tipi statica.

L'obiettivo di questa appendice è fornire alcuni elementi di base di Haskell per rendere chiari gli esempi riportati nei \Cref{chapter:side-effect-e-loro-modellazione,chapter:stack-di-monadi} al lettore che non abbia familiarità con il linguaggio.

\section{Sintassi di base}
\subsection{Definizione e invocazione di funzioni}
Haskell adotta una sintassi simile a quella dei linguaggi nella famiglia ML. In particolare, per dichiarare una funzione è sufficiente indicarne il nome, una lista di argomenti e il corpo con l'implementazione:
\begin{lstlisting}[language=haskell]
exampleFunction arg1 arg2 = ... {* function body *}
\end{lstlisting}

Il \term{type system} di Haskell è basato sul \term{type system} di Hindley e Milner e, nella maggior parte dei casi, è in grado di inferire automaticamente il tipo delle funzioni senza bisogno di annotazioni.
Per esempio, data la seguente funzione:
\begin{lstlisting}[language=haskell]
hello name = "Hello, " ++ name ++ "!"
\end{lstlisting}
Il tipo della inferito è \lstinline{String -> String}: la funzione prende in input una stringa e restituisce una stringa. L'operatore \lstinline{++} viene utilizzato per la concatenazione di stringhe e di liste.

Se si volesse indicare esplicitamente il tipo di una funzione è possibile aggiungere un'annotazione nel seguente modo:
\begin{lstlisting}[language=haskell]
hello :: String -> String
hello name = "Hello, " ++ name ++ "!"
\end{lstlisting}

Si consideri una funzione con più argomenti:
\begin{lstlisting}[language=haskell]
sum :: Int -> Int -> Int
sum x y = x + y
\end{lstlisting}
In questo caso il tipo della funzione è \lstinline{Int -> Int -> Int}: i tipi degli argomenti sono separati da \lstinline{->} e l'ultimo tipo indicato è il tipo di ritorno della funzione. Quindi, la funzione \lstinline{sum} prende in input due argomenti (\lstinline{x} e \lstinline{y}) di tipo intero e restituisce un valore intero.

Per poter invocare una funzione è sufficiente fornirle i parametri separandoli con degli spazi; a differenza di linguaggi come C e Scala l'invocazione di funzione non richiede la scrittura di parentesi tonde che ne racchiudano gli argomenti. Per esempio per invocare la funzione \lstinline{hello} è sufficiente scrivere \lstinline{hello "Haskell"} e il risultato della sua invocazione sarà \lstinline{"Hello, Haskell!"}.
Invece, l'invocazione di \lstinline{sum} può avvenire passandole due argomenti di tipo intero: \lstinline{sum 1 2 = 3}.

Inoltre, qualunque funzione abbia due soli argomenti può sempre essere invocata come se fosse un operatore infisso inserendola fra \lstinline{`}:
\begin{lstlisting}[language=haskell]
invokeSum =
  let result1 = sum 1 2
      result2 = 1 `sum` 2
   in (result1, result2)
\end{lstlisting}
Le due invocazioni daranno lo stesso risultato. La sintassi \lstinline{let ... in ...} serve a definire dei valori locali al corpo di una funzione.
Il risultato di un blocco \emph{let-in} è il valore indicato dopo la parola chiave \lstinline{in}; in questo caso sarà la tupla \lstinline{(result1, result2)}.

\section{ADT e pattern matching}
Haskell permette di definire ADT tramite la parola chiave \lstinline{data}; in maniera analoga alle enumerazioni Scala, un ADT può essere generico e i suoi costruttori avere campi differenti:
\begin{lstlisting}[language=haskell]
data Option a
  = Some a
  | None
\end{lstlisting}
A differenza di Scala, i tipi generici sono denotati con lettere minuscole e vengono semplicemente indicati dopo il nome del tipo. In questo caso è stato dichiarato un tipo \lstinline{Option} generico su un tipo \lstinline{a} con due costruttori:
\begin{itemize}
  \item Il costruttore \lstinline{Some} richiede un solo argomento di tipo \lstinline{a}
  \item Il costruttore \lstinline{None} non richiede alcun argomento
\end{itemize}

Per poter accedere ai campi di un costruttore è necessario procedere per pattern matching, in Haskell questo è possibile con la parola chiave \lstinline{case}:
\begin{lstlisting}[language=haskell]
mapOption :: Option a -> (a -> b) -> Option b
mapOption option f =
  case option of
    Some value -> Some (f value)
    None -> None 
\end{lstlisting}
La funzione \lstinline{mapOption} prende in input un valore opzionale di tipo \lstinline{Option a} -- dove \lstinline{a} è un tipo generico -- e una funzione \lstinline{a -> b}. Nel corpo della funzione viene fatto pattern matching sul valore opzionale e, nel caso in cui corrisponda al costruttore \lstinline{Some}, viene applicata la funzione \lstinline{f} al valore contenuto al suo interno.

Si confronti la funzione con l'equivalente versione Scala:
\begin{lstlisting}[language=scala3]
def mapOption[A, B](option: Option[A])(f: A => B): Option[B] =
  match option
    case Some(value) => Some(f(value))
    case None => None
\end{lstlisting}
In Scala è necessario indicare anticipatamente quali sono i tipi generici della funzione mentre in Haskell è possibile utilizzarli senza una previa indicazione.

Un esempio di utilizzo della funzione \lstinline{mapOption} potrebbe essere il seguente:
\begin{lstlisting}[language=haskell]
useMapOption =
  mapOption (Some 1) (\value -> value + 1)
\end{lstlisting}
La sintassi \lstinline{\arg0 arg1 ... argn -> ...} permette di definire una funzione anonima.

\section{Type classes}
Haskell dà supporto al \emph{polimorfismo ad hoc} tramite l'uso delle \term{type classes}~\cite{cit:type-classes-in-haskell}.
Una \term{type class} permette di specificare una serie di funzioni che devono essere definite per un tipo generico:
\begin{lstlisting}[language=haskell]
class Show s where
  show :: s -> String
\end{lstlisting}
In questo caso è stata dichiarata la \term{type class} \lstinline{Show} e un qualunque tipo generico \lstinline{s} voglia implementarla deve fornire un'implementazione della funzione \lstinline{show} che trasforma un tipo \lstinline{s} in una stringa.

Una \term{type class} può essere interpretata come un predicato espresso su un tipo: per esempio, la \term{type class} \lstinline{Show} può essere letta come \emph{``Dato un generico tipo \lstinline{s}, per questo vale il predicato \lstinline{Show} se esiste una funzione \lstinline{show} in grado di convertire \lstinline{s} in una stringa''}.

Istanziare una \term{type class} equivale quindi a dimostrare che uno specifico tipo rispetta il predicato fornendo come dimostrazione l'implementazione delle funzioni definite nella \term{type class}.
Riprendendo l'esempio di \lstinline{Show} è possibile definire un'istanza per il tipo \lstinline{Option String}:
\begin{lstlisting}[language=haskell]
instance Show (Option String) where
  show option =
    case option of
      Some string -> "Some " ++ string
      None -> "None"
\end{lstlisting}
Una volta fornita l'istanza per uno specifico tipo sarà possibile utilizzare la funzione \lstinline{show} su valori di quel tipo:
\begin{lstlisting}[language=haskell]
main = println (show (Some "Hello"))
-- "Some Hello"
\end{lstlisting}

\subsection{Uso delle type class per fornire un contesto alle funzioni}
In Haskell è possibile sfruttare il meccanismo delle \term{type class} per poter vincolare i tipi generici delle funzioni.
Si consideri la seguente type class:
\begin{lstlisting}[language=haskell]
class Eq a where
  (==) :: a -> a -> Bool
\end{lstlisting}
Se un tipo \lstinline{a} fornisce un'istanza di \lstinline{Eq} sarà possibile confrontarne gli elementi tramite la funzione \lstinline{==}. Il linguaggio fornisce una serie di implementazioni standard per alcuni tipi di base come \lstinline{Int} e \lstinline{String}.

Si immagini di dover realizzare una funzione che permetta di confrontare due valori opzionali: questi sono uguali solo se sono entrambi \lstinline{None}, oppure nel caso in cui siano entrambi definiti e gli elementi contenuti al loro interno siano uguali:
\begin{lstlisting}[language=haskell]
areEqual :: Option a -> Option a -> Bool
areEqual optional1 optional2 =
  case (optional1, optional2) of
    (None, None) -> True
    (Just x, Just y) -> ...
    _ -> False
\end{lstlisting}
Per poter confrontare i due valori \lstinline{x} e \lstinline{y} è necessario poter utilizzare una funzione come \lstinline{==}. Tuttavia, \lstinline{==} può essere definito solo per un tipo concreto che fornisca un'istanza della \term{type class} \lstinline{Eq}.

Per poter implementare la funzione \lstinline{areEqual} è possibile esprimere un vincolo sul tipo generico \lstinline{a} nel seguente modo:
\begin{lstlisting}[language=haskell]
areEqual :: (Eq a) => Option a -> Option a -> Bool
areEqual optional1 optional2 =
  case (optional1, optional2) of
    (None, None) -> True
    (Just x, Just y) -> x == y
    _ -> False
\end{lstlisting}
Si osservi come è cambiato il tipo della funzione: prima della lista dei tipi degli argomenti è stato specificato un \emph{vincolo} sul generico tipo \lstinline{a}. La notazione \lstinline{(Eq a) =>} impone che la funzione \lstinline{areEqual} possa essere invocata solo su valori opzionali parametrizzati su un tipo per cui sia disponibile un'istanza della classe \lstinline{Eq}.

Riprendendo l'interpretazione delle type class come predicato si può interpretare leggere il vincolo come \emph{``se i valori di tipo \lstinline{a} rispettano il predicato \lstinline{Eq} allora è possibile definire una funzione \lstinline{areEqual} che confronta valori opzionali parametrizzati su \lstinline{a}''}.

Si noti come, una volta imposto il vincolo, all'interno del corpo della funzione è possibile utilizzare la funzione \lstinline{==} per confrontare i valori \lstinline{x} e \lstinline{y} di tipo \lstinline{a}.

