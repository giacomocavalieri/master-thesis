\documentclass[12pt,a4paper]{book}

\usepackage[utf8]{inputenc}
\usepackage{thesis-style}

\begin{document}

% ! TeX root = master-thesis.tex
\newgeometry{margin=0.8in}
\begin{titlepage}
	\begin{center}
		\large
		\textbf{ALMA MATER STUDIORUM -- UNIVERSITÀ DI BOLOGNA \\ CAMPUS DI CESENA}
		\\
		\noindent\hrulefill
		\vspace{0.4cm}

		\Large
		Scuola di Ingegneria e Architettura \\
		Corso di Laurea Magistrale in Ingegneria e Scienze Informatiche

		\Huge
		\vspace{4cm}
		\textbf{Gestione degli effetti in linguaggi di programmazione funzionale: tecniche di modellazione e interpretazione}

		\large
		\vspace{1cm}
		Tesi di laurea in
		\\
		\textsc{Paradigmi di Programmazione e Sviluppo}

		\vspace{5.5cm}
		\begin{minipage}[t]{0.64\textwidth}
			\begin{flushleft}
				\textit{Relatore}
				\\
				\textbf{Prof.} \textbf{Mirko Viroli}
				\\
				\vspace{0.4cm}
				\textit{Correlatore}
				\\
				\textbf{Dott.} \textbf{Roberto Casadei}
			\end{flushleft}
		\end{minipage}
		\begin{minipage}[t]{0.34\textwidth}
			\begin{flushright}
				\textit{Candidato}
				\\
				\textbf{Giacomo Cavalieri}
			\end{flushright}
		\end{minipage}\\

		\vfill
		\noindent\hrulefill
		\vspace{0.3cm}
		\Large

		Quarta Sessione di Laurea
		\\
		Anno Accademico 2021-2022
	\end{center}
\end{titlepage}
\restoregeometry
\tableofcontents
\newpage

\listoftodos

%\todo{Aggiungere una parte introduttiva di più largo respiro, con osservazioni sul paradigma funzionale e sull'espressività/efficacia della programmazione in questo paradigma}
\mustfix{Spiegare perché notazione haskell ed esempi di codice in scala nell'introduzione. Serve contestualizzare sul perché si cambia la notazione: si usa Scala come linguaggio di riferimento in quanto linguaggio funzionale con side effect è utile per mostrare esempi di codice sia monadico che con side effect.}

\chapter{Modellazione dei side effect}
\label{chapter:side-effect-e-loro-modellazione}

La ragion d'essere di un qualunque programma è quella di produrre un qualche effetto tangibile sul mondo esterno: che sia scrivere dei dati su disco, inviare un messaggio sulla rete, stampare dei caratteri a schermo ecc.
Linguaggi dalla natura imperativa come C, Java o Scala forniscono delle funzioni apposite per ottenere tali effetti; in Scala, per esempio, si potrebbe implementare la seguente funzione per apporre una stringa al contenuto di un file su disco\footnote{In questa funzione così come in tutte le altre funzioni Scala a seguire viene adottata la sintassi introdotta da Scala 3~\cite{cit:new-in-scala-3}.}:

\begin{lstlisting}[language=scala3]
def appendToFile(file: File, line: String): Unit =
  println(f"Appending $line to $file")
  val writer = FileWriter(file, true)
  try writer.write(f"$line\n")
  finally writer.close
\end{lstlisting}
Lo scopo di funzioni come \lstinline{appendToFile}, \lstinline{write} e \lstinline{println} non è quello di produrre un risultato -- il loro valore di ritorno è sempre \lstinline{Unit} -- ma di mettere in atto dei \emph{side effect}.

Più in generale, con side effect si intende una qualunque interazione della funzione con un ambiente diverso da quello locale.
Possono essere quindi considerati side effect il modificare una variabile globale o un parametro passato per riferimento, lanciare un'eccezione, l'effettuare operazioni di input e output -- come mostrato precedentemente in \lstinline{appendToFile} -- o il chiamare una funzione che a sua volta presenta dei side effect~\cite{cit:on-the-prevalence-of-function-side-effects-in-general-purpose-open-source-software-systems}.

Le funzioni che presentano side effect sono spesso anche dette \emph{impure} mentre funzioni che non hanno side effect sono anche dette \emph{pure} o caratterizzate da \emph{trasparenza referenziale}.
\section{Modellazione esplicita dei side effect}
\label{modellazione-esplicita-dei-side-effect}

Date le criticità evidenziate nella sezione precedente, diverse pratiche di buona programmazione suggeriscono di ridurre al minimo le funzioni che presentano side effect~\cite[p.~44]{cit:clean-code-a-handbook-of-agile-software-craftsmanship} e, quando inevitabili, di renderli espliciti nel nome della funzione~\cite[p.~313]{cit:clean-code-a-handbook-of-agile-software-craftsmanship}.

Tuttavia, si possono individuare altre tecniche più sofisticate per tracciare i side effect delle funzioni.
Queste nascono nel contesto dei linguaggi funzionali puri, perciò è necessario comprendere come questa classe di linguaggi possa conciliare la presenza di side effect con la loro natura puramente funzionale.

\subsection{Linguaggi funzionali puri e side effect}
\label{linguaggi-funzionali-puri-e-side-effect}
Un linguaggio funzionale si dice \emph{puro} se le sue funzioni sono referenzialmente trasparenti.
Questa totale assenza di side effect sembra in netto contrasto con la possibilità di scrivere codice di una qualche utilità: come può un programma puro interagire con il mondo esterno, leggere il valore di uno stato globale o lanciare eccezioni?

La soluzione consiste nella possibilità di ``simulare'' la presenza di side effect tramite opportune modifiche ai tipi delle funzioni mantenendole pure.
In seguito sono riportati alcuni esempi di questo approccio in Scala\footnote{Sebbene Scala sia un linguaggio impuro può essere utilizzato come se fosse un linguaggio puro evitando di ricorrere ai meccanismi che fornisce per produrre side effects.}.

\subsubsection{Eccezioni}
\label{eccezioni}
Le eccezioni sono un meccanismo di controllo del flusso che permette di interrompere l'esecuzione di una funzione e di ritornare il controllo al chiamante.
Tuttavia, una funzione che ne fa uso non avrà trasparenza referenziale:
\begin{lstlisting}[language=scala3]
def divBy(n: Int): Int =
  n match
    case 0 => throw Exception("n = 0")
    case _ => 10 / n
\end{lstlisting}
\lstinline|divBy| non può essere assimilata a una funzione $divBy : \mathbb{Z} \rightarrow \mathbb{Z}$ in termini matematici; infatti, per certi input -- in questo caso 0 -- la funzione non restituisce un valore appartenente al proprio codominio ma lancia un'eccezione.

Per rendere esplicito il fatto che la funzione possa terminare in maniera anomala per determinati input si può estenderne il codominio: ritornando all'analogia matematica si può pensare ad \lstinline{divBy} come a una funzione $divBy : \mathbb{Z} \rightarrow \mathbb{Z} \cup \{ \texttt{Error} \}$.
  L'equivalente implementazione Scala sarebbe:
  \begin{lstlisting}[language=scala3]
enum Result:
  case Ok(value: Int)
  case Error

def safeDivBy(n: Int): Result =
  n match
    case 0 => Error
    case _ => Ok(10 / n)
\end{lstlisting}
  In questo caso è evidente dal tipo di ritorno di \lstinline|safeDivBy| che questa può fallire restituendo un valore di tipo \lstinline{Error}.
  La funzione non nasconde questo comportamento tramite il meccanismo delle eccezioni, lo rende invece evidente nel proprio tipo \lstinline{Int => Result}.
  Il grande vantaggio di rendere esplicita la possibilità di fallimento nel tipo di ritorno sta nel fatto che il programmatore dovrà obbligatoriamente gestire anche il caso in cui la computazione fallisca o il compilatore solleverà un errore a tempo di compilazione:
  \begin{lstlisting}[language=scala3]
def useSafeDivBy(n: Int): Int =
  // safeDivBy(n) + safeDivBy(n + 1) darebbe un errore
  // a tempo di compilazione dato che safeDivBy(x)
  // ha come tipo Result e non Int
  safeDivBy(n) match
    case Error      => 0
    case Ok(value1) => safeDivBy(n + 1) match
      case Error      => 0
      case Ok(value2) => value1 + value2
\end{lstlisting}

  \subsubsection{Lettura e modifica di uno stato globale}
  \label{lettura-e-modifica-di-uno-stato-globale}

  Dal momento in cui una funzione legge una variabile globale perde la propria trasparenza referenziale: il suo risultato non dipende più dai soli valori passati in input ma da un nuovo input \emph{nascosto}, lo stato a cui accede.
  La soluzione per rendere esplicita questa dipendenza è passare come parametri di tutti gli elementi necessari al funzionamento della funzione, senza affidarsi alla definizione in uno scope esterno di variabili globali a cui accedere implicitamente.

  Questa semplice trasformazione permette di rimuovere il side effect che consiste nella lettura di uno stato globale.
  Sfortunatamente, non è sufficiente per modellare anche la \emph{modifica} di una variabile globale. In questo caso può essere utile tornare all'analogia con le funzioni matematiche.
  Prendiamo come esempio la funzione \lstinline{f} definita in precedenza:
  \begin{lstlisting}[language=scala3]
var counter = 0
def f(x: Int): Int =
  counter = counter + 1 
  x + counter
\end{lstlisting}
  Questa prende in input un numero, ha il side effect di incrementare un contatore globale e poi ne legge il valore per aggiungerlo all'input. Sebbene il tipo di \lstinline{f} sia \lstinline{Int => Int}, a causa dei suoi side effect questa funzione non può essere modellata come una funzione matematica $f: \mathbb{Z} \rightarrow \mathbb{Z}$: a parità di input non darà sempre lo stesso output.

  Dato che la funzione necessita di leggere una variabile definita esternamente, questa dovrà essergli passata in come input esplicitamente:
  \begin{lstlisting}[language=scala3]
def wrongF(x: Int, counter: Int): Int =
  // Come modificare lo stato di current?
  val newCounterState = counter + 1
  x + newCounterState
\end{lstlisting}
  Rimane il problema di come poter modellare la modifica dello stato globale in modo che tale effetto si ripercuota anche su chiamate successive. Infatti, il side effect di aumentare il valore del contatore fa parte della logica applicativa di \lstinline{f} e rimuoverlo comporterebbe un cambiamento nella sua semantica.

  La soluzione consiste nel trasformare la funzione in modo tale che restituisca il valore del nuovo stato così che possa essere utilizzato per chiamate successive:
  \begin{lstlisting}[language=scala3]
def betterF(x: Int, counter: Int): (Int, Int) =
  val newCounter = counter + 1
  val result = x + newCounter
  (result, newCounter)
\end{lstlisting}
  Quindi lo stato dovrà essere passato in maniera esplicita da una chiamata a funzione alla successiva:
  \begin{lstlisting}[language=scala3, label=lst:use-better-f]
def useBetterF: (Int, Int) =
  val startingCounter = 1
  val (result1, counter1) = betterF(1, startingCounter)
  // Il nuovo stato counter1 viene passato alla seconda
  // chiamata di betterF
  val (result2, counter2) = betterF(1, counter1)
  // Il nuovo stato counter2 viene passato alla terza
  // chiamata di betterF
  val (result3, finalCounter) = betterF(1, counter2)
  (result1 + result2 + result3, finalCounter)
\end{lstlisting}
  La funzione non solo prende in input lo stato a cui deve accedere ma restituisce in output la nuova versione dello stato modificato\footnote{In un linguaggio con strutture dati mutabili come Scala la funzione potrebbe anche non restituire la nuova versione dello stato ma modificare lo stato ricevuto come argomento per riferimento. Tuttavia, come descritto in precedenza anche la modifica degli argomenti è un side effect; questo approccio non risolverebbe il problema del poter tracciare esplicitamente gli effetti di una funzione.}, a questo punto è nuovamente possibile modellarla come una funzione matematica $f : \mathbb{Z}\times\mathbb{Z} \rightarrow \mathbb{Z}\times\mathbb{Z}$ che, per ogni coppia di valori ricevuti in input, darà sempre lo stesso risultato.

In questo modo diventa più facile testare il comportamento della funzione:
\begin{lstlisting}[language=scala3]
def testBetterF(): Unit =
  val counter = 0
  val (result, newCounter) = betterF(1, counter)
  result shouldBe 2
  newCounter shouldBe 1
\end{lstlisting}
Il test è completamente autonomo, non sono più necessarie operazioni preliminari di preparazione dello stato: l'output di \lstinline{betterF}, infatti, dipende unicamente dai valori passati in input e non da una qualche variabile globale che potrebbe essere utilizzata e modificata anche da altri test.

\subsection{Svantaggi della modellazione esplicita}
\label{svantaggi-della-modellazione-esplicita}
Con le trasformazioni elencate in precedenza è possibile mantenere la trasparenza referenziale delle funzioni modellandone i side effect in maniera esplicita. Questo ha il vantaggio di aiutare il programmatore nella composizione delle funzioni rendendo chiaro, a tempo di compilazione, quale potrebbe essere il loro comportamento.

Nonostante ciò, come forse si è già potuto intuire da alcuni degli esempi riportati, questo approccio rende necessario lo scrivere codice spesso molto più verboso.
In seguito sono riportati due esempi di questo problema e delle soluzioni ad hoc che possono essere adottate per porvi rimedio.

\subsubsection{Gestione del fallimento di una funzione}
Consideriamo la seguente funzione:
\begin{lstlisting}[language=scala3]
def halve(n: Int): Option[Int] =
  n % 2 match
    case 0 => Some(n / 2)
    case _ => None 
\end{lstlisting}
Nel caso in cui il numero passato come input sia pari ne restituirà la metà, altrimenti fallirà (il side effect del fallimento è reso esplicito tramite l'uso del tipo \lstinline{Option}). Per realizzare una funzione analoga che, se l'input è divisibile per 8 ne restituisca il risultato della divisione, un programmatore potrebbe comporre insieme   più chiamate a \lstinline{halve}:
\begin{lstlisting}[language=scala3]
def eighth(n: Int): Option[Int] =
  halve(n) match
    case None       => None
    case Some(half) => halve(half) match
      case None         => None
      case Some(fourth) => halve(fourth)
\end{lstlisting}
Nonostante la semplicità della funzione, si può osservare come il dover gestire in maniera esplicita il fallimento di ogni chiamata ad \lstinline{halve} renda il codice più verboso e meno leggibile. Nella funzione si mescolano la logica applicativa -- il dover dividere ripetutamente per due -- e la gestione del possibile fallimento della divisione intera -- il \emph{pattern matching} sul risultato.

Il problema può essere risolto osservando una struttura comune a tutti i passaggi: se uno qualunque dei risultati intermedi restituisce \lstinline{None} allora la funzione fallisce immediatamente restituendo \lstinline{None}, come se si fosse verificata un'eccezione; altrimenti, si prosegue continuando a dividere il valore ottenuto.
Questo comportamento può essere fattorizzato in un'apposita funzione\footnote{In Scala per poter utilizzare \lstinline{andThen} come una funzione infissa questa deve essere dichiarata come \emph{extension method}~\cite{cit:scala-extension-methods} degli oggetti di tipo \lstinline{Option}.}:
\begin{lstlisting}[language=scala3]
extension [A](a: Option[A])
	def andThen[B](f: A => Option[B]): Option[B] =
		a match
			case None         => None
			case Some(result) => f(result)
\end{lstlisting}
che permetterà di scrivere la funzione \lstinline{eighth} in maniera molto più chiara:
\begin{lstlisting}[language=scala3]
def eighth(n: Int): Option[Int] =
  halve(n).andThen(halve).andThen(halve)
\end{lstlisting}
La funzione è solo interessata dalla logica applicativa: dividere tre volte il valore in input. La gestione del fallimento è delegata ad \lstinline{andThen}\footnote{In Scala è già definito per \lstinline{Option} il metodo \lstinline{flatMap} con lo stesso comportamento di \lstinline{andThen} mostrato nell'esempio. Per questi esempi si preferisce utilizzare \lstinline{andThen} in quanto rende più chiaro il ruolo della funzione.} che, in caso di fallimento, restituisce immediatamente \lstinline{None}.

\nicetohave{Per completezza si potrebbbe anche mostrare come alcuni linguaggi (come Kotlin, Rust) adottano un approccio analogo ma con un supporto diretto del compilatore e annesso zucchero sintattico? In questo caso la soluzione è ad hoc e non è generica come per le monadi ma comunque interessante}

\subsubsection{Gestione della lettura e modifica di uno stato globale}
\label{gestione-della-lettura-e-modifica-di-uno-stato-globale}

Come già discusso, è possibile modificare una funzione che accede e modifica uno stato globale in una funzione pura; in modo che riceva lo stato globale come input e ne restituisca la nuova versione in output.
Lo svantaggio di tale approccio sta nel fatto che il programmatore dovrà gestire manualmente il passaggio dello stato fra le diverse chiamate; ciò, oltre ad aumentare il \emph{boilerplate,} rende più probabile compiere errori come dimenticare di passare lo stato aggiornato a una chiamata successiva:
\begin{lstlisting}[language=scala3]
...
val (res1, state1) = statefulFunction1(initialState)
val (res2, state2) = statefulFunction2(res1, state1)
val (res3, state3) = statefulFunction3(res2, state2)
statefulFunction4(res3, state2)
// bug: statefulFunction4 ha ricevuto come input lo
// stato precedente a quello aggiornato da 
// statefulFunction3!
...
\end{lstlisting}

Anche in questo caso si può ottenere una soluzione che permetta di separare la logica applicativa dalla gestione manuale del passaggio dello stato fra un chiamata e la successiva:
\begin{lstlisting}[language=scala3]
extension [A, S](f: S => (A, S))
  def andThen[B](g: A => S => (B, S)): S => (B, S) =
    state0 => 
      val (result, state1) = f(state0)
      g(result)(state1)
\end{lstlisting}
Grazie a questa funzione è possibile riscrivere la funzione dell'esempio precedente in maniera più chiara:
\begin{lstlisting}[language=scala3]
...
statefulFunction1
  .andThen(statefulFunction2)
  .andThen(statefulFunction3)
  .andThen(statefulFunction4)
  .apply(initialState)
...
\end{lstlisting}
La gestione del \emph{threading} dello stato fra una chiamata e la successiva è delegato ad \lstinline{andThen}; in questo modo è possibile indicare la sequenza di operazioni che si vogliono compiere e lasciare che il boilerplate relativo al passaggio dello stato fra una chiamata e la successiva venga gestito automaticamente.

\section{Modellazione degli effetti tramite monadi}
Osservando le soluzioni adottate nella Sezione precedente si può osservare come a queste sottende un meccanismo comune. Infatti in entrambi i casi è stata definita una funzione \lstinline{andThen} per permettere di combinare in sequenza più operazioni con side effect. Il risultato ottenuto è una descrizione dichiarativa della sequenza di operazioni da svolgere; il particolare meccanismo con cui le operazioni vengono concatenate -- gestione del fallimento prematuro, o passaggio implicito dello stato ad ogni passaggio -- viene delegato alla funzione \lstinline{andThen}.

Questo meccanismo può essere catturato dall'astrazione delle monadi.

\subsection{Cos'è una monade?}
\label{cos-e-una-monade}
Il concetto di monade, nato nell'ambito della teoria delle categorie~\cite{cit:categories-for-the-working-mathematician}, venne utilizzato da Eugenio Moggi come mezzo per strutturare la semantica denotazionale di aspetti di un programma come lo stato mutabile, la gestione delle eccezioni e delle continuazioni~\cite{cit:an-abstract-view-of-programming-languages}.
Fu poi Philip Wadler, ispirato dal lavoro di Moggi e Michael Spivey~\cite{cit:a-functional-theory-of-exceptions}, a intuire che questa stessa tecnica potesse essere sfruttata direttamente per \emph{strutturare} un programma funzionale -- non solo per descriverne la semantica come fatto da Moggi~\cite{cit:comprehending-monads,cit:the-essence-of-functional-programming}.

Una monade è una tripla \lstinline{(M, pure, flatMap)} dove:
\begin{itemize}
  \item \lstinline{M} è un costruttore di tipi; ovvero prende in input un tipo \lstinline{A} e restituisce un tipo \lstinline{M[A]}. Un valore di tipo \lstinline{M[A]} può essere interpretato come una computazione che restituisce un valore di tipo \lstinline{A} e può avere un qualche side effect
  \item \lstinline{pure} è una funzione polimorfa con tipo \lstinline{A => M[A]}\footnote{\lstinline{pure} è spesso anche indicato come \lstinline{return}.}
  \item \lstinline{flatMap} è una funzione polimorfa con tipo \lstinline{(M[A], A => M[B]) => M[B]}\footnote{\lstinline{flatMap} è anche indicato come \lstinline{bind} o \lstinline{>>=}. Nella sua versione \lstinline{>>=} viene generalmente utilizzato come operatore binario infisso: vale a dire che \lstinline{m >>= f} è equivalente a indicare \lstinline{flatMap(m, f)}.}; rappresenta la combinazione in sequenza di due computazioni che possono presentare side effect
\end{itemize}
Inoltre è richiesto che valgano le seguenti \emph{leggi monadiche}:
\begin{itemize}
  \item (Identità sinistra) \lstinline{pure(a) >>= f = f(a)}
  \item (Identità destra) \lstinline{m >>= pure = m}
  \item (Associatività) \lstinline{(m >>= f) >>= g = m >>= (x => f(x) >>= g)}
\end{itemize}
Le prime due leggi servono a garantire che \lstinline{pure} sia l'elemento neutro per l'operazione di concatenazione \lstinline{flatMap}: \lstinline{pure} può essere quindi visto come l'operazione che trasforma un valore di tipo \lstinline{A} in un valore di tipo \lstinline{M[A]} senza compiere alcun side effect.
La terza legge, garantisce che \lstinline{flatMap} sia associativa, dunque scrivere \lstinline{m >>= f >>= g} è equivalente a scrivere \lstinline{(m >>= f) >>= g} o \lstinline{m >>= (x => f(x) >>= g)}.

\subsection{Encoding di una monade}
È possibile esprimere le soluzioni ad hoc adottate alla \Cref{svantaggi-della-modellazione-esplicita} in termini del concetto di monade, garantendo un'interfaccia uniforme per diversi side effect.
Tuttavia, prima di poter generalizzare gli esempi precedenti è necessario capire come il concetto di monade possa essere implementato in un linguaggio di programmazione.

\subsubsection{Encoding in Haskell}
Come descritto in precedenza una monade è composta da tre elementi fondamentali: un costruttore di tipo e due funzioni \lstinline{return} e \lstinline{>>=}. In Haskell è possibile esprimere direttamente tale concetto tramite l'uso di una \emph{type class}, un meccanismo utilizzato per supportare il \emph{polimorfismo ad hoc}~\cite{cit:type-classes-in-haskell}:
\begin{lstlisting}[language=haskell]
class Monad m where
  return :: a -> m a
  (>>=)  :: m a -> (a -> m b) -> m b
\end{lstlisting}

La dichiarazione di una \emph{type class} può essere interpretata come un predicato su un tipo o, come in questo caso, su un costruttore di tipi.
Quindi, la precedente definizione può essere letta come: \emph{``Un generico costruttore di tipi \lstinline{m} è una monade se esistono due funzioni \lstinline{return} con tipo \lstinline{a -> m a} e \lstinline{>>=} con tipo \lstinline{m a -> (a -> m b)   -> m b}''}.
Dato un concreto costruttore di tipi si può istanziare la \emph{type class} \lstinline{Monad} fornendo l'implementazione delle funzioni che questa dichiara:
\begin{lstlisting}[language=haskell]
instance Monad Maybe where
  return = Just
  Nothing >>= f = Nothing
  Just x  >>= f = f x
\end{lstlisting}

Interpretando la \emph{type class} come un predicato, definire un'istanza consiste nel provare che lo specifico tipo -- in questo caso \lstinline{Maybe} -- soddisfa tale predicato e viene fornita come dimostrazione l'implementazione delle funzioni \lstinline{return} e \lstinline{>>=}.

\subsubsection{Encoding in Scala}
In Scala è possibile codificare una \emph{type class} sfruttando il passaggio implicito di parametri~\cite{cit:type-classes-as-objects-and-implicits}. La definizione di una \emph{type class} si riduce quindi a indicare un'interfaccia:
\scalaFromFile{8}{10}{monads/Monad.scala}

Potrà essere istanziata per uno specifico costruttore di tipi fornendo un'implementazione come istanza implicita~\cite{cit:scala-book-type-classes}:
\scalaFromFile{15}{21}{monads/Monad.scala}

\subsection{Esempi di monadi}
Gli esempi mostrati alla \Cref{svantaggi-della-modellazione-esplicita} presentano una struttura comune: si definisce una struttura dati che descrive il risultato di un'operazione con il possibile side effect e si definisce poi una funzione -- negli esempi chiamata \lstinline{andThen} -- che permette di combinare in sequenza passaggi intermedi.

Si può osservare come in entrambi i casi la funzione \lstinline{andThen} abbia la stessa firma descritta per \lstinline{flatMap}: infatti, i casi mostrati in precedenza non sono altro che esempi di monadi. In seguito viene formalizzata la definizione di monade per entrambi gli esempi riportando in aggiunta l'implementazione di \lstinline{pure}. Inoltre viene mostrata la definizione di una monade banale che non esegue alcun side effect.

\subsubsection{La monade identità}
\label{la-monade-identita}
La più semplice monade possibile è quella che non applica alcun side effect. Tale monade può essere definita come segue:
\scalaFromFile{3}{10}{monads/Identity.scala}

\begin{itemize}
  \item \lstinline{Identity} è il costruttore di tipi: preso un tipo \lstinline{A} restituisce un tipo \lstinline{Identity[A]} che rappresenta una computazione che produce un valore di tipo \lstinline{A} senza attuare alcun side effect
  \item \lstinline{pure} permette di trasformare un valore di tipo  \lstinline{A} in uno di tipo \lstinline{Identity[A]}. In questo caso corrisponde alla funzione identità che non modifica il valore di tipo \lstinline{A}
  \item \lstinline{flatMap} permette di mettere in sequenza valori di tipo \lstinline{Identity}; dato che la monade non introduce alcun side effect corrisponde all'applicazione di funzione
\end{itemize}

Una dimostrazione del rispetto delle leggi monadiche è riportata in \Cref{dimostrazione-per-la-monade-identita}.

L'utilità di una monade che non introduce alcun side effect sarà resa evidente nel \Cref{chapter:stack-di-monadi}.

\subsubsection{La monade Option}
\label{la-monade-optional}
Per gestire il fallimento prematuro di una funzione era stato sfruttato il costruttore di tipi \lstinline{Option} e la funzione \lstinline{andThen} per concatenare in sequenza passaggi intermedi e propagare il fallimento.
Si può mostrare come \lstinline{Option} sia una monade implementando l'operazione \lstinline{>>=} come \lstinline{andThen}:
\begin{lstlisting}[language=scala3]
enum Option[+A]:
  case Some(a: A)
  case None

given Monad[Option] with
	def pure[A](a: A): Option[A] = Some(a)
	extension [A](m: Option[A]) 
		def flatMap[B](f: A => Option[B]): Option[B] =
			m match
        case None    => None
				case Some(a) => f(a)
\end{lstlisting}

\begin{itemize}
  \item \lstinline{Option} è il costruttore di tipi: preso un tipo \lstinline{A}, restituisce un tipo \lstinline{Option[A]} che rappresenta una computazione che può fallire o produrre un valore di tipo \lstinline{A}
  \item \lstinline{pure} permette di trasformare un valore di tipo  \lstinline{A} in uno di tipo \lstinline{Option[A]} senza avere side effect. In questo caso il side effect sarebbe fallire restituendo \lstinline{None}; quindi \lstinline{pure(a)} restituisce \lstinline{Some(a)}
  \item \lstinline{flatMap} permette di concatenare in sequenza passaggi intermedi propagando il fallimento
\end{itemize}

Come descritto in precedenza, perché \lstinline{Option} sia effettivamente una monade deve rispettare le tre leggi monadiche; una dimostrazione è riportata all'\Cref{dimostrazione-per-la-monade-optional}.

\subsubsection{La monade State}
\label{la-monade-state}

È possibile definire un'istanza di monade anche per l'esempio dello stato globale mutabile mostrato alla \Cref{gestione-della-lettura-e-modifica-di-uno-stato-globale}; per fare ciò può tornare utile definire prima un'apposita struttura che incapsula una funzione che prende in input lo stato e restituisce il risultato e lo stato aggiornato:
\scalaFromFile{5}{5}{monads/State.scala}

Un problema nel definire l'istanza di monade per \lstinline{State} sta nel fatto che questo è un costruttore di tipi che accetta in input \emph{due tipi} \lstinline{S} e \lstinline{A} -- quello dello stato manipolato e quello del risultato -- e restituisce un tipo \lstinline{State[S, A]}. Per poter essere un costruttore valido secondo la definizione di monade deve prendere in input un solo tipo; per ovviare a tale problema si può fissare uno dei due tipi e lasciare l'altro libero: in questo caso si è scelto di fissare il tipo dello stato \lstinline{S} e lasciare libero di variare il tipo del risultato\footnote{In Scala 3 si può ottenere questa applicazione parziale del costruttore di tipi utilizzando il meccanismo delle \emph{type lambda}~\cite{cit:scala-reference-type-lambdas}; quindi per fissare il tipo \lstinline{S} e lasciare libero \lstinline{A} in \lstinline{State} si può utilizzare la seguente sintassi: \lstinline{[A] =>> State[S, A]}. Nel codice riportato nell'esempio viene utilizzata una sintassi abbreviata che utilizza il carattere \lstinline{_} come segnaposto nella lambda a livello di tipi in maniera analoga a come viene utilizzato per le lambda a livello di termini~\cite{cit:scala-reference-wildcard-arguments-in-types}. Tale sintassi non è ancora stata adottata come default in Scala 3 ma verrà introdotta in futuro, al momento è disponibile utilizzando l'estensione del compilatore \lstinline{"-Ykind-projector:underscores"}~\cite{cit:scala-reference-kind-projector-migration}.}:
\scalaFromFile{11}{18}{monads/State.scala}

\begin{itemize}
  \item \lstinline{State[S, _]} è il costruttore di tipi: preso un tipo \lstinline{A} restituisce un tipo \lstinline{State[S, A]} che rappresenta una computazione che può modificare uno stato globale di tipo \lstinline{S} e restituisce un valore di tipo \lstinline{A}
  \item \lstinline{pure} permette di trasformare un valore di tipo  \lstinline{A} in uno di tipo \lstinline{State[S, A]} senza avere side effect. Il side effect sarebbe modificare lo stato globale, quindi in questo caso lo stato globale viene restituito inalterato e il risultato è il valore passato in input
  \item \lstinline{flatMap} permette di concatenare in sequenza operazioni che operano su uno stato globale mutabile di tipo \lstinline{S} e passa in automatico la sua versione aggiornata da un una chiamata alla successiva
\end{itemize}

Come mostrato nell'\Cref{dimostrazione-per-la-monade-state} l'istanza di monade per \lstinline{State} rispetta le tre leggi monadiche.

Nel caso della monade \lstinline{State} alcune operazioni di base sono quelle che permettono di leggere o modificare il valore dello stato mutabile:
\scalaFromFile{8}{9}{monads/State.scala}
Queste funzioni possono essere utilizzate come base per ottenere operazioni più complesse. Un esempio più articolato che combina queste operazioni è dato dalla seguente funzione \lstinline{incrementCounter}: questa ha come tipo \lstinline{State[Int, String]} dunque può accedere a uno stato mutabile di tipo \lstinline{Int} e produce come risultato una stringa:
\scalaFromFile{21}{28}{monads/State.scala}
Come prima azione accede allo stato globale con \lstinline{get}, ne aumenta il valore con \lstinline{set} e infine legge nuovamente il valore dopo averlo modificato con un'ultimo \lstinline{get}. Il risultato è una stringa contenente il nuovo valore appena letto.
Si noti come, per mettere in sequenza le operazioni di lettura e scrittura sia necessario utilizzare \lstinline{flatMap}. Per poter eseguire la computazione con uno specifico stato iniziale sarà sufficiente utilizzare il metodo \lstinline{runState}:
\begin{lstlisting}[language=scala3]
incrementCounter.runState(0) // -> ("counter is: 1", 1)
incrementCounter.runState(2) // -> ("counter is: 3", 3)
\end{lstlisting}

\subsection{Zucchero sintattico per codice monadico}
Osservando l'esempio precedente è possibile notare come la messa in sequenza delle operazioni tramite l'uso di \lstinline{flatMap} può rendere il codice più complesso da leggere: infatti, ad ogni concatenazione successiva aumenta il livello di annidamento del codice portando alla cosiddetta \emph{pyramid of doom}.
Scrivere codice di questo genere diventerebbe impraticabile molto rapidamente anche per brevi sequenze di funzioni concatenate con \lstinline{flatMap}.

Per questo motivo, linguaggi come Haskell e Scala forniscono dello zucchero sintattico che permette di scrivere codice monadico in modo più leggibile e con un aspetto più ``imperativo''.

\subsubsection{\emph{For comprehension} in Scala}
\label{sec:for-comprehension-in-scala}
Per risolvere questo problema, Scala fornisce la \emph{for comprehension}~\cite{cit:scala-book-control-structures}; per esempio la funzione \lstinline{incrementCounter} mostrata in precedenza può essere scritta in modo equivalente come segue:
\scalaFromFile{30}{35}{monads/State.scala}

Il codice appare come una sequenza di operazioni imperative. In realtà, il compilatore Scala traduce il codice in una sequenza di chiamate a \lstinline{flatMap} e \lstinline{pure}.
Le regole adottate per il \emph{desugaring} possono essere descritte ricorsivamente come segue\footnote{In realtà nella \emph{for comprehension} sarebbe possibile utilizzare anche altre espressioni che non siano nella forma \lstinline{name <- expr} ma, per semplicità, non verranno considerate in questa sezione.}:

\begin{lstlisting}
for {name <- expr} yield res = expr.map(name => res)

for {name <- expr; exprs} yield res =
  expr.flatMap(name => for {exprs} yield res)
\end{lstlisting}

La funzione \lstinline{map} utilizzata nel \emph{desugaring} di una singola espressione è semanticamente equivalente, per una monade, alla seguente composizione di di \lstinline{flatMap} e \lstinline{pure}: \lstinline{m.map(f) = m.flatMap(x => pure(f(x)))}.

\subsubsection{\emph{Do notation} in Haskell}
Haskell adotta una soluzione analoga fornendo la cosiddetta \emph{do notation}; il precedente codice Scala preso ad esempio potrebbe essere implementato analogamente in Haskell come segue:
\begin{lstlisting}[language=haskell]
incrementCounter = do
  counter <- get
  set (counter + 1)
  newCounter <- get
  return ("counter is:" ++ show newCounter)
\end{lstlisting}
Anche in questo caso, il compilatore traduce il codice in una serie di chiamate a \lstinline{>>=} e \lstinline{return}. In particolare, le regole per la traduzione sono le seguenti:
\begin{lstlisting}
do { expr } = expr

do { name <- expr; exprs } =
  expr >>= \name -> do { exprs }

do { expr; exprs } = expr >>= \_ -> do { exprs }
\end{lstlisting}

Quindi, la forma senza zucchero sintattico di \lstinline{incrementCounter} sarebbe:
\begin{lstlisting}[language=haskell]
incrementCounter =
  get >>= (\counter ->
    set (counter + 1) >>= (\_ ->
      get >>= (\newCounter ->
        return ("counter is:" ++ show newCounter))))
\end{lstlisting}

\subsection{Vantaggi nell'uso delle monadi}
\subsubsection{Separazione del codice impuro dal codice puro}
Un primo importante vantaggio sta nella possibilità di esprimere a livello di \emph{type system} quali funzioni presentano side effect e quali no. Questo è un aiuto fondamentale per il programmatore: infatti, per capire se una funzione è pura è sufficiente analizzarne il tipo, senza dover cercare di capirlo dal suo nome -- che potrebbe non rispecchiare l'effettivo comportamento della funzione -- o ispezionandone il corpo.
Consideriamo come esempio le seguenti funzioni:
\begin{lstlisting}[language=scala3]
def incrementCounter: State[Int, ()] = ...
def first[A](xs: List[A]): Option[A] = ...
def double(n: Int): Int = ...
\end{lstlisting}
Senza bisogno di conoscerne le implementazioni si può capire che la prima funzione può modificare uno stato mutabile di tipo intero e che la seconda funzione può fallire non producendo alcun valore. Inoltre è possibile capire che la terza funzione non ha side effect: non potrà fallire, modificare uno stato globale o effettuare operazioni di input o output; il suo output sarà determinato unicamente dal valore dei suoi parametri\footnote{In realtà, dato che in Scala è comunque possibile scrivere codice impuro la funzione \lstinline{double} potrebbe avere side effect; quanto detto vale sotto l'assunzione che il programmatore stia modellando esplicitamente i side effect tramite l'approccio descritto. In altri linguaggi come Haskell, invece, è il compilatore stesso a fare in modo che questa regola venga rispettata rendendo impossibile lo scrivere funzioni che non siano pure. Perciò, si avrebbe la certezza che una funzione come \lstinline{double} non possa avere side effect.}.

\subsubsection{Side effect come cittadini di prima classe del linguaggio}
Utilizzare il concetto di monade come astrazione unificante delle diverse tipologie di side effect ha un ulteriore vantaggio: le azioni con side effect sono valori di prima classe che possono essere combinati in maniera modulare e astraendo dallo specifico tipo di effetti.

È possibile definire l'equivalente di strutture di controllo imperative tramite funzioni generiche; per poter ripetere più volte gli effetti di un'azione è possibile implementare una funzione come segue:
\begin{lstlisting}[language=scala3]
extension [A, M[_]: Monad](m: M[A])
  def >>[B](other: M[B]) = m.flatMap(_ => other)

def repeat[M[_]: Monad, A]
  (times: Int)
  (action: M[A]): M[Unit] =
	times match
		case 0 => pure(())
		case n => action >> repeat(n-1)(action)
\end{lstlisting}

Se invece si volesse implementare una versione del ciclo \lstinline{for} che permetta di ripetere un qualunque side effect per tutti gli elementi di una lista si potrebbe implementare la seguente funzione:
\begin{lstlisting}[language=scala3]
def forLoop[A, M[_]: Monad]
  (as: List[A])
  (f: A => M[Unit]): M[Unit] = 
  as match 
    case Nil     => pure(())	
    case a :: as => f(a) >> forLoop(as)(f)
\end{lstlisting}
\lstinline{forLoop} prende in input una lista e una funzione che, dato un elemento della lista, determina l'azione da intraprendere; il risultato sarà un'unica computazione che esegue tutti i side effect dati dall'applicazione della funzione a ciascun elemento della lista. Poiché la funzione è generica sul tipo della monade è possibile utilizzarla per qualsiasi tipo di side effect!
Questa è una tecnica molto potente che lascia al programmatore la libertà di inventare le proprie strutture di controllo senza essere doversi limitare a quelle predefinite dal linguaggio~\cite{cit:tackling-the-awkward-squad}.

\section{Input e output puri}
\label{section:input-e-output-puri}

Nelle precedenti sezioni è stato mostrato come sia possibile ``simulare'' la presenza di side effect -- come eccezioni e modifica di uno stato mutabile -- con opportune modifiche al tipo delle funzioni in modo da rendere esplicito il fatto che queste possano avere side effect.
Inoltre, è stato evidenziato come il concetto di monade permetta di fornire un meccanismo comune per la modellazione e messa in sequenza di tali side effect.

Tuttavia, fino ad ora è stato tralasciato un side effect fondamentale: l'esecuzione di input e output.
Chiaramente una funzione Scala potrebbe effettuare input e output semplicemente utilizzando le funzioni standard fornite dal linguaggio:
\begin{lstlisting}[language=scala3]
def addTo(x: Int): Int =
	val y = scala.io.StdIn.readInt() // side effect!
	x + y
\end{lstlisting}\label{code:addToScalaImpure}
Tuttavia, dalla sola analisi del tipo della funzione \lstinline{sum : Int => Int} non è possibile capire se questa interagirà con il mondo esterno o meno.

Per comprendere come sia possibile tracciare tale side effect a livello di tipi verrà preso come esempio paradigmatico il linguaggio Haskell; successivamente verrà mostrato come le stesse intuizioni possano essere applicate in Scala.

\subsection{Modello di valutazione \emph{lazy}}
Haskell è un linguaggio funzionale con strategia di valutazione \emph{lazy} (anche detta \emph{call-by-need}): ciò significa che gli argomenti delle funzioni vengono valutati solo se strettamente necessario e non sono valutati prima di essere passati alla funzione. Si consideri per esempio la seguente funzione:
\begin{lstlisting}[language=haskell]
lazy :: Int -> Int -> Int
lazy x y = x * 3
\end{lstlisting}
Quando la funzione viene chiamata, anziché valutare i suoi argomenti prima di eseguire il corpo della funzione, vengono allocati due \emph{thunk} che rappresentano le espressioni da valutare. Sarà poi il corpo della funzione, in base al bisogno, a stabilire di quali \emph{thunk} forzare la valutazione. Nell'esempio specifico valutato solo il \emph{thunk} del primo argomento. Si considerino le possibili chiamate alla funzione \lstinline{lazy}:
\begin{lstlisting}[language=haskell]
lazy (2 + 3) (expensiveFunction 2)
lazy (2 + 3) (1 + undefined)
\end{lstlisting}
Nel primo esempio il \emph{thunk} che rappresenta l'espressione \lstinline{expensiveFunction 2} non verrà mai valutato; allo stesso modo nella seconda chiamata il \emph{thunk} dell'espressione \lstinline{1 + undefined}\footnote{In Haskell il valore \lstinline{undefined} è un valore speciale che comporta il crash dell'applicazione nel momento in cui viene valutato. In questo caso, dato che si trova in un \emph{thunk} che verrà scartato senza essere valutato non verrà sollevata alcuna eccezione.} non sarà valutato; il risultato sarà 15 in entrambi i casi.

Grazie a questa strategia di valutazione è possibile definire direttamente operatori con \emph{short-circuiting} come \lstinline{&&} e \lstinline{||}:
\begin{lstlisting}[language=haskell]
(&&) :: Bool -> Bool -> Bool
x && y = case x of
  True  -> y
  False -> False 

(||) :: Bool -> Bool -> Bool
x || y = case x of
  True  -> True
  False -> y
\end{lstlisting}
Le due funzioni hanno una struttura simile: inizialmente viene forzata la valutazione del primo argomento tramite \emph{pattern matching} sul suo valore. In seguito viene restituito il secondo argomento o un valore predefinito in base al risultato del pattern matching. In entrambi i casi non viene forzata la valutazione del secondo argomento attuando la logica di \emph{short-circuiting} che ci si potrebbe aspettare dagli operatori logici \lstinline{&&} e \lstinline{||}: il risultato di \lstinline{False && undefined} sarà \lstinline{False} e il programma non terminerà con un'eccezione dato che l'espressione \lstinline{undefined} non sarà valutata.

\subsubsection{Incompatibilità di \emph{laziness} e side effect}
Un aspetto negativo della strategia \emph{lazy} è che può diventare estremamente complesso capire l'ordine con il quale le espressioni vengono valutate. Infatti, come mostrato negli esempi precedenti la valutazione dei \emph{thunk} viene forzata solo quando strettamente necessario. Per questo motivo sarebbe pressoché impossibile riuscire a compiere in una sequenza prevedibile i side effect delle funzioni~\cite{cit:tackling-the-awkward-squad}.\nicetohave{Forse si potrebbe aggiungere un esempio più complesso in un listato a parte che mostri la difficoltà di avere side effect in un contesto lazy? Per esempio con lettura e chiusura di file handle}

È proprio questa caratteristica che ha fatto sì che Haskell rimanesse un linguaggio puro e ha portato all'invenzione dell'input e output monadico: ``forse il più grande beneficio della \emph{laziness} non è la \emph{laziness} in sè, quanto il fatto che ci abbia forzato a rimanere puri, motivando così una grande quantità di lavoro sulle monadi''\footnote{Traduzione dal testo originale: ``[...] perhaps the biggest single benefit of laziness is not laziness per se, but rather that laziness kept us pure, and thereby motivated a great deal of productive work on monads [...]''~\cite{cit:a-history-of-haskell-being-lazy-with-class}}.

\subsection{I/O monadico in Haskell}
\label{sub:io-monadico-haskell}
La soluzione adottata da Haskell per permettere di effettuare operazioni di I/O è quello di fornire un nuovo costruttore di tipi chiamato \lstinline{IO} che sia una monade. Un valore di tipo \lstinline{IO a} modella una computazione che produce un valore di tipo \lstinline{a} e può avere il side effect di interagire con il sistema, per esempio effettuando operazioni di input o output.

Il programmatore potrà definire nuove operazioni combinando le funzioni di libreria fornite dal linguaggio sfruttando l'interfaccia delle monadi:
\begin{lstlisting}[language=haskell]
echo :: IO ()
echo = do
  line <- getLine
  putStrLn line
\end{lstlisting}
Partendo dal tipo della funzione si può comprendere come questa rappresenti una computazione che una volta eseguita restituisce un valore di tipo \lstinline{()} e che può effettuare I/O. La funzione è implementata mettendo in sequenza due operazioni più semplici: prima viene letta una riga dallo \emph{standard input} e il contenuto letto viene stampato sullo \emph{standard output} tale e quale.
È interessante osservare come la \emph{do notation} nasconda l'applicazione di \lstinline{>>=} e \lstinline{pure} dando al codice un tipico aspetto imperativo. Nonostante l'apparente ``imperatività'' è fondamentale ricordare che i valori di tipo \lstinline{IO} -- come \lstinline{getLine} e \lstinline{putStrLn} -- non sono funzioni che provocano side effect ma descrivono i side effect che devono avere luogo. L'equivalente versione senza zucchero sintattico è:
\begin{lstlisting}[language=haskell]
echo :: IO ()
echo = getLine >>= putStrLn
\end{lstlisting}

Il punto d'ingresso di ogni programma Haskell è la funzione \lstinline{main}:
\begin{lstlisting}[language=haskell]
main :: IO ()
main = echo
\end{lstlisting}
Quindi un programma non è altro che una struttura dati immutabile che \emph{descrive} la sequenza di operazioni che il \emph{runtime system} del linguaggio deve eseguire a tempo d'esecuzione.

\subsubsection{Separazione di codice puro e impuro}
Grazie all'approccio appena mostrato non è possibile mescolare inavvertitamente codice puro e codice con side effect -- proprio come per i casi di eccezioni e stato mutabile mostrati in precedenza. Per esempio, consideriamo come potrebbe essere riscritta la funzione impura mostrata all'inizio della sezione; semplicemente leggere un valore intero da standard input non è possibile:
\begin{lstlisting}[language=haskell]
readInt :: IO Int
readInt = fmap read getLine

addTo :: Int -> Int
addTo x = let y = readInt in x + y
\end{lstlisting}
Il codice mostrato non compilerebbe in quanto \lstinline{x} ha tipo \lstinline{Int} mentre \lstinline{readInt} è un valore di tipo \lstinline{IO Int}: non è un valore intero bensì una computazione che produrrà un intero. Per far sì che \lstinline{addTo} possa utilizzare una funzione impura come \lstinline{readInt}, è necessario rendere esplicito a livello di tipi il fatto che anche \lstinline{addTo} sia impura:
\begin{lstlisting}[language=haskell]
addTo :: Int -> IO Int
addTo x = do
  y <- getInt
  pure (x + y)
\end{lstlisting}

\subsubsection{Programmi come valori di prima classe}
Come già descritto, un valore di tipo \lstinline{IO a} non è altro che una struttura dati che descrive una sequenza di computazioni per produrre un valore di tipo \lstinline{a}. Ciò permette di passare programmi come valori di prima classe e costruire una ricca serie di funzioni generiche che operano su programmi e producono nuovi programmi in output. Per esempio la funzione
\begin{lstlisting}[language=haskell]
forever :: IO a -> IO b
forever action = action >> forever action
\end{lstlisting}
prende in input un programma e restituisce un programma che lo esegue in loop all'infinito.

Un ulteriore esempio può essere la funzione \lstinline{retry} definita come segue:
\begin{lstlisting}[language=haskell]
retry :: Int -> (a -> Bool) -> IO a -> IO (Maybe a)
retry 0 _ _ = pure Nothing
retry times shouldRetry action = do
  result <- action
  if shouldRetry result
    then retry (times - 1) shouldRetry action
    else pure (Just result)
\end{lstlisting}
Questa restituisce in output un programma che ripete fino a un massimo numero di volte un programma passato in input secondo una certa logica di ripetizione definita dal predicato \lstinline{shouldRetry}.
Queste funzioni possono essere combinate in programmi più complessi\footnote{Nell'esempio per effettuare le richieste a un server viene utilizzata la libreria \emph{http-conduit}~\cite{cit:http-conduit}}:
\begin{lstlisting}[language=haskell]
URLToResource :: String -> IO (Response ByteString)
URLToResource url = httpLBS (fromString url)

shouldRetry :: Response a -> Bool
shouldRetry response =
  let statusCode = getResponseStatusCode response
   in statusCode `elem' [500, 503]

main :: IO ()
main = forever $ do
  let times = 10
  putStrLn "URL of the resource: "
  url <- getLine  
  result <- retry times shouldRetry (URLToResource url)
  case result of
    Nothing -> putStrLn "Failed after 10 retries"
    Just _  -> putStrLn "Got a response"
\end{lstlisting}
Sfruttando la funzione \lstinline{retry} è possibile definire un programma che ripete una richiesta HTTP fino a un massimo di 10 volte in caso di errore 500 o 503\footnote{Ai fini dell'esempio la funzione \lstinline{retry} è piuttosto semplice e non implementa logiche complesse per attendere prima di ripetere una richiesta evitando di oberare il server. Il package \emph{retry}~\cite{cit:retry} implementa una funzione analoga a quella mostrata con la possibilità di specificare delle \emph{policy} per stabilire la strategia con cui riprovare l'azione.}. Utilizzando la funzione \lstinline{forever} è possibile fare in modo che il programma continui a chiedere input al programmatore all'infinito.

Un ulteriore vantaggio dato dal fatto che \lstinline{IO} è una monade sta nella possibilità di sfruttare codice generico sul tipo di monade:

\begin{lstlisting}[language=haskell]
void :: Monad m => m a -> m ()
void = m >>= (\_ -> pure ())

sequence :: Monad m => [m a] -> m [a]
sequence [] = pure []
sequence (m:ms) = do
  a  <- m
  as <- sequence ms
  pure (a:as)

main :: IO ()
main =
  let messages = ["message1", "message2", "message3"]
  in void (sequence (map putStrLn messages))
\end{lstlisting}

In realtà le stesse funzioni \lstinline{forever} e \lstinline{retry} possono essere definite in maniera generica rispetto al tipo di effetto in considerazione:
\begin{lstlisting}[language=haskell]
forever :: Monad m => m a -> m b
retry :: Monad m
  => Int
  -> (a -> Bool)
  -> m a
  -> m (Maybe a)
\end{lstlisting}
L'implementazione è tralasciata in quanto identica a quanto riportato nell'esempio di codice mostrato in precedenza: l'unico cambiamento necessario per rendere la funzione più generica è stato quello di sostituire nel tipo \lstinline{IO} con una generica monade \lstinline{m}.

\subsection{I/O monadico in Scala}
In Haskell la monade \lstinline{IO} deve essere implementata come un tipo di dato opaco con un supporto speciale del compilatore. Haskell infatti è un linguaggio puro con una modalità di valutazione \emph{lazy} che, come mostrato in precedenza, è incompatibile con la presenza di side effect.
In Scala non è necessario un supporto diretto del compilatore in quanto è già possibile definire computazioni che svolgono side effect; perciò una semplice implementazione\footnote{L'implementazione vuole unicamente mostrare come si possa immaginare l'implementazione della monade \lstinline{IO} in un linguaggio come Scala. Tuttavia, un'implementazione simile non è \emph{stack safe:} interpretare un'azione \lstinline{IO} ottenuta componendo molti blocchi di base può portare a un errore di \emph{stack overflow}. Questo problema può essere risolto complicando l'implementazione della monade sfruttando una tecnica nota come \emph{trampolining}~\cite{cit:stackless-scala-with-free-monads} ed è l'approccio adottato da librerie come Cats Effect~\cite{cit:cats-effect-stack-safety}.} della monade \lstinline{IO} potrebbe essere la seguente:
\scalaFromFile{7}{14}{monads/IO.scala}
\begin{itemize}
  \item \lstinline{IO} è un costruttore di tipi che preso in input un valore di tipo \lstinline{A} restituisce un valore di tipo \lstinline{IO[A]}. Un valore con questo tipo rappresenta una computazione che, quando eseguita, può effettuare input o output e restituisce un valore di tipo \lstinline{A}
  \item \lstinline{pure} permette di trasformare un valore di tipo \lstinline{A} in uno di tipo \lstinline{IO[A]}. Non introduce alcun side effect e restituisce semplicemente il valore passato in input
  \item \lstinline{flatMap} permette di concatenare in sequenza due operazioni che possono svolgere I/O. Il risultato sarà una singola computazione che esegue i side effect di ciascuna in sequenza
\end{itemize}

Le funzioni impure della libreria standard di Scala possono quindi essere espresse in termini di \lstinline{IO}:
\scalaFromFile{16}{17}{monads/IO.scala}
\lstinline{putStrLn} e \lstinline{getLine} sono valori di tipo \lstinline{IO}; vale a dire strutture dati immutabili che descrivono come eseguire un side effect nel momento in cui la computazione verrà interpretata. Si noti la differenza fra una computazione \lstinline{IO} e l'equivalente versione impura:
\begin{lstlisting}[language=scala3]
val res1 = putStrLn("Hello, World!")
// res1 : IO[Unit]

val res2 = println("Hello, World!")
// res2 : Unit
// -> "Hello, World!"
\end{lstlisting}
La valutazione di \lstinline{putStrLn} produce un valore di tipo \lstinline{IO[Unit]} senza alcun side effect; d'altro canto, la valutazione di \lstinline{println} produce un valore di tipo \lstinline{Unit} e ha il side effect di stampare in output la stringa.

L'unico modo per poter estrarre un valore dalla monade \lstinline{IO} interpretandone il contenuto è tramite l'uso di \lstinline{unsafeRun}:
\begin{lstlisting}[language=scala3]
val program = for
  _ <- putStrLn("Line 1")
  _ <- putStrLn("Line 2")
yield ()

program.unsafeRun()
// -> Line 1
// -> Line 2
\end{lstlisting}
Il metodo è stato chiamato \lstinline{unsafeRun} a suggerire l'eccezionalità nella sua invocazione. Infatti, l'intero programma dovrebbe essere descritto all'interno della monade \lstinline{IO} per poter poi essere eseguito nel main del programma con un'unica chiamata a \lstinline{unsafeRun}.

I vantaggi ottenuti grazie a questo approccio sono gli stessi già descritti nella \Cref{sub:io-monadico-haskell}: è possibile separare chiaramente il codice impuro dal codice puro, i programmi diventano cittadini di prima classe che possono essere presi come input e restituiti in output e diventa possibile sfruttare tutte le funzioni generiche sul tipo di monade per comporre programmi complessi. Per esempio, come mostrato al \Cref{lst:io-http}, il codice mostrato in precedenza per effettuare delle richieste a un server può essere scritto in Scala in maniera molto simile\footnote{Nell'esempio per effettuare le richieste a un server viene utilizzata la libreria \emph{Requests-Scala}~\cite{cit:requests-scala}}.

\begin{figure}[htp]
  \begin{lstlisting}[language=scala3, caption={Esempio di codice monadico che incapsula i side effect all'interno della monade IO per implementare una politica di \emph{retry} per le richieste HTTP.}, label={lst:io-http}]
    extension[A](m: IO[A])
      def forever[B]: IO[B] = m.flatMap(_ => m.forever)
      def retry(times: Int, shouldRetry: A => Boolean):
        IO[Option[A]] =
        times match
          case 0 => IO.pure(None)
          case n => m.flatMap{ result => 
            if shouldRetry(result)
            then m.retry(n-1, shouldRetry)
            else IO.pure(Some(result))
          }

    def urlToResource(url: String): IO[Try[Response]] =
      IO(() => Try(requests.get(url)))

    def shouldRetry(response: Try[Response]): Boolean =
      response match
        case Failure(exception: RequestFailedException) =>
          val statusCode = exception.response.statusCode
          List(500, 503).contains(statusCode)
        case _ => false

    @main def main: Unit = 
      val times = 10
      val step = for 
        _ <- putStrLn("URL of the resource: ")
        url <- getLine
        result <- URLToResource(url)
                    .retry(times, shouldRetry)
        _ <- result match
          case None => putStrLn("Failed after 10 retries")
          case Some(_) => putStrLn("Got a response")
      yield ()
      step.forever.unsafeRun()
  \end{lstlisting}
\end{figure}

\chapter{Stack di monadi}

Nel capitolo precedente si è mostrato come sia possibile implementare semplici monadi per poter modellare la presenza di diversi side effect.
Tuttavia, seguendo tale approccio non è evidente come sia possibile gestire contemporaneamente più side effect. Infatti ogni monade descritta permette di modellare un singolo side effect per volta: uno stato mutabile per \lstinline{State}, la possibilità di fallimenti con \lstinline{Option} e la capacità di effettuare input e output con \lstinline{IO}.

In questo capitolo verrà introdotto il concetto di \term{monad transformer}: un meccanismo che permette di unire monadi elementari combinandone le caratteristiche.



\chapter{MTL}
\label{chapter:mtl}

\section{Problemi nell'uso dei monad transformers}
\subsection{Lifting manuale delle operazioni}
Come mostrato negli esempi riportati al capitolo precedente, perché un \term{transformer} possa sfruttare gli effetti della monade che va ad arricchire deve ricorrere all'operazione di \term{lifting}. Questa operazione permette di portare all'interno del \term{transformer} un valore dalla monade base senza aggiungere ulteriori side effect.
Dunque, scrivendo codice monadico all'interno di un \term{transformer} il programmatore si troverà spesso a dover ricorrere all'operazione di \term{lifting}:
\begin{lstlisting}[language=scala3]
def manualLifting: StateT[Int, OptionT[IO, _], String] =
  for
    _ <- IO.putStrLn("test").lift[OptionT].lift[StateTFixS[Int]]
    _ <- OptionT.fail[IO, Any].lift[StateTFixS[Int]]
  yield "result"
\end{lstlisting}
Il tipo che descrive tutti i side effect della funzione \lstinline{manualLifting} è \lstinline{StateT[Int, OptionT[IO, _], String]}; alla base dello \term{stack} si trova la monade \lstinline{IO} a cui viene aggiunta la possibilità di fallimento grazie al \term{transformer} \lstinline{OptionT} e di modificare uno stato mutabile.
Per poter eseguire l'operazione di stampa in output il cui tipo è \lstinline{IO[Unit]} è quindi necessario trasformarla in un valore compatibile con lo \term{stack} utilizzato. In questo caso la prima operazione di \term{lifting} trasforma il tipo in \lstinline{OptionT[IO, Unit]}; l'operazione è applicata un'ultima volta per portare il valore nello \term{stack} descritto.

Come è possibile osservare il codice è molto verboso e introduce la necessità di inserire diverse chiamate a \lstinline{lift} il cui unico scopo è quello di far combaciare i tipi delle operazioni.
Il risultato sarà un codice poco leggibile dove la logica applicativa viene offuscata dalla presenza di numerose operazioni di \term{lifting}.
Infatti, specialmente in \term{stack} composti da diversi \term{transformer}, è necessario ricorrere a più operazioni di \term{lifting} per ogni singola azione con side effect. Questo comporta un significativo sbilanciamento fra le porzioni di codice effettivamente rilevanti -- che codificano la logica applicativa -- e il \term{boilerplate} necessario per poter utilizzare correttamente lo \term{stack} di monadi.

Inoltre, il codice scritto utilizzando uno specifico \term{stack} di monadi sarà ``viscoso'': opporrà maggiore resistenza al cambiamento della struttura dello \term{stack} di monadi comportando la riscrittura di diverse porzioni di codice. Per esempio, aggiungendo o rimuovendo un ulteriore \term{transformer} allo \term{stack} utilizzato sarà necessario aggiungere o rimuovere da ogni operazione la chiamata al metodo \lstinline{lift}. Immaginando di rimuovere il \term{transformer} \lstinline{StateT} dalla funzione \lstinline{manualLifting} mostrata in precedenza il codice dovrà essere trasformato in:
\begin{lstlisting}[language=scala3]
def manualLifting: OptionT[IO, String] =
  for
    _ <- IO.putStrLn("test").lift[OptionT]
    _ <- OptionT.fail[IO, Any]
  yield "result"
\end{lstlisting}
In questo caso il cambiamento nel tipo di ritorno della funzione ha comportato il dover modificare tutte le operazioni coinvolte rimuovendo le chiamate a \lstinline{lift}.

Anche il semplice riordinare gli elementi dello \term{stack} di monadi comporta la necessità di apportare modifiche al codice. Sempre riprendendo l'esempio di \lstinline{manualLifting} si immagini di dover cambiare lo \term{stack} di monadi in modo da avere un tipo \lstinline{OptionT[StateT[Int, IO, _], String]}. In questo caso il codice dovrà essere modificato come segue:
\begin{lstlisting}[language=scala3]
  def manualLifting: OptionT[StateT[Int, IO, _], String] =
    for
      _ <- IO.putStrLn("test").lift[StateTFixS[Int]].lift[OptionT]
      _ <- OptionT.fail[StateT[Int, IO, _], Any]
    yield "result"
\end{lstlisting}
Si noti come sia stato necessario modificare l'ordine nelle annotazioni dei tipi del \term{lifting} relativo all'operazione di stampa in output.

\subsection{Principio di privilegio minimo}
Una possibile soluzione al dover intervallare operazioni relative alla logica applicativa con le operazioni di \term{lifting} può consistere nel fissare lo \term{stack} da utilizzare. Una volta stabiliti i side effect di cui l'applicazione avrà bisogno -- e quindi quale \term{stack} di monadi sia necessario utilizzare -- si può realizzare primitive di base che restituiscano valori all'interno di tale \term{stack} e comporre queste per ottenere programmi complessi. Si consideri il seguente esempio: si vuole realizzare un programma che legga un file CSV, ne parsi il contenuto e calcoli la somma dei valori contenuti in una specifica colonna. Immaginando un'API per poter effettuare parsing di file CSV il risultato potrebbe essere il seguente:
\begin{lstlisting}[language=scala3]
  type App = OptionT[IO, _]
  def mainAction: App[Int] = for
    rawData <- readFile("data.csv") // : App[String]
    csv     <- parseCSV(rawData) // : App[CSV]
    column  <- csv.getIntColumn("column") // : App[List[Int]]
  yield column.sum
\end{lstlisting}
La logica applicativa viene espressa in maniera concisa e leggibile. Ciò è reso possibile dal fatto che ogni operazione intermedia restituisce un valore all'interno della monade \lstinline{App} e quindi non è necessario effettuare \term{lifting} delle operazioni per permettere ai tipi di combaciare.

Questo vantaggio dal punto di vista della leggibilità del codice viola però il \emph{principio di privilegio minimo}: il programmatore, per poter esprimere in maniera concisa la logica applicativa, è costretto a incapsulare ogni valore intermedio all'interno dell'intero \term{stack} di monadi dell'applicazione anche qualora ciò non sia necessario. Per esempio, l'operazione di \term{parsing} del contenuto del file potrebbe fallire nel caso in cui questo sia mal formato; tuttavia, il \term{parsing} non richiede di effettuare operazioni di input e output. Nonostante ciò, \lstinline{parseCSV} -- per poter essere composto con le altre operazioni -- restituisce un valore di tipo \lstinline{OptionT[IO, CSV]}. Vale a dire che l'operazione di \term{parsing} potrebbe potenzialmente effettuare input e output anche se dal punto di vista logico ciò non ha senso.
Lo stesso ragionamento vale per la chiamata al metodo \lstinline{getIntColumn} che per esempio potrebbe fallire nel caso in cui non sia presente una colonna col nome desiderato; non c'è alcun motivo per cui tale operazione dovrebbe poter effettuare operazioni di input e output.

Dunque, il tipo di ciascuna delle funzioni intermedie è meno generale di quanto non sia effettivamente necessario per esprimerne il comportamento. Le operazioni \lstinline{parseCSV} e \lstinline{getIntColumn} non effettuano input e output ma il tipo di ritorno non lo impedisce; solo un'analisi dell'implementazione del corpo delle funzioni può permettere di capire se svolgono o meno uno dei determinati effetti previsti dallo \term{stack} di monadi utilizzato.

\subsection{Violazione dell'incapsulamento}
Un ulteriore problema che deriva dall'uso dei \term{monad transformer} sta nel fatto che il codice che utilizza tali \term{stack} di monadi è fortemente legato alla specifica modellazione del side effect che viene utilizzata. Essenzialmente viene violato il \emph{principio di incapsulamento} per cui sarebbe necessario programmare basandosi su un'interfaccia piuttosto che su una specifica implementazione~\cite[p.~94]{cit:clean-code-a-handbook-of-agile-software-craftsmanship}.

Nell'esempio mostrato in precedenza la possibilità di fallimento viene espressa tramite l'uso del transformer \lstinline{OptionT}. Vale a dire che, una volta interpretata, se la computazione dovesse fallire ritornerà un valore \lstinline{None}. In questo modo, il programma è legato alla specifica modellazione del side effect del fallimento che viene fornita dal \term{transformer} \lstinline{OptionT}.
In circostanze differenti si potrebbe voler modificare il modo in cui un side effect viene implementato. Per esempio, nel caso del side effect del fallimento, si potrebbe sfruttare la monade \lstinline{IO} per descrivere una computazione che fallisca con un'eccezione una volta interpretata:
\begin{lstlisting}[language=scala3]
def fail[A]: IO[A] = IO(() => throw Exception())
\end{lstlisting}
Questo è un possibile modo di ottenere il side effect del fallimento sfruttando direttamente la monade \lstinline{IO} senza dover ricorrere al \term{transformer} \lstinline{OptionT}. Nella descrizione della logica applicativa non importa quale sia la concreta implementazione utilizzata per descrivere tale side effect; l'aspetto rilevante è che il fallimento comporti l'interruzione della computazione.

Codificare le operazioni in maniera diretta sfruttando uno specifico \term{stack} di monadi crea un accoppiamento fra la descrizione della logica applicativa e la sua effettiva implementazione.

\section{Monad Transformer Library}

\emph{Monad Transformer Library}\footnote{D'ora in avanti abbreviato in MTL} è una libreria Haskell nata con l'obiettivo di semplificare la gestione di stack di monad transformer~\cite{cit:mtl} permettendo di comporre in maniera modulare computazioni che modellano la presenza di side effect.
Questa libreria è diventata nel tempo uno standard \emph{de facto} per la realizzazione di sistemi complessi nell'ecosistema di Haskell: circa il 30\% dei pacchetti presenti su Hackage\footnote{Hackage è l'archivio di software open source della comunità Haskell utilizzato per pubblicare librerie e programmi.} ne dipendono direttamente~\cite{cit:which-monads-haskell-developers-use-an-exploratory-study} ed è il sesto pacchetto più utilizzato su Stackage\footnote{Stackage è un'infrastruttura utilizzata per mantenere degli \emph{snapshot} di pacchetti Hackage compatibili fra loro per ottenere delle \emph{build} stabili.}~\cite{cit:evolution-of-a-haskell-repository-and-its-use-of-monads-an-exploratory-study-of-stackage}.

L'approccio di MTL è stato poi adottato anche in altri linguaggi; per esempio in Scala nell'ecosistema di Typelevel è stata realizzata la libreria Cats MTL~\cite{cit:cats-mtl}.

\subsection{Idea alla base di MTL}
L'obiettivo di MTL è di permettere la descrizione di computazioni con side effect in maniera flessibile, modulare e facilmente componibile, con la possibilità di cambiare senza difficoltà l'implementazione sottostante utilizzata per gestire i side effect.
Questo obiettivo può essere raggiunto sfruttando nuovamente il meccanismo delle \emph{type class} per definire \emph{famiglie} di effetti, ciascuna delle quali è caratterizzata da una serie di primitive che permettono di esprimere le operazioni che possono essere effettuate~\cite{cit:functional-programming-with-overloading-and-higher-order-polymorphism}.

\subsubsection{Modifica di uno stato mutabile}
Nello stile di MTL il side effect del cambiamento di uno stato mutabile può essere catturato dalla seguente interfaccia:
\scalaFromFile{7}{9}{monads/mtl/State.scala}
L'interfaccia descrive, per un generico costruttore di tipi \lstinline{M[_]}, le operazioni che questo deve fornire per permettere di gestire uno stato mutabile.
Sfruttando il meccanismo dei parametri impliciti di Scala è possibile definire una funzione che accetta come parametro il contesto \lstinline{M[_]} che implementa l'interfaccia \lstinline{State}:
\scalaFromFile{30}{36}{monads/mtl/State.scala}

Inoltre, è possibile modificare la funzione per rendere in maniera più ``dichiarativa'' i side effect che \lstinline{M[_]} deve avere. Innanzitutto, è necessario definire il seguente \emph{type alias}:
\scalaFromFile{6}{6}{monads/mtl/State.scala}
In questo modo, sfruttando lo zucchero sintattico che Scala offre per definire \emph{context bounds}~\cite{cit:context-bounds}, è possibile modificare la firma della funzione precedente:
\begin{lstlisting}[language=scala3]
def update[S, M[_]: Monad: HasState[S]](f: S => S): M[Unit] = ???
\end{lstlisting}
Nella definizione del generico \lstinline{M[_]} è presente una lista di \emph{type class} che ne descrivono le funzionalità; la definizione può essere letta come: \emph{``Dato un qualunque \lstinline{M} che sia una monade e abbia uno stato di tipo \lstinline{S}''}.
Infine, si può aggiungere al \emph{companion object} di \lstinline{State} un metodo che permetta di ottenere il parametro passato implicitamente:
\scalaFromFile{11}{13}{monads/mtl/State.scala}
In questo modo il corpo della funzione potrà essere implementato come:
\scalaFromFile{38}{42}{monads/mtl/State.scala}

Dunque, \lstinline{M[_]} può essere interpretato come il generico ``contesto'' all'interno del quale si svolge la computazione descritta; i \emph{context bounds} permettono di vincolare il tipo \lstinline{M} indicando i side effect che dev'essere possibile svolgere all'interno di tale contesto.

\subsubsection{Fallimento di una computazione}
In maniera analoga a quanto mostrato per lo stato mutabile, è possibile definire in astratto il side effect del fallimento di una computazione:
\scalaFromFile{8}{9}{monads/mtl/Fail.scala}
Anche in questo l'interfaccia definisce per un generico contesto \lstinline{M[_]} le operazioni rilevanti per permettere di descrivere il fallimento di una computazione.

Un'operazione che richiede tale side effect può ricevere implicitamente in input una generica monade \lstinline{M[_]} che implementi tale interfaccia:
\scalaFromFile{29}{33}{monads/mtl/Fail.scala}
Anche in questo caso è possibile sfruttare lo zucchero sintattico di Scala per definire in maniera concisa il \emph{context bound}:
\scalaFromFile{11}{12}{monads/mtl/Fail.scala}
\scalaFromFile{35}{40}{monads/mtl/Fail.scala}

\subsection{Composizione di più effetti}
\label{sub:composizione-di-piu-effetti}
Il grande vantaggio offerto da MTL sta nella possibilità di combinare fra loro side effect differenti con grande semplicità. Si consideri la seguente computazione:
\scalaFromFile{68}{73}{monads/mtl/MTL.scala}
Si possono osservare alcuni aspetti interessanti:
\begin{itemize}
  \item La computazione sfrutta due diversi effetti che sono resi espliciti tramite l'uso dei \emph{context bound}: la definizione del tipo generico \lstinline{M[_]} può essere letta come \emph{``Dato un qualunque \lstinline{M} che permetta di modificare uno stato di tipo \lstinline{S}, che permetta di far fallire la computazione e che sia una monade''}. Mentre i primi due requisiti sono fondamentali per poter utilizzare i side effect, richiedere che \lstinline{M} sia una monade ha lo scopo di poter accedere all'operazione di \lstinline{flatMap} per mettere in sequenza più operazioni
  \item Non viene concretizzato il tipo che permetterà di interpretare i diversi effetti, tutta la definizione è basata su un generico \lstinline{M[_]}; la semantica delle operazioni potrà essere stabilita in un secondo momento in maniera indipendente. Perciò, a differenza dei transformer, la descrizione della computazione e di come questa verrà eseguita sono disaccoppiate
\end{itemize}

Inoltre viene eliminata la rigidità data dai monad transformer nel comporre operazioni con tipi differenti: in un contesto \lstinline{M[_]} è possibile eseguire operazioni che richiedono entrambi gli effetti, solo uno di essi o nessuno di essi. Nell'esempio precedente viene utilizzata la funzione \lstinline{divide} che richiede solo la capacità di fallire; ciò è possibile dato che è uno degli effetti forniti nel contesto della funzione \lstinline{effect}. Non è necessario applicare operazioni di \emph{lifting} nè modificare artificialmente il tipo di ritorno delle funzioni per semplificarne la composizione come veniva fatto nel caso dei transformer.

\subsection{Interpretazione della computazione}
Una computazione descritta secondo lo stile MTL definisce in astratto quali sono le operazioni che dovranno essere eseguite; quindi, per poterla interpretare è necessario fornire un tipo concreto che implementi tutte le operazioni richieste.
L'implementazione viene sempre fornita come istanza di una \emph{type class} in maniera analoga a quanto fatto per le monadi.
Per esempio il transformer \lstinline{OptionT} può essere usato come meccanismo concreto per interpretare computazioni che possono fallire:
\scalaFromFile{14}{15}{monads/mtl/Fail.scala}
Può essere utile riprendere l'interpretazione delle \emph{type class} come predicati: fornire un'istanza equivale a dimostrare che \lstinline{OptionT} può fallire e la dimostrazione è data dall'implementazione del metodo \lstinline{fail}. Tuttavia, come già descritto in precedenza, \lstinline{OptionT} non è l'unica monade che permette di modellare una computazione che fallisca; il fallimento può essere modellato anche nella monade \lstinline{IO} tramite l'uso di eccezioni:
\scalaFromFile{17}{18}{monads/mtl/Fail.scala}

Una volta definite tali istanze sarà possibile sfruttarle per interpretare una qualunque computazione che richieda il side effect del fallimento:
\scalaFromFile{42}{45}{monads/mtl/Fail.scala}
Utilizzare un diverso interprete per la computazione si riduce a indicare quale monade utilizzare come concreta implementazione del contesto \lstinline{M[_]}. Nel primo caso viene utilizzata la monade \lstinline{IO} e, se la computazione dovesse fallire, verrà sollevata un'eccezione; nel secondo caso viene utilizzato \lstinline{OptionT} e in caso di fallimento il risultato sarà \lstinline{None}.

L'istanza per l'effetto della modifica di uno stato è simile a quella del fallimento; in questo caso si mostra come il transformer \lstinline{StateT} permetta di ottenere tale effetto:
\scalaFromFile{15}{18}{monads/mtl/State.scala}

\subsubsection{Integrazione di stack di monadi e MTL}
Le istanze fornite fino a questo momento permettono di ottenere semplici effetti ma non sono sufficienti per interpretare computazioni che necessitino di più effetti contemporaneamente: per esempio il semplice \lstinline{OptionT} non può essere utilizzato per interpretare computazioni che richiedono uno stato mutabile.
Una computazione simile richiede necessariamente uno stack di monadi che abbia al proprio interno una monade in grado di gestire il cambiamento di stato (come per esempio \lstinline{StateT}), in quel caso l'intero stack potrà essere sfruttato per interpretare la computazione:
\scalaFromFile{21}{25}{monads/mtl/State.scala}
In questo caso la definizione dell'istanza è più complessa:
\begin{itemize}
  \item L'istanza viene definita per un generico transformer \lstinline{T} e una generica monade \lstinline{M}; se \lstinline{M} può gestire uno stato mutabile di tipo \lstinline{S}, allora anche \lstinline{T} con \lstinline{M} al proprio interno potrà farlo
  \item Per poter ottenere lo stato globale, \lstinline{T} delega alla monade \lstinline{M} l'operazione di \lstinline{get} che viene poi inserita all'interno del transformer tramite \lstinline{lift}. L'operazione di \lstinline{set} è implementata in maniera analoga
\end{itemize}

Le due istanze \lstinline{stateHasState} e \lstinline{transformerHasState} appena mostrate permettono di ottenere un'istanza di \lstinline{State} per un qualunque stack di monadi che abbia \lstinline{StateT} al proprio interno: \lstinline{stateHasState} funge da caso base per la ricorsione mentre \lstinline{transformerHasState} permette di applicare tutti i \lstinline{lift} necessari automaticamente indipendentemente dalla composizione dello stack di monadi.

Si consideri il seguente stack: \lstinline{Transformer1[Transformer2[StateT[Int, IO, _], _], _]} (si supponga che \lstinline{Transformer1} e \lstinline{Transformer2} siano due monad transformer). Il compilatore Scala potrà derivare in automatico un'istanza di \lstinline{HasState[Int]} per l'intero stack secondo il seguente procedimento:
\begin{enumerate}
  \item \lstinline{StateT[Int, IO, _]} ha un'istanza di \lstinline{State} che può essere ottenuta tramite \lstinline{stateHasState}
  \item Poiché \lstinline{Transformer2} è un transformer e \lstinline{StateT[Int, IO, _]} è una monade con un'istanza di \lstinline{State} allora anche \lstinline{Transformer2[StateT[Int, IO, _], _]} ha un'istanza di \lstinline{State} che può essere sintetizzata grazie a \lstinline{transformerHasState}
  \item Poiché \lstinline{Transformer1} è un transformer e \lstinline{Transformer2[StateT[Int, IO, _], _]} è una monade con un'istanza di \lstinline{State} -- generata al passaggio precedente -- allora anche \lstinline{Transformer1[Transformer2[StateT[Int, IO, _], _], _]} ha un'istanza di \lstinline{State} che può essere sintetizzata grazie a \lstinline{transformerHasState}
\end{enumerate}

Nel concreto una chiamata a \lstinline{get} in questo stack sarà interpretata come \lstinline{StateT.get.lift[Transformer2].lift[Transformer1]}; tutti i \lstinline{lift} saranno applicati automaticamente grazie alle istanze generate dal compilatore.

Il caso di fallimento è analogo a quanto osservato per \lstinline{State}; se un transformer ha al proprio interno una monade che permette di ottenere tale side effect allora può a sua volta fornire un'istanza di \lstinline{Fail}:
\scalaFromFile{21}{24}{monads/mtl/Fail.scala}

Una volta definite tali istanze una computazione complessa come \lstinline{effects} mostrata nella \Cref{sub:composizione-di-piu-effetti} può essere interpretata specificando uno stack contenente i transformer necessari:
\scalaFromFile{77}{88}{monads/mtl/MTL.scala}
Il compilatore fornirà implicitamente le istanze necessarie a interpretare la computazione. A seconda dello stack utilizzato si potrà dare una diversa interpretazione degli effetti specificati:
\begin{itemize}
  \item Nel primo risultato ogni modifica allo stato prima del fallimento della computazione sarebbe comunque valida: infatti, il tipo di ritorno non incapsula anche lo stato \lstinline{Int} all'interno di \lstinline{Option}
  \item Nel secondo caso lo stato è gestito in maniera ``transazionale'': se la computazione fallisce allora lo stato non viene modificato
  \item Nel terzo caso l'interruzione della computazione viene ottenuta tramite le eccezioni e interpretarla potrebbe comportare il fallimento del programma a \emph{runtime}
\end{itemize}

\section{MTL come effect system}

\ac{MTL} può essere visto come un primo esempio di \term{effect system}: un sistema formale che permette di descrivere in maniera concisa i side effect di una computazione, le sue azioni osservabili dall'esterno~\cite[p.~943]{cit:design-concepts-in-programming-languages}.

Per esempio, il meccanismo delle eccezioni di Java rappresenta un rudimentale \term{effect system} che permette di tracciare una sola tipologia di side effect: il lancio di un'eccezione~\cite[p.~985]{cit:design-concepts-in-programming-languages}.
Anche in Scala, tramite l'uso delle \term{capability}~\cite{cit:scala-3-reference-canthrow}, è stato definito un sistema di effetti che permette di verificare staticamente la presenza di eccezioni~\cite{cit:safer-exceptions-for-scala}. Attualmente è possibile abilitare questo meccanismo di cattura delle eccezioni come feature sperimentale; tuttavia, diverse ricerche sono attualmente in atto per permettere l'utilizzo delle \term{capability} per gestire qualunque tipologia di effetto~\cite{cit:effects-capabilities-and-boxes-from-scope-based-reasoning-to-type-based-reasoning-and-back,cit:caprese}.

La versione originale della libreria \ac{MTL} include diverse classi per trattare un'ampia gamma di side effect comuni: il fallimento di una computazione, la lettura di uno stato immutabile, la modifica di uno stato mutabile, ecc.

Tuttavia, l'approccio \ac{MTL} è molto generale e permette di definire in \term{user space} quali sono gli effetti rilevanti per un determinato dominio applicativo, lasciando ampia libertà al programmatore di scegliere il livello di astrazione più adatto.
Un \term{effect system} può diventare quindi uno strumento di design per la realizzazione di sistemi orientati agli effetti.

\subsection{Definizione di effetti arbitrari}
\label{sec:mtl-effetti-arbitrari}
È possibile definire \term{type class} per classi di effetti arbitrari e al livello di granularità più adatto al problema da risolvere. Si immagini di dover realizzare una porzione di codice che necessita di interfacciarsi con un database contenente degli utenti; ai fini dell'esempio un utente viene modellato in maniera semplificata come:
\scalaFromFile{11}{12}{monads/mtl/Design.scala}

Le operazioni che si vogliono implementare sono il recupero di un utente in base al suo identificativo, il salvataggio di un utente e la sua cancellazione.

Sicuramente tali operazioni potrebbero essere descritte direttamente all'interno della monade \lstinline{IO}, stabilendo una connessione con il database ed eseguendo le query SQL corrispondenti. Tuttavia, tale approccio rende più difficile la realizzazione di test unitari: nell'esecuzione delle query queste cercheranno di stabilire una connessione con il database e potrebbero alterarne lo stato se non opportunamente gestite. Una soluzione comunemente adottata consiste nell'attuare \term{dependency injection} simulando il comportamento del database senza effettivamente stabilire una connessione, oppure connettendo i test a un database di test separato da quello in produzione.

\subsubsection{Definizione degli effetti}
Secondo lo stile \ac{MTL} vengono innanzitutto definite le operazioni rilevanti per il dominio: primitive astratte che definiscono le operazioni che si possono eseguire sul database. La codifica mostrata in precedenza per gli effetti del fallimento e dello stato mutabile può essere utilizzata anche in questo caso:
\scalaFromFile{14}{17}{monads/mtl/Design.scala}
L'interfaccia definisce un insieme minimale e ortogonale di operazioni di base che si vuole poter compiere nell'interazione con il database contenente gli utenti. Queste operazioni possono poi essere composte per ottenere logiche più complesse:
\scalaFromFile{29}{40}{monads/mtl/Design.scala}
Lo scopo di \lstinline{updateOrDelete} è di permettere di aggiornare un utente a partire dal suo identificativo, oppure di cancellarlo secondo la logica stabilita da una funzione passata come parametro.
Tale funzione è definita in astratto per un qualunque tipo \lstinline{M[_]} che permetta di ottenere le operazioni di \lstinline{UserStore}.

\subsubsection{Interpretazione in un ambiente di produzione}
Il vantaggio dato dall'\term{encoding} \ac{MTL} delle operazioni sta nella possibilità di modificare come queste vengono interpretate in base alla necessità.
Per esempio le operazioni possono essere interpretate in un ambiente di produzione dove viene effettivamente stabilita una connessione con una database. In questo caso, lo \term{stack} utilizzato per interpretare la computazione dovrà sicuramente poter permettere l'esecuzione di IO e gestire uno stato che contenga la connessione al database:
\scalaFromFile{54}{55}{monads/mtl/Design.scala}
In questo caso l'interprete adottato per un ambiente di produzione può manipolare uno stato -- nell'esempio chiamato \lstinline{Runtime} -- che contiene la connessione al database.

Una volta stabilito lo \term{stack} da utilizzare è sufficiente definire un'istanza di \lstinline{UserStore}:
\scalaFromFile{57}{63}{monads/mtl/Design.scala}
L'implementazione recupera la connessione a partire dal \term{runtime} e la sfrutta per comunicare con il database, l'effettiva implementazione in questo caso non è rilevante ed è stata tralasciata.

L'idea fondamentale sta nella possibilità di disaccoppiare la \emph{descrizione} della logica applicativa dalla sua effettiva \emph{interpretazione}. Modificare l'interprete non comporta cambiamenti nelle funzioni descritte in astratto e viceversa.

\subsubsection{Interpretazione in un ambiente di test}
Per mostrare come sia possibile modificare l'interprete in base alle necessità si consideri la seguente porzione di codice:
\scalaFromFile{43}{48}{monads/mtl/Design.scala}
La funzione, preso un identificativo di un utente, lo elimina se questo è minorenne o ne aumenta l'età di un anno. Come è possibile osservare dalla firma del metodo, questo necessita di poter effettuare gli effetti descritti in astratto da \lstinline{UserStore}.

Per poter testare la correttezza della logica applicativa -- vale a dire che utenti minorenni siano effettivamente cancellati -- potrebbe non essere pratico realizzare e connettersi ad un database di test. In questo caso è possibile definire un interprete che simuli il comportamento del database mantenendo un insieme di utenti in memoria senza necessità di effettuare operazioni di input e output. L'unico elemento di cui il runtime ha bisogno è una mappa degli utenti:
\scalaFromFile{76}{77}{monads/mtl/Design.scala}
L'interprete \lstinline{TestRunner} non è altro che la monade \lstinline{State} che manipola tale mappa. L'implementazione delle operazioni risulta essere molto semplice:
\scalaFromFile{88}{91}{monads/mtl/Design.scala}
\scalaFromFile{79}{85}{monads/mtl/Design.scala}

Diventa quindi possibile testare la logica applicativa senza dover effettuare alcuna operazione di input e output; il test stabilisce lo stato iniziale del sistema e verifica che la logica dell'operazione eseguita sia corretta:
\scalaFromFile{105}{110}{monads/mtl/Design.scala}


\chapter{Free Monads}
\label{ch:free-monads}

Fino ad ora si è osservato come sia possibile modellare i side effect tramite l'uso di monadi, sia ricorrendo ai \term{monad transformers} che all'approccio MTL. In quest'ultimo caso la computazione viene descritta tramite l'uso di metodi astratti definiti in un'interfaccia; l'interpretazione concreta di tali chiamate a metodo può essere definita in un secondo momento assegnando la semantica desiderata a ciascuna di esse.

L'approccio adottato dalle \term{free monad} consiste nel descrivere la computazione tramite un \term{abstract syntax tree}\footnote{Successivamente abbreviato in AST} che permette di comporre in sequenza più operazioni definite in maniera astratta. In questo modo è possibile dare semantica alle operazioni definendo interpreti che, attraversando l'AST, possono tradurre le operazioni astratte in una versione ``eseguibile''.

\section{Implementazione di una Free Monad}

L'AST descritto tramite una free monad ha l'obiettivo di catturare la struttura sintattica di una computazione monadica~\cite{cit:programming-monads-operationally-with-unimo}.
Come già descritto in precedenza, l'interfaccia delle monadi permette di descrivere una computazione come una sequenza di passi successivi che, una volta terminati, restituiscono un valore.

In seguito verrà formalizzata tale definizione arrivando in maniera graduale alla definizione di una free monad.
L'encoding adottato è quello delle cosiddette \emph{Free Operational Monads}~\cite{cit:programming-monads-operationally-with-unimo}; tuttavia esistono svariate tecniche per ottenere risultati analoghi~\cite{cit:data-types-a-la-carte,cit:freer-monads-more-extensible-effects,cit:fusion-for-free}.

\subsection{Descrizione astratta di un programma monadico}
\subsubsection{Istruzioni di un programma monadico}
In generale si può considerare un programma monadico come definito da una serie di istruzioni appartenenti ad un insieme $I$ messe in sequenza fra loro.
Ogni istruzione, una volta eseguita, può restituire un valore differente; si consideri per esempio il seguente programma definito nella monade \lstinline{State}:
\begin{lstlisting}[language=scala3]
def program[S](f: S => S): State[S, String] =
  for
    state <- get
    newState = f(state)
    _ <- set(newState)
  yield "end"
\end{lstlisting}
Le operazioni di base di cui si compone sono \lstinline{get} e \lstinline{set}; mentre la prima restituisce un valore di tipo \lstinline{S}, la seconda restituisce \lstinline{Unit}.
Quindi, in questo caso, l'insieme delle istruzioni sarebbe $I = \{get, set\}$; questo insieme può essere codificato in Scala con una semplice enumerazione:
\scalaFromFile{7}{9}{monads/free/State.scala}
Un valore di tipo \lstinline{StateDSL[S, A]} rappresenta una singola istruzione che opera su uno stato di tipo \lstinline{S} e restituisce un valore di tipo \lstinline{A}:
\begin{itemize}
  \item \lstinline{Get} opera su uno stato di tipo \lstinline{S} e lo restituisce, quindi ha tipo \lstinline{StateDSL[S, S]}
  \item \lstinline{Set} ha tipo \lstinline{StateDSL[S, Unit]} in quanto, come descritto in precedenza, ha l'effetto di modificare lo stato \lstinline{S} ma non restituisce alcun valore d'interesse
\end{itemize}

\subsubsection{Esecuzione di istruzioni in un programma monadico}
Come descritto all'inizio del capitolo una free monad cattura in una struttura dati la struttura sintattica di un programma monadico.
Un tipo di dato che rappresenti l'AST di un programma monadico deve quindi permettere di rappresentare l'esecuzione di una singola istruzione.
Per disaccoppiare l'AST dallo specifico tipo di istruzioni eseguite un programma verrà definito in maniera generica rispetto alle istruzioni che può eseguire:
\begin{lstlisting}[language=scala3]
enum Program[I[_], A]: ...
\end{lstlisting}
Il tipo generico \lstinline{I} rappresenta il set di istruzioni che possono essere eseguite dal programma monadico, mentre \lstinline{A} rappresenta il tipo del valore di ritorno ottenuto dall'esecuzione del programma. Per esempio un valore di tipo \lstinline{Program[StateDSL[Int, _], String]} rappresenterà un programma monadico che può eseguire operazioni su uno stato di tipo \lstinline{Int} e restituisce un valore di tipo \lstinline{String}.

Per poter costruire un programma che esegue una singola istruzione può essere definito un caso ad hoc dell'enumerazione e uno \emph{smart constructor}:
\begin{lstlisting}[language=scala3]
enum Program[I[_], A]:
  case Instruction[I[_], A](instruction: I[A])
    extends Program[I, A]

object Program:
  def fromInstruction[I[_], A](instruction: I[A]): Program[I, A] =
    Instruction(instruction)
\end{lstlisting}

Questa definizione permette di creare semplici programmi che eseguono una singola istruzione:
\begin{lstlisting}[language=scala3]
  def get[S] = Program.fromInstruction(StateDSL.Get())
  def set[S](state: S) =
    Program.fromInstruction(StateDSL.Set(state))
\end{lstlisting}

Chiaramente un programma che esegue una singola istruzione non è particolarmente interessante; per poter combinare semplici istruzioni e realizzare programmi complessi è necessario espandere la definizione di \lstinline{Program} fornita.

\subsubsection{Terminazione di un programma monadico}
Un'ulteriore caratteristica di ogni computazione monadica -- espressa dal metodo \lstinline{pure} dell'interfaccia delle monadi -- sta nella possibilità di poter inserire un qualunque valore all'interno della sequenza delle computazioni senza aggiungervi ulteriori side effect.
Tuttavia, fino a questo momento un programma monadico si limita a permettere l'esecuzione di una singola istruzione appartenente al set \lstinline{I} specificato. L'AST di \lstinline{Program} può essere esteso per permettere questa operazione:
\begin{lstlisting}[language=scala3]
enum Program[I[_], A]:
  case Instruction[I[_], A](instruction: I[A])
    extends Program[I, A]
  case Return[I[_], A](value: A) extends Program[I, A]

object Program:
  def fromValue[I[_], A](value: A): Program[I, A] =
    Return(value)
\end{lstlisting}

Questo ulteriore costruttore permette di definire programmi che restituiscono valori senza alcun side effect, ovvero senza eseguire nessuna delle operazioni dell'insieme \lstinline{I}. Per esempio
\begin{lstlisting}[language=scala3]
val program: Program[StateDSL[Int, _], String] =
  Program.fromValue("result")
\end{lstlisting}
Restituisce il valore \lstinline{"result"} senza modificare lo stato mutabile tramite le operazioni definite nello \lstinline{StateDSL}.

\subsubsection{Messa in sequenza di programmi}
L'aspetto più importante di una monade sta nella possibilità di mettere in sequenza operazioni tramite l'uso di \lstinline{flatMap}.
Anche l'AST di \lstinline{Program} deve esprimere questo concetto per poter catturare la struttura di un programma monadico. Per questo motivo viene definito un ultimo costruttore:
\scalaFromFile{9}{16}{monads/free/lib/Free.scala}
Il costruttore \lstinline{Then} permette di catturare la messa in sequenza di operazioni monadiche: il primo argomento è il primo programma da eseguire che produrrà un valore di tipo \lstinline{A}; il secondo argomento è una continuazione che, preso il valore prodotto dal primo programma, restituisce un secondo programma da eseguire.

È possibile definire un \emph{extension method} per rendere più facile la costruzione di programmi complessi:
\begin{lstlisting}[language=scala3]
extension [I[_], A](program: Program[I, A])
  def andThen[B](continuation: A => Program[I, B]) =
    Then(program, continuation)
\end{lstlisting}

Per esempio il programma con stato mostrato alla sezione precedente può essere ora espresso come:
\begin{lstlisting}[language=scala3]
def program[S](f: S => S): Program[StateDSL[S, _], String] =
  get.andThen { state => 
    val newState = f(state)
    set(newState).andThen { _ => 
      Program.fromValue("end")
    }
  }
\end{lstlisting}

È possibile osservare come \lstinline{Program} rispetti per costruzione l'interfaccia delle monadi, indipendentemente dal tipo di istruzioni che utilizza: i costruttori \lstinline{Result} e \lstinline{Then} equivalgono rispettivamente alle operazioni \lstinline{pure} e \lstinline{flatMap}:
\scalaFromFile{36}{41}{monads/free/lib/Free.scala}
In questo modo è possibile definire programmi nella monade \lstinline{Program} sfruttando lo zucchero sintattico della \emph{for comprehension}; l'esempio precedente può essere riscritto in maniera più chiara come:
\begin{lstlisting}[language=scala3]
def program[S](f: S => S): Program[StateDSL[S, _], String] =
  for
    state <- get
    newState = f(state)
    _ <- set(newState)
  yield "end"
\end{lstlisting}

\section{Interpretazione di una free monad}

\subsection{Interpretazione delle istruzioni di un programma}
I valori di tipo \lstinline{Program} mostrati fino ad ora non sono altro che degli AST che rappresentano sotto forma di struttura dati un programma monadico.
Questa struttura dati può essere attraversata e interpretata per assegnare una semantica alle istruzioni di cui si compone.
Prima di poter definire come interpretare un programma monadico è necessario definire come interpretarne le singole istruzioni; perciò si introduce il concetto di interprete\footnote{Questo concetto è noto in letteratura come trasformazione naturale\cite{cit:monad-transformers-and-modular-algebraic-effects}. Nella trattazione si preferisce utilizzare il termine ``interprete''  in quanto rende in maniera efficace il suo utilizzo nell'interpretazione delle operazioni di una free monad.}:
\scalaFromFile{4}{7}{monads/free/lib/Interpreter.scala}
Un interprete, dato un generico insieme di istruzioni \lstinline{F} permette di trasformare -- tramite il metodo \lstinline{apply} -- una qualunque istruzione di tipo \lstinline{F[A]} in un valore \lstinline{G[A]}. Il tipo \lstinline{G[_]} è generico e potrebbe per esempio essere un secondo insieme di istruzioni o una monade.

Per esempio, è possibile definire un interprete che trasforma le istruzioni dello \lstinline{StateDSL} in azioni concrete all'interno del transformer \lstinline{StateT}:
\scalaFromFile{25}{30}{monads/free/State.scala}
L'interprete si limita a tradurre le istruzioni di \lstinline{Get} e \lstinline{Set} nelle corrispondenti operazioni del transformer.

\subsection{Interpretazione di un'intero programma}
È possibile sfruttare gli interpreti che traducono singole istruzioni per interpretare interi programmi definiti all'interno della monade \lstinline{Program}.

L'unico accorgimento necessario è che l'interprete traduca ciascuna istruzione in un tipo che rispetti l'interfaccia di monade. Grazie a questo vincolo è possibile mettere automaticamente in sequenza le operazioni una volta che sono state tradotte:
\scalaFromFile{43}{53}{monads/free/lib/Free.scala}

\begin{itemize}
  \item \lstinline{Return} corrisponde ad una chiamata di \lstinline{pure} nella generica monade \lstinline{M}
  \item \lstinline{Instruction} permette di eseguire una singola istruzione che viene direttamente interpretata dall'interprete fornito in input
  \item \lstinline{Then} rappresenta la composizione in sequenza di due programmi. Il primo programma viene interpretato ottenendo un valore di tipo \lstinline{A}; questo viene fornito in input alla continuazione per ottenere il secondo programma che ricorsivamente viene interpretato producendo il risultato finale
\end{itemize}

\subsection{Ispezione dell'AST di una free monad}
Il metodo mostrato nella sezione precedente permette di dare semantica al programma stabilendo come ciascuna delle istruzioni debba essere interpretata.
Tuttavia, il vantaggio dato dalla rappresentazione della computazione come AST sta nella possibilità di poter ispezionare il programma per avere un controllo più fine sulla sua interpretazione.
Infatti, è possibile estrarre dal programma la prima istruzione che questo deve eseguire e la continuazione che determina come l'esecuzione deve procedere. Per fare ciò è possibile definire la seguente struttura:
\scalaFromFile{19}{24}{monads/free/lib/Free.scala}
\lstinline{ProgramView} permette di avere una vista uniforma di un programma limitando i casi possibili a due sole opzioni: il programma esegue un'istruzione e prosegue con una data continuazione, oppure il programma termina restituendo un valore.

Per poter estrarre una \lstinline{ProgramView} a partire da un programma è possibile definire un metodo:
\begin{lstlisting}[language=scala3]
extension [I[_], A](program: Program[I, A])
  @tailrec def next: ProgramView[I, A] = program match
  case Return(value) => ProgramView.Return(value)
  case Instruction(instruction) =>
    ProgramView.Then(instruction, Return(_))
  case Then(program, f) =>
    program match
      case Return(value) => f(value).next
      case Instruction(instruction) =>
        ProgramView.Then(instruction, f)
      case Then(program, g) =>
        program.andThen(x => g(x).andThen(f)).next
\end{lstlisting}

I diversi casi del pattern matching coprono tutte le possibili conformazioni di un programma:
\begin{itemize}
  \item Se il programma restituisce un valore questo viene mappato nella vista corrispondente
  \item Se il programma esegue un'istruzione questa viene inserita nella vista come prossima istruzione e la continuazione si limita a restituire il valore dell'istruzione
  \item Nel caso vengano composti più programmi in sequenza è necessario osservare la composizione del primo programma:
        \begin{itemize}
          \item Se si limita a restituire un valore questo viene fornito alla continuazione e viene restituita la prima istruzione del programma ottenuto
          \item Se si limita a eseguire un'istruzione questa viene tradotta nella vista corrispondente che racchiude al proprio interno l'istruzione e la continuazione
          \item Se è a sua volta la composizione sequenziale di due programmi, allora viene restituita la prima operazione del programma più interno
        \end{itemize}
\end{itemize}

Questo meccanismo rende possibile realizzare funzioni che, ispezionando di volta in volta la successiva istruzione di un programma, permette di interpretarlo. Per esempio, è possibile realizzare una funzione che esegue una computazione che fa uso dello \lstinline{StateDSL}:
\scalaFromFile{40}{47}{monads/free/State.scala}
La funzione ottiene la prima istruzione da eseguire e nel caso in cui sia una \lstinline{Get} fornisce alla continuazione lo stato corrente. Invece, nel caso in cui l'istruzione sia \lstinline{Set} la continuazione viene ripresa fornendole il valore \lstinline{Unit} e sul programma viene chiamato il metodo \lstinline{runWithState} ricorsivamente modificando lo stato fornito.

\subsubsection{A poor man’s concurrency (free) monad}
\label{sec:poor-man}
È possibile sfruttare la possibilità di ispezionare l'AST di un programma descritto in questo modo per realizzare interpreti più complessi.

Questa sezione riprende l'eccellente esempio mostrato in~\cite{cit:a-poor-mans-concurrency-monad} e illustra come sia possibile implementare una \emph{``poor man's concurrency monad''} utilizzando la free monad appena mostrata.
In particolare, la possibilità di accedere alla continuazione che determina come deve procedere la computazione rende piuttosto semplice l'implementazione di una forma di concorrenza in \emph{user space}.

Il linguaggio preso in considerazione è il seguente:
\scalaFromFile{9}{15}{monads/free/PoorManConcurrency.scala}
Le operazioni mostrate sono sufficienti per descrivere un meccanismo di multithreading cooperativo:
\begin{itemize}
  \item \lstinline{Fork} permette di creare un nuovo thread che esegue il programma specificato
  \item \lstinline{YieldControl} permette a un thread di segnalare esplicitamente allo scheduler di voler mettere in pausa la propria esecuzione lasciando il controllo ad altri thread
  \item \lstinline{Stop} permette a un thread di terminare prematuramente la propria esecuzione
  \item \lstinline{Perform} viene utilizzato come meccanismo per eseguire una qualunque azione con side effect; la funzione che produce il valore di tipo \lstinline{A} può eseguire side effect arbitrari
\end{itemize}

È possibile definire alcuni \emph{smart constructor} per permettere di costruire più semplicemente un programma concorrente:
\scalaFromFile{17}{22}{monads/free/PoorManConcurrency.scala}
Un esempio di programma ottenuto a partire da queste operazioni di base potrebbe essere il seguente:
\scalaFromFile{59}{68}{monads/free/PoorManConcurrency.scala}

Per poter stabilire l'effettiva semantica delle operazioni e permettere l'esecuzione del programma è necessario definire una funzione che permetta di interpretarlo. In questo esempio il comportamento desiderato è quello di realizzare uno scheduler che esegua i thread in maniera cooperativa: un thread esegue le proprie operazioni ininterrotto fino a quando non termina o rende esplicito il voler cedere il controllo tramite una chiamata a \lstinline{yield}; a quel punto lo scheduler stabilirà il nuovo thread da mandare in esecuzione con la stessa politica.
Il metodo per eseguire un programma secondo questa politica di scheduling è il seguente:
\scalaFromFile{25}{33}{monads/free/PoorManConcurrency.scala}
Per semplificare l'implementazione dello scheduler, questo utilizza una coda in cui si trovano i thread da eseguire e nel momento in cui avviene lo \lstinline{yield} il thread corrente viene messo in fondo alla coda e viene dato il controllo al thread in testa alla coda.
L'intera logica di esecuzione delle istruzioni è codificata nella funzione \lstinline{runInstructions} mostrata al \Cref{lst:run-instructions}.
A seconda dell'istruzione che viene eseguita, il comportamento è il seguente:
\begin{itemize}
  \item \lstinline{Perform}: viene eseguita l'azione incapsulata nel costruttore e il risultato ottenuto è dato come input alla continuazione per ottenere il programma che contiene la continuazione del thread corrente. Questo programma viene inserito in testa alla coda dei thread da eseguire in quanto la semantica scelta stabilisce che un thread possa essere interrotto solo se esegue esplicitamente \lstinline{yield}. Si noti come sarebbe possibile modificare questa politica interrompendo con \emph{preemption} un thread ad ogni operazione: sarebbe sufficiente inserire il programma che rappresenta la continuazione in fondo alla coda così da dare precedenza ad altri thread
  \item \lstinline{Stop}: interrompe l'esecuzione del thread. In questo caso non viene aggiunta alcuna continuazione alla lista di thread da eseguire. Si noti inoltre un aspetto interessante: poiché \lstinline{Stop} ha tipo \lstinline{ConcurrentDSL[Nothing]} in questo ramo del pattern matching la continuazione necessiterebbe di un valore di tipo \lstinline{Nothing} per poter generare il programma che rappresenta la continuazione del thread corrente. Poiché non esiste alcun valore che possa assumere il tipo \lstinline{Nothing}\footnote{Tecnicamente sarebbe possibile creare un valore di tipo \lstinline{Nothing} (per esempio \lstinline{???} o una qualunque eccezione) ma questo comporterebbe una terminazione anomala del programma o la sua divergenza senza poter eseguire in ogni caso la continuazione} è impossibile ottenere la continuazione e l'unica operazione sensata è quella di terminare l'esecuzione del thread
  \item \lstinline{YieldControl}: in questo caso la continuazione del thread corrente viene messa in fondo alla lista dei thread
  \item \lstinline{Fork}: è l'unica operazione che permette di accrescere il numero di programmi contenuti nella lista dei thread da eseguire. In questo caso la continuazione del thread corrente viene messa in cima alla lista in modo che possa continuare l'esecuzione mentre il thread di cui è stato effettuato il fork viene messo in fondo
\end{itemize}

\begin{figure}[htp]
  \begin{lstlisting}[language=scala3, caption={Implementazione della funzione \lstinline{runInstructions}.}, label={lst:run-instructions}]
    def runInstruction(
      instruction: ProgramView[ConcurrentDSL, Unit],
      threads: List[Concurrent[Unit]]
    ): Unit =
      instruction match
        case ProgramView.Return(_) => runThreads(threads)
        case ProgramView.Then(instruction, continuation) =>
          instruction match
            case Perform(action) =>
              val result = action()
              val newThreads = continuation(result) +: threads
              runThreads(newThreads)
            case Stop =>
              runThreads(threads)
            case YieldControl =>
              val newThreads = threads :+ continuation(())
              runThreads(newThreads)
            case Fork(process) =>
              val newThreads =
                continuation(()) +: threads :+ process
              runThreads(newThreads)
  \end{lstlisting}
\end{figure}






\section{Composizione di più DSL}

\subsection{Composizione modulare di linguaggi}
Negli esempi mostrati fino ad ora è sempre stato utilizzato un programma il cui insieme di istruzioni è definito da una sola enumerazione. Tuttavia, sarebbe desiderabile poter combinare fra loro più linguaggi di base per poter descrivere programmi complessi.

Se il programmatore fosse obbligato a definire l'intero insieme di operazioni in una sola enumerazione allora questa finirebbe per essere l'equivalente di una \emph{god interface} con moltissimi metodi che mescolano operazioni appartenenti a diversi ambiti.

Si immagini di realizzare un programma che deve leggere e scrivere dal terminale e permettere di effettuare logging delle operazioni che esegue. Sicuramente sarebbe possibile definire un solo set di istruzioni che racchiuda tutte le funzionalità richieste:
\begin{lstlisting}[language=scala3]
enum LogLevel:
  case Info, Warning, Error

enum LogAndConsole[A]:
  case Log(level: LogLevel, msg: String)
    extends LogAndConsole[Unit]
  case GetLine() extends LogAndConsole[String]
  case PrintLine(msg: String) extends LogAndConsole[Unit]
\end{lstlisting}
Tuttavia, il linguaggio \lstinline{LogAndConsole} unisce due funzionalità differenti: la gestione del logging e la gestione del terminale.
Se un programma dovesse avere bisogno unicamente di accedere al side effect del logging dovrebbe comunque essere definito in termini di \lstinline{LogAndConsole}.
Idealmente, dovrebbe essere possibile definire i due DSL separatamente per poterli combinare in un secondo momento, se necessario.

\subsubsection{Iniezione di istruzioni in un linaguaggio generico}
Si considerino due degli \emph{smart constructor} mostrati come esempi nelle sezioni precedenti:
\begin{lstlisting}[language=scala3]
def get[S]: Program[StateDSL, S] = ...
def stop: Program[ConcurrentDSL, Nothing] = ...
\end{lstlisting}
I tipi di questi programmi sono troppo specifici per poter essere utilizzati nel comporre programmi complessi che fanno uso di più DSL contemporaneamente. Infatti, ciascuno limita l'insieme di istruzioni a cui il programma può accedere a quelle di uno specifico DSL: \lstinline{StateDSL} nel primo e \lstinline{ConcurrentDSL} nel secondo.

Il problema sta nella definizione della funzione \lstinline{Program.fromInstruction} utilizzata per creare gli smart constructor. Questa funzione, infatti, restituisce un programma le cui istruzioni devono essere tutte appartenenti al tipo dell'istruzione specificata. È possibile definire uno \emph{smart constructor} più generico che permette di ``iniettare'' un'istruzione all'interno di un linguaggio più ampio; per fare ciò è possibile sfruttare nuovamente il concetto di interprete introdotto in precedenza:
\scalaFromFile{32}{33}{monads/free/lib/Free.scala}

La funzione utilizza implicitamente un interprete \lstinline{T} che permette di trasformare un'istruzione dell'insieme \lstinline{I} in un'istruzione dell'insieme \lstinline{I2}. L'istruzione passata in input viene quindi trasformata in un'istruzione del secondo insieme sfruttando tale interprete.
In questo modo è possibile realizzare degli \emph{smart constructor} che non limitano il set di istruzioni a cui il programma può accedere:
\begin{lstlisting}[language=scala3]
def stop[I[_]](using ConcurrentDSL ~> I): Program[I, Nothing] =
  Program.inject(ConcurrentDSL.Stop())
\end{lstlisting}
Il costruttore stabilisce che, dato un qualunque insieme di istruzioni \lstinline{I} per il quale sia possibile inserirvi istruzioni di tipo \lstinline{ConcurrentDSL}, è possibile creare un programma che utilizza le istruzioni di \lstinline{I}.
La notazione può essere ulteriormente alleggerita definendo un \emph{type alias} per gli interpreti:
\scalaFromFile{8}{8}{monads/free/lib/Interpreter.scala}
In questo modo, \lstinline{With} può essere utilizzato per descrivere in maniera dichiarativa le istruzioni che un generico linguaggio deve poter offrire:
\begin{lstlisting}[language=scala3]
def stop[I[_]: With[ConcurrentDSL]]: Program[I, Nothing] =
  Program.inject(ConcurrentDSL.Stop())
\end{lstlisting}
La firma del metodo può essere letta come \emph{``Dato un qualunque linguaggio \lstinline{I} che ha a disposizione le operazioni del \lstinline{ConcurrentDSL}, restituisco un programma che può utilizzare le istruzioni di \lstinline{I} e restituisce \lstinline{Nothing} quando termina''}.

\subsubsection{Esempio di un programma che combina più DSL}
Sfruttando le definizioni appena mostrate è possibile definire un programma che combina le funzionalità di più DSL. Si consideri nuovamente l'esempio di un programma che deve poter interagire con il terminale ed effettuare logging.

Gli effetti richiesti dal programma possono quindi essere descritti da due DSL distinti:
\scalaFromFile{12}{13}{monads/free/CompositionExample.scala}
\scalaFromFile{24}{26}{monads/free/CompositionExample.scala}
La differenza fondamentale rispetto agli altri esempi mostrati fino ad ora sta nella definizione degli \emph{smart constructor}: questi utilizzeranno la funzione \lstinline{inject} mostrata in precedenza per non fissare a priori il set di istruzioni da utilizzare. Per esempio nel caso del \lstinline{LogDSL} il risultato sarà:
\scalaFromFile{15}{17}{monads/free/CompositionExample.scala}

In questo modo è possibile utilizzare entrambi gli insiemi di istruzioni in un unico programma ottenuto tramite la \emph{for comprehension} e i vincoli imposti dai singoli costruttori si accumuleranno definendo una lista di DSL che il programma può utilizzare:
\scalaFromFile{40}{51}{monads/free/CompositionExample.scala}
Il programma \lstinline{echo} può utilizzare un qualunque insieme di istruzioni \lstinline{I} purché permetta di utilizzare le istruzioni di base del \lstinline{LogDSL} e del \lstinline{ConsoleDSL}. Per poter interpretare ed eseguire il programma è necessario un ultimo passo: concretizzare il linguaggio \lstinline{I} in un tipo specifico che rispetti i vincoli imposti.

\subsubsection{Concretizzazione di un linguaggio generico}
Per permettere di combinare fra loro più set di istruzioni può essere conveniente definire un coprodotto\footnote{Un coprodotto è spesso indicato anche col nome di \emph{sum type} o unione disgiunta. In Scala 3 può essere ottenuto utilizzando \lstinline{sealed trait} e \lstinline{case class} o delle enumerazioni.} che, combinando due linguaggi, permette di avere istruzioni provenienti dall'uno o dall'altro~\cite{cit:data-types-a-la-carte}.
Sfruttando il supporto diretto agli \emph{union type}~\cite{cit:scala3-union-types} di Scala 3 è possibile utilizzare la seguente definizione:
\scalaFromFile{10}{10}{monads/free/lib/Interpreter.scala}
Grazie a questo \emph{type alias} è piuttosto semplice definire linguaggi composti da un numero arbitrario di DSL:
\begin{lstlisting}[language=scala3]
enum LogDSL[A]: ... 
enum ConsoleDSL[A]: ...
enum FileSystemDSL[A]: ...

type ComposedDSL[A] =
  (LogDSL :| ConsoleDSL :| FileSystemDSL)[A]
val program: Program[ComposedDSL, Int] = ...
\end{lstlisting}
Un programma definito tramite il linguaggio \lstinline{ComposedDSL} può utilizzare le istruzioni provenienti da ciascuno dei componenti di base dell'unione. Infatti, espandendo i diversi \emph{type alias} il tipo \lstinline{ComposedDSL} non sarà altro che una serie di opzioni rappresentate da uno \emph{union type}:
\begin{lstlisting}
ComposedDSL[A] =
= (LogDSL :| ConsoleDSL :| FileSystemDSL)[A]
= ((LogDSL :| ConsoleDSL) :| FileSystemDSL)[A]
= (LogDSL :| ConsoleDSL)[A] | FileSystemDSL[A]
= (LogDSL[A] | ConsoleDSL[A]) | FileSystemDSL[A]
= LogDSL[A] | ConsoleDSL[A] | FileSystemDSL[A]
\end{lstlisting}

Il programma \lstinline{echo} mostrato in precedenza potrebbe usare come linguaggio concreto \lstinline{ConsoleDSL :| LogDSL}. Non rimane che trovare un modo per generare in automatico le istanze di \lstinline{Interpreter} necessarie per inserire le operazioni dei singoli DSL all'interno del linguaggio composto.

Una prima osservazione fondamentale sta nel fatto che un linguaggio \lstinline{F} può sempre essere interpretato in un linguaggio \lstinline{F :| G}:
\scalaFromFile{13}{14}{monads/free/lib/Interpreter.scala}
Una qualunque istruzione di tipo \lstinline{F[A]} è automaticamente un sottotipo di \lstinline{(F :| G)[A]}; infatti, \lstinline{(F :| G)[A]} equivale all'unione \lstinline{F[A] | G[A]}.

La sola istanza \lstinline{left} non è sufficiente per generare in automatico gli interpreti necessari a combinare più linguaggi. Infatti permette solo di inserire \lstinline{F} in un'unione che contiene \lstinline{F} come primo elemento. Una generalizzazione di quest'istanza permette di inserire \lstinline{F} in un'unione il cui primo elemento possa a sua volta contenere \lstinline{F}:
\scalaFromFile{16}{19}{monads/free/lib/Interpreter.scala}
Dati tre linguaggi \lstinline{F, G} e \lstinline{H} se è possibile interpretare \lstinline{F} nel linguaggio \lstinline{G} allora è anche possibile definire un interprete per \lstinline{F} nel linguaggio \lstinline{G :| H}.

Questa coppia di definizioni è sufficiente perché il compilatore possa automaticamente generare qualunque istanza di interprete per linguaggi ottenuti componendo un numero arbitrario di DSL tramite il combinatore \lstinline{(:|)}.

Al \Cref{lst:echo} è mostrato come sia possibile definire un interprete che permette di eseguire un programma con un set di istruzioni composto da \lstinline{ConsoleDSL} e \lstinline{LogDSL}.

\begin{figure}[htp]
  \begin{lstlisting}[language=scala3, caption={Esempio di interpretazione di un programma composto da più DSL. Il \emph{pattern matching} sulle istruzioni permette di gestire istruzioni provenienti da entrambi i DSL di base.}, label={lst:echo}]
@tailrec def interpret[A](
  program: Program[LogDSL :| ConsoleDSL, A]
): A =
  program.next match
    case ProgramView.Return(a) => a
    case ProgramView.Then(instruction, continuation) =>
      instruction match
        case ConsoleDSL.PrintLine(msg) =>
          println(msg)
          interpret(continuation(()))
        case ConsoleDSL.GetLine() =>
          val line = scala.io.StdIn.readLine()
          interpret(continuation(line))
        case LogDSL.Log(logLevel, msg) =>
          println(f"[$logLevel] $msg")
          interpret(continuation(()))
  \end{lstlisting}
\end{figure}


% \subsubsection{Composizione di interpreti}

\nicetohave{Non sono nella scaletta ma, se c'è tempo, si potrebbe parlare anche delle Hierarchical Free Monads come approccio alternativo per combinare linguaggi diametralmente opposto alla composizione mostrata.}

%\subsection{Hierarchical Free Monads}
%Definizione di \emph{subsystem} che stabiliscono con precisione e in maniera puntuale quali sono le operazioni che possono essere eseguite. Non c'è libertà, non è estensibile (un possibile escape hatch al massimo è permettere l'esecuzione di codice arbitrario)

%Quindi forza a scegliere un buon design per la realizzazione di un framework fornendo unicamente allo sviluppatore un DSL che non può che essere usato come previsto. Separa la parte di interfaccia che poi viene usata per realizzare "script" monadici da quella di implementazione degli interpreti che permettono di eseguire le implementazioni. Netta separazione fra i due aspetti.

%Inoltre riduce fortemente il carico cognitivo per lo sviluppatore che deve utilizzare il linguaggio esposto dal DSL. Non ci sono type class, non ci sono context bound o complicazioni simili; l'unico modo è osservare l'interfaccia del DSL, l'entry point e come questo interagisce con altri subsystem.

\section{Free monad come effect system}
Il meccanismo delle free monad introdotto in questo capitolo è sicuramente un esempio di effect system in quanto permette di descrivere e interpretare effetti arbitrati. Inoltre, è stato mostrato come sia possibile modellare separatamente e comporre fra loro DSL che descrivono effetti pertinenti a domini differenti.

I programmi realizzati tramite l'uso di questo meccanismo non sono altro che strutture dati che descrivono in maniera astratta quali effetti devono avere luogo. L'interpretazione degli effetti avviene in un secondo momento e può assegnare un significato alle diverse istruzioni in base alle necessità del contesto in cui si sta lavorando.

\subsection{Definizione di effetti arbitrari}
\label{sec:free-effetti-arbitrari}
Anche in questo caso, così come per l'approccio MTL, è possibile definire qualunque tipo di effetto sia ritenuto rilevante ai fini della computazione; la scelta del livello di dettaglio con cui questi vengono descritti è lasciato a discrezione dell'autore del DSL.

In seguito viene ripreso l'esempio mostrato per MTL nella \Cref{sec:mtl-effetti-arbitrari} reimplementandolo in maniera analoga nello stile delle free monad. Gli effetti da modellare sono il recupero di un utente, il salvataggio e la cancellazione:
\scalaFromFile{9}{12}{monads/free/UserStore.scala}
Il DSL è analogo a quanto definito per MTL; una differenza importante sta nella necessità, nel caso delle free monad, di dover definire dei costruttori per ogni operazione del DSL:
\scalaFromFile{14}{22}{monads/free/UserStore.scala}

Come già mostrato in precedenza, è possibile codificare programmi complessi in termini delle operazioni di base. La logica applicativa sarà basata sull'interfaccia astratta definita dall'insieme di operazioni che compongono il DSL:
\scalaFromFile{33}{37}{monads/free/UserStore.scala}
L'implementazione è essenzialmente identica a quella mostrata per l'esempio di MTL; l'unica sostanziale differenza sta nel tipo di ritorno delle funzioni. Da un lato MTL richiede di definire una generica monade \lstinline{M} all'interno della quale sarà incapsulato il valore di ritorno; l'approccio delle free monad, invece, restituisce un valore di tipo \lstinline{Program} parametrizzato su un generico DSL \lstinline{I}.
L'implementazione di \lstinline{Program} fa si che questo sia automaticamente una monade; la firma del metodo non dovrà quindi occuparsi di specificare alcun vincolo di questo tipo.

Nell'approccio MTL è necessario indicare esplicitamente il rispetto dell'interfaccia di monade per poter mettere in sequenza le diverse operazioni tramite l'uso di \lstinline{flatMap}. In questo caso, invece, una volta che un'operazione è stata incapsulata all'interno di \lstinline{Program} il programmatore guadagna automaticamente la possibilità di combinarla in sequenza con altri programmi.

\subsubsection{Interpretazione e testing dei programmi}
Potendo definire interpreti arbitrari per le istruzioni di un programma è piuttosto facile realizzare dei \emph{mock} che permettono di testare effetti complessi senza dover complicare significativamente il processo di testing della logica applicativa.

A differenza dell'approccio MTL, l'interpretazione può essere definita come un'analisi della sequenza di istruzioni del programma.
In questo caso, quindi, non è necessario introdurre esplicitamente il concetto di monade che rimane un dettaglio implementativo sfruttato da \lstinline{Program}.

Nel caso di MTL, poiché il vincolo di monade è reso esplicito all'interno della firma del metodo, per implementare un interprete è necessario avere un tipo di dato che rispetti l'istanza di monade introducendo la complessità di dover lavorare esplicitamente con i monad transformer.
In questo caso, invece, l'interpretazione può essere implementata come una semplice funzione ricorsiva che di passo in passo assegna semantica all'istruzione corrente e interpreta la continuazione.

Si consideri per esempio la funzione \lstinline{updateAge} già mostrata nel caso di MTL:
\scalaFromFile{47}{52}{monads/free/UserStore.scala}
Al \Cref{lst:mock-free} è possibile osservare l'implementazione di un interprete che implementa il \emph{mock} di un database in memoria e testa la corretta implementazione della logica applicativa.

\begin{figure}
  \begin{lstlisting}[language=scala3, label={lst:mock-free}, caption={Esempio di un interprete di test per un programma che usa lo il DSL per l'accesso a un database di utenti. In questo caso l'uso di una mappa degli utenti permette di testare semplicemente la logica applicativa delle operazioni senza dover ricorrere a un vero e proprio database}]
extension [A](program: Program[UserStoreDSL, A])
  @tailrec
  def runMocked(users: Map[UserId, User]): Map[UserId, User] =
    program.next match
      case ProgramView.Return(value) => users
      case ProgramView.Then(instruction, continuation) =>
        instruction match
          case Get(userId) =>
            continuation(users.get(userId)).runMocked(users)
          case Save(user) =>
            val updatedUsers = users + ((user.id, user))
            continuation(()).runMocked(updatedUsers)
          case Delete(userId) =>
            continuation(()).runMocked(users - userId)

def testUpdateAgeDeletesUnderageUsers: Unit =
  val user = User(UserId(1), "Giacomo", 12)
  val users = Map(user.id -> user)
  val finalUsers = updateAge(user.id).runMocked(users)
  assert(finalUsers.isEmpty)
  \end{lstlisting}
\end{figure}



% https://degoes.net/articles/modern-fp
% https://github.com/jdegoes/scalaworld-2015
% https://www.tweag.io/blog/2018-02-05-free-monads/

% FREE 

% FREE (are in their infancy, just getting started)
% Beauty in the beast: a functional semantics for the akward squad
% Data types a la carte 
% Freer monads (faster!)

% Denotational semantics: the meaning of a term is given by defining in a compositional way a mapping to a lower language. Layers of an onion
% NAtural transfomration takes an operation described in f and turns it into a sequential program described in g

% Defining algebras of operation

% Analisi: un programma è dati. Free applicatives (Totally parallel operations) possono essere esplorati senza produrre valori a runtime quindi è possibile fare ottimizzazioni e analisi statiche
% Es. ho un file system language molto semplice con poche operazioni e posso darci una semantica denotazionale mostrando come vengono tradotte in chiamate al SO

% MOCKING
% Kill integration and system testing



\appendix
\mustfix{Aggiungere un'appendice in cui si parla da un'introduzione della sintassi haskell (magari come confronto con scala): il minimo è mostrare applicazione di funzione, associatività e precedenza degli operatori, costruttori di tipi e currying dei tipi delle funzioni per poter capire gli esempi}
\chapter{Dimostrazioni delle leggi monadiche}
\label{dimostrazioni-delle-leggi-monadiche}

\section{Dimostrazione per la monade identità}
\label{dimostrazione-per-la-monade-identita}

Alla \Cref{la-monade-identita} è stata data una definizione di monade per \lstinline{Identity}. In seguito è riportata una dimostrazione del rispetto delle leggi monadiche elencate nella \Cref{cos-e-una-monade} per la definizione fornita\footnote{Nei passaggi della dimostrazione, così come in quelle successive, sono utilizzati i nomi canonici utilizzati per l'implementazione Scala; quindi, si utilizza \lstinline{flatMap} anziché \lstinline{>>=} e \lstinline{pure} anziché \lstinline{return}}.

Dimostrazione dell'identità sinistra, ovvero che \lstinline{pure(a).flatMap(f) = f(a)}:

\begin{tabularx}{\textwidth}{ll}
  \lstinline{pure(a).flatMap(f) =} & \emph{Definizione di \lstinline{pure}} \\
  \\
  \lstinline{a.flatMap(f) =} & \emph{Definizione di \lstinline{flatMap}}    \\
  \\
  \lstinline{f(a)}$\qed$ &
\end{tabularx}

Dimostrazione dell'identità destra, ovvero che \lstinline{m.flatMap(pure) = m}:

\begin{tabularx}{\textwidth}{ll}
  \lstinline{m.flatMap(pure) =} & \emph{Definizione di \lstinline{flatMap}} \\
  \\
  \lstinline{pure(m) = m}$\qed$ & \emph{Definizione di \lstinline{pure}}    \\
\end{tabularx}

Dimostrazione dell'associatività, ovvero che \lstinline{(m.flatMap(f)).flatMap(g) = m.flatMap(x => f(x).flatMap(g))}:

\begin{tabularx}{\textwidth}{ll}
  \lstinline{(m.flatMap(f)).flatMap(g) =} & \emph{Definizione di \lstinline{flatMap}} \\
  \\
  \lstinline{f(m).flatMap(g) =} & \emph{Definizione di \lstinline{flatMap}}           \\
  \\
  \lstinline{g(f(m)) =} & \emph{Composizione di funzione}                             \\
  \\
  \lstinline{(x => g(f(x)))(m)} & \emph{Definizione di \lstinline{flatMap}}           \\
  \\
  \lstinline{m.flatMap(x => g(f(x))) =} & \emph{Definizione di \lstinline{flatMap}}   \\
  \\
  \lstinline{m.flatMap(x => f(x).flatMap(g))}$\qed$ &
\end{tabularx}


\section{Dimostrazione per la monade Optional}
\label{dimostrazione-per-la-monade-optional}

Alla \Cref{la-monade-optional} è stata data una definizione di monade per \lstinline{Option}. In seguito è riportata una dimostrazione del rispetto delle leggi monadiche elencate nella \Cref{cos-e-una-monade} per la definizione fornita\footnote{Nei passaggi della dimostrazione, così come in quelle successive, sono utilizzati i nomi canonici utilizzati per l'implementazione Scala; quindi, si utilizza \lstinline{flatMap} anziché \lstinline{>>=} e \lstinline{pure} anziché \lstinline{return}}.

Dimostrazione dell'identità sinistra, ovvero che \lstinline{pure(a).flatMap(f) = f(a)}:

\begin{tabularx}{\textwidth}{ll}
\lstinline{pure(a).flatMap(f) =} & \emph{Definizione di \lstinline{pure}}\\
\\
\lstinline{Some(a).flatMap(f) =} & \emph{Definizione di \lstinline{flatMap}}\\
\\
\lstinline{f(a)}$\qed$ &
\end{tabularx}

Dimostrazione dell'identità destra, ovvero che \lstinline{m.flatMap(pure) = m}:

\begin{tabularx}{\textwidth}{ll}
  Procedo per casi su \lstinline{m}: & \\
  & \\
  \emph{Se \lstinline{m = None}} & \\
  \\
  \lstinline{\ \ m.flatMap(pure) =} & \emph{Per ipotesi \lstinline{m = None}}\\
  \\
  \lstinline{\ \ None.flatMap(pure) =} & \emph{Definizione di \lstinline{flatMap}}\\
  \\
  \lstinline{\ \ None = m} & \emph{Per ipotesi \lstinline{None = m}} \\
  \\
  \emph{Se \lstinline{m = Some(a)}} & \\
  \\
  \lstinline{\ \ m.flatMap(pure) =} & \emph{Per ipotesi \lstinline{m = Some(a)}}\\
  \\
  \lstinline{\ \ Some(a).flatMap(pure) =} & \emph{Definizione di \lstinline{flatMap}}\\
  \\
  \lstinline{\ \ pure(a) =} & \emph{Definizione di \lstinline{pure}}\\
  \\
  \lstinline{\ \ Some(a) = m}$\qed$ & \emph{Per ipotesi \lstinline{Some(a) = m}}
\end{tabularx}

Dimostrazione dell'associatività, ovvero che \lstinline{(m.flatMap(f)).flatMap(g) = m.flatMap(x => f(x).flatMap(g))}:

\begin{tabularx}{\textwidth}{ll}
  Procedo per casi su \lstinline{m}: & \\
  & \\
  \emph{Se \lstinline{m = None}} & \\
  \\
  \lstinline{\ \ (m.flatMap(f)).flatMap(g) =} & \emph{Per ipotesi \lstinline{m = None}}\\
  \\
  \lstinline{\ \ (None.flatMap(f)).flatMap(g) =} & \emph{Definizione di \lstinline{flatMap}}\\
  \\
  \lstinline{\ \ None.flatMap(g) =} & \emph{Definizione di \lstinline{flatMap}}\\
  \\
  \lstinline{\ \ None =} & \emph{Definizione di \lstinline{flatMap}}\\
  \\
  \lstinline{\ \ None.flatMap(x => f(x).flatMap(g)) =} & \emph{Per ipotesi \lstinline{None = m}}\\
  \\
  \lstinline{\ \ m.flatMap(x => f(x).flatMap(g))} &  \\
  \\
  \emph{Se \lstinline{m = Some(a)}} & \\
  \\
  \lstinline{\ \ (m.flatMap(f)).flatMap(g) =} & \emph{Per ipotesi \lstinline{m = Some(a)}}\\
  \\
  \lstinline{\ \ (Some(a).flatMap(f)).flatMap(g) =} & \emph{Definizione di \lstinline{flatMap}}\\
  \\
  \lstinline{\ \ f(a).flatMap(g) =} & \emph{Applicazione di funzione}\\
  \\
  \lstinline{\ \ (x => f(x).flatMap(g))(a) =} & \emph{Definizione di \lstinline{flatMap}} \\
  \\
  \lstinline{\ \ Some(a).flatMap(x => f(x).flatMap(g)) =} & \emph{Per ipotesi \lstinline{Some(a) = m}} \\
  \\
  \lstinline{\ \ m.flatMap(x => f(x).flatMap(g))}$\qed$ & 
\end{tabularx}


\section{Dimostrazione per la monade State}
\label{dimostrazione-per-la-monade-state}

Alla \Cref{la-monade-state} è stata data una definizione di monade per \lstinline{State}. In seguito è riportata una dimostrazione del rispetto delle leggi monadiche per la definizione fornita.

Dimostrazione dell'identità sinistra, ovvero che \lstinline{pure(a).flatMap(f) = f(a)}:

\begin{tabularx}{\textwidth}{ll}
\lstinline{pure(a).flatMap(f) =}               & \emph{Definizione di \lstinline{pure}}\\
\\
\lstinline{State(s => (a, s)).flatMap(f) =}    & \emph{Sia \lstinline{m = State(s => (a, s))}}\\
\\
\lstinline{m.flatMap(f) =}                     & \emph{Definizione di 
\lstinline{flatMap}}\\
\\
\lstinline{State(s0 =>} \\
\lstinline{\ \ val (res1, s1) = m.runState(s0)}\\
\lstinline{\ \ f(res1).runState(s1)) =}      & \emph{Definizione di \lstinline{runState}}\\
\\
\lstinline{State(s0 =>} \\
\lstinline{\ \ val (res1, s1) = (s => (a, s))(s0)}\\
\lstinline{\ \ f(res1).runState(s1)) =}      & \emph{Applicazione di funzione}\\
\\
\lstinline{State(s0 =>} \\
\lstinline{\ \ val (res1, s1) = (a, s0)}\\
\lstinline{\ \ f(res1).runState(s1)) =}      & \emph{Pattern matching su una tupla}\\
\\
\lstinline{State(s0 => f(a).runState(s0)) =} & \emph{Sia \lstinline{f(a) = State(g)}}\\
\\
\lstinline{State(s0 => State(g).runState(s0)) =} & \emph{Definizione di \lstinline{runState}}\\
\\
\lstinline{State(s0 => g(s0)) =}                   & \emph{$\eta$-riduzione}\\
\\
\lstinline{State(g) =}                             & \emph{Per definizione di \lstinline{f(a)}}\\
\\
\lstinline{f(a)}$\qed$ &
\end{tabularx}

Dimostrazione dell'identità destra, ovvero che \lstinline{m.flatMap(pure) = m}:

\begin{tabularx}{\textwidth}{ll}
\lstinline{m.flatMap(pure) =} & \emph{Definizione di \lstinline{flatMap}}\\
\\
\lstinline{State(s0 =>} \\
\lstinline{\ \ val (res1, s1) = m.runState(s0)}\\
\lstinline{\ \ pure(res1).runState(s1)) =} & \emph{Definizione di \lstinline{pure}}\\
\\
\lstinline{State(s0 =>} \\
\lstinline{\ \ val (res1, s1) = m.runState(s0)}\\
\lstinline{\ \ State(s => (res1, s))} \\
\lstinline{\ \ \ \ .runState(s1)) =}& \emph{Definizione di \lstinline{runState}}\\
\\
\lstinline{State(s0 =>} \\
\lstinline{\ \ val (res1, s1) = m.runState(s0)}\\
\lstinline{\ \ (s => (res1, s))(s1)) =} & \emph{Applicazione di funzione}\\
\\
\lstinline{State(s0 =>} \\
\lstinline{\ \ val (res1, s1) = m.runState(s0)}\\
\lstinline{\ \ (res1, s1)) =} & \emph{Eliminazione pattern matching}\\
\\
\lstinline{State(s0 => m.runState(s0)) =} & \emph{Sia \lstinline{m = State(g)}} \\
\\
\lstinline{State(s0 => State(g).runState(s0)) =} & \emph{Definizione di \lstinline{runState}} \\
\\
\lstinline{State(s0 => g(s0)) =} & \emph{$\eta$-riduzione} \\
\\
\lstinline{State(g) =} & \emph{Per definizione di \lstinline{m}} \\
\\
\lstinline{m} $\qed$ 
\end{tabularx}

Dimostrazione dell'associatività, ovvero che \lstinline{(m.flatMap(f)).flatMap(g) = m.flatMap(x => f(x).flatMap(g))}:

\begin{tabularx}{\textwidth}{ll}
\lstinline{(m.flatMap(f)).flatMap(g) =} & \emph{Definizione di \lstinline{flatMap}}\\
\\
\lstinline{State(s0 =>} \\
\lstinline{\ \ val (res2, s2) =}\\
\lstinline{\ \ \ \ (m.flatMap(f)).runState(s0)} \\
\lstinline{\ \ g(res2).runState(s2)) =} & \emph{Definizione di \lstinline{flatMap}}\\
\\
\lstinline{State(s0 =>} \\
\lstinline{\ \ val (res2, s2) = (State(s =>} \\
\lstinline{\ \ \ \ val (res1, s1) = m.runState(s)}\\
\lstinline{\ \ \ \ f(res1).runState(s1)}\\
\lstinline{\ \ \ \ .runState(s0)} \\
\lstinline{\ \ g(res2).runState(s2)) =} & \emph{Definizione di \lstinline{runState}}\\
\\
\lstinline{State(s0 =>} \\
\lstinline{\ \ val (res2, s2) =} \\
\lstinline{\ \ \ \ val (res1, s1) = m.runState(s0)} \\
\lstinline{\ \ \ \ f(res1).runState(s1)))} \\
\lstinline{\ \ g(res2).runState(s2)) =} & \emph{Estrazione dichiarazioni locali} \\
\\
\lstinline{State(s0 =>} \\
\lstinline{\ \ val (res1, s1) = m.runState(s0)} \\
\lstinline{\ \ val (res2, s2) = f(res1).runState(s1)} \\
\lstinline{\ \ g(res2).runState(s2)) =} & \emph{Definizione di \lstinline{State}} \\
\\
\lstinline{State(s0 =>} \\
\lstinline{\ \ val (res1, s1) = m.runState(s0)} \\
\lstinline{\ \ State(s =>} \\
\lstinline{\ \ \ \ val (res2, s2) = f(res1).runState(s)} \\
\lstinline{\ \ \ \ g(res2).runState(s2))} \\
\lstinline{\ \ \ \ .runState(s1)) =} & \emph{Definizione di \lstinline{flatMap}} \\
\\
\lstinline{State(s0 =>} \\
\lstinline{\ \ val (res1, s1) = m.runState(s0)} \\
\lstinline{\ \ (f(res1).flatMap(g))} \\
\lstinline{\ \ \ \ .runState(s1)) =} & \emph{Applicazione di funzione} \\
\\
\lstinline{State(s0 =>} \\
\lstinline{\ \ val (res1, s1) = m.runState(s0)} \\
\lstinline{\ \ (x => f(x).flatMap(g))(res1)} \\
\lstinline{\ \ \ \ .runState(s1)) =} & \emph{Definizione di \lstinline{flatMap}} \\
\\
\lstinline{m.flatMap(x => f(x).flatMap(g))} $\qed$\\
\end{tabularx}



%----------------------------------------------------------------------------------------
% BIBLIOGRAPHY
%----------------------------------------------------------------------------------------
\nocite{*} % Show all elements in bibliography
\printbibliography[nottype=online, title={Riferimenti Bibliografici}]
\printbibliography[type=online, title={Riferimenti Sitografici}]

\end{document}