\documentclass[12pt,a4paper]{book}

\usepackage[utf8]{inputenc}
\usepackage{thesis-style}

\begin{document}

% ! TeX root = master-thesis.tex
\newgeometry{margin=0.8in}
\begin{titlepage}
	\begin{center}
		\large
		\textbf{ALMA MATER STUDIORUM -- UNIVERSITÀ DI BOLOGNA \\ CAMPUS DI CESENA}
		\\
		\noindent\hrulefill
		\vspace{0.4cm}

		\Large
		Scuola di Ingegneria e Architettura \\
		Corso di Laurea Magistrale in Ingegneria e Scienze Informatiche

		\Huge
		\vspace{4cm}
		\textbf{Gestione degli effetti in linguaggi di programmazione funzionale: tecniche di modellazione e interpretazione}

		\large
		\vspace{1cm}
		Tesi di laurea in
		\\
		\textsc{Paradigmi di Programmazione e Sviluppo}

		\vspace{5.5cm}
		\begin{minipage}[t]{0.64\textwidth}
			\begin{flushleft}
				\textit{Relatore}
				\\
				\textbf{Prof.} \textbf{Mirko Viroli}
				\\
				\vspace{0.4cm}
				\textit{Correlatore}
				\\
				\textbf{Dott.} \textbf{Roberto Casadei}
			\end{flushleft}
		\end{minipage}
		\begin{minipage}[t]{0.34\textwidth}
			\begin{flushright}
				\textit{Candidato}
				\\
				\textbf{Giacomo Cavalieri}
			\end{flushright}
		\end{minipage}\\

		\vfill
		\noindent\hrulefill
		\vspace{0.3cm}
		\Large

		Quarta Sessione di Laurea
		\\
		Anno Accademico 2021-2022
	\end{center}
\end{titlepage}
\restoregeometry
\tableofcontents
\newpage

\listoftodos

%\todo{Aggiungere una parte introduttiva di più largo respiro, con osservazioni sul paradigma funzionale e sull'espressività/efficacia della programmazione in questo paradigma}
\mustfix{Spiegare perché notazione haskell ed esempi di codice in scala nell'introduzione. Serve contestualizzare sul perché si cambia la notazione: si usa Scala come linguaggio di riferimento in quanto linguaggio funzionale con side effect è utile per mostrare esempi di codice sia monadico che con side effect.}

\chapter{Modellazione dei side effect}
\label{chapter:side-effect-e-loro-modellazione}

La ragion d'essere di un qualunque programma è quella di produrre un qualche effetto tangibile sul mondo esterno: che sia scrivere dei dati su disco, inviare un messaggio sulla rete, stampare dei caratteri a schermo ecc.
Linguaggi dalla natura imperativa come C, Java o Scala forniscono delle funzioni apposite per ottenere tali effetti; in Scala, per esempio, si potrebbe implementare la seguente funzione per apporre una stringa al contenuto di un file su disco\footnote{In questa funzione così come in tutte le altre funzioni Scala a seguire viene adottata la sintassi introdotta da Scala 3~\cite{cit:new-in-scala-3}.}:

\begin{lstlisting}[language=scala3]
def appendToFile(file: File, line: String): Unit =
  println(f"Appending $line to $file")
  val writer = FileWriter(file, true)
  try writer.write(f"$line\n")
  finally writer.close
\end{lstlisting}
Lo scopo di funzioni come \lstinline{appendToFile}, \lstinline{write} e \lstinline{println} non è quello di produrre un risultato -- il loro valore di ritorno è sempre \lstinline{Unit} -- ma di mettere in atto dei \emph{side effect}.

Più in generale, con side effect si intende una qualunque interazione della funzione con un ambiente diverso da quello locale.
Possono essere quindi considerati side effect il modificare una variabile globale o un parametro passato per riferimento, lanciare un'eccezione, l'effettuare operazioni di input e output -- come mostrato precedentemente in \lstinline{appendToFile} -- o il chiamare una funzione che a sua volta presenta dei side effect~\cite{cit:on-the-prevalence-of-function-side-effects-in-general-purpose-open-source-software-systems}.

Le funzioni che presentano side effect sono spesso anche dette \emph{impure} mentre funzioni che non hanno side effect sono anche dette \emph{pure} o caratterizzate da \emph{trasparenza referenziale}.
\section{Modellazione esplicita dei side effect}
\label{modellazione-esplicita-dei-side-effect}

Date le criticità evidenziate nella sezione precedente, diverse pratiche di buona programmazione suggeriscono di ridurre al minimo le funzioni che presentano side effect~\cite[p.~44]{cit:clean-code-a-handbook-of-agile-software-craftsmanship} e, quando inevitabili, di renderli espliciti nel nome della funzione~\cite[p.~313]{cit:clean-code-a-handbook-of-agile-software-craftsmanship}.

Tuttavia, si possono individuare altre tecniche più sofisticate per tracciare i side effect delle funzioni.
Queste nascono nel contesto dei linguaggi funzionali puri, perciò è necessario comprendere come questa classe di linguaggi possa conciliare la presenza di side effect con la loro natura puramente funzionale.

\subsection{Linguaggi funzionali puri e side effect}
\label{linguaggi-funzionali-puri-e-side-effect}
Un linguaggio funzionale si dice \emph{puro} se le sue funzioni sono referenzialmente trasparenti.
Questa totale assenza di side effect sembra in netto contrasto con la possibilità di scrivere codice di una qualche utilità: come può un programma puro interagire con il mondo esterno, leggere il valore di uno stato globale o lanciare eccezioni?

La soluzione consiste nella possibilità di ``simulare'' la presenza di side effect tramite opportune modifiche ai tipi delle funzioni mantenendole pure.
In seguito sono riportati alcuni esempi di questo approccio in Scala\footnote{Sebbene Scala sia un linguaggio impuro può essere utilizzato come se fosse un linguaggio puro evitando di ricorrere ai meccanismi che fornisce per produrre side effects.}.

\subsubsection{Eccezioni}
\label{eccezioni}
Le eccezioni sono un meccanismo di controllo del flusso che permette di interrompere l'esecuzione di una funzione e di ritornare il controllo al chiamante.
Tuttavia, una funzione che ne fa uso non avrà trasparenza referenziale:
\begin{lstlisting}[language=scala3]
def divBy(n: Int): Int =
  n match
    case 0 => throw Exception("n = 0")
    case _ => 10 / n
\end{lstlisting}
\lstinline|divBy| non può essere assimilata a una funzione $divBy : \mathbb{Z} \rightarrow \mathbb{Z}$ in termini matematici; infatti, per certi input -- in questo caso 0 -- la funzione non restituisce un valore appartenente al proprio codominio ma lancia un'eccezione.

Per rendere esplicito il fatto che la funzione possa terminare in maniera anomala per determinati input si può estenderne il codominio: ritornando all'analogia matematica si può pensare ad \lstinline{divBy} come a una funzione $divBy : \mathbb{Z} \rightarrow \mathbb{Z} \cup \{ \texttt{Error} \}$.
  L'equivalente implementazione Scala sarebbe:
  \begin{lstlisting}[language=scala3]
enum Result:
  case Ok(value: Int)
  case Error

def safeDivBy(n: Int): Result =
  n match
    case 0 => Error
    case _ => Ok(10 / n)
\end{lstlisting}
  In questo caso è evidente dal tipo di ritorno di \lstinline|safeDivBy| che questa può fallire restituendo un valore di tipo \lstinline{Error}.
  La funzione non nasconde questo comportamento tramite il meccanismo delle eccezioni, lo rende invece evidente nel proprio tipo \lstinline{Int => Result}.
  Il grande vantaggio di rendere esplicita la possibilità di fallimento nel tipo di ritorno sta nel fatto che il programmatore dovrà obbligatoriamente gestire anche il caso in cui la computazione fallisca o il compilatore solleverà un errore a tempo di compilazione:
  \begin{lstlisting}[language=scala3]
def useSafeDivBy(n: Int): Int =
  // safeDivBy(n) + safeDivBy(n + 1) darebbe un errore
  // a tempo di compilazione dato che safeDivBy(x)
  // ha come tipo Result e non Int
  safeDivBy(n) match
    case Error      => 0
    case Ok(value1) => safeDivBy(n + 1) match
      case Error      => 0
      case Ok(value2) => value1 + value2
\end{lstlisting}

  \subsubsection{Lettura e modifica di uno stato globale}
  \label{lettura-e-modifica-di-uno-stato-globale}

  Dal momento in cui una funzione legge una variabile globale perde la propria trasparenza referenziale: il suo risultato non dipende più dai soli valori passati in input ma da un nuovo input \emph{nascosto}, lo stato a cui accede.
  La soluzione per rendere esplicita questa dipendenza è passare come parametri di tutti gli elementi necessari al funzionamento della funzione, senza affidarsi alla definizione in uno scope esterno di variabili globali a cui accedere implicitamente.

  Questa semplice trasformazione permette di rimuovere il side effect che consiste nella lettura di uno stato globale.
  Sfortunatamente, non è sufficiente per modellare anche la \emph{modifica} di una variabile globale. In questo caso può essere utile tornare all'analogia con le funzioni matematiche.
  Prendiamo come esempio la funzione \lstinline{f} definita in precedenza:
  \begin{lstlisting}[language=scala3]
var counter = 0
def f(x: Int): Int =
  counter = counter + 1 
  x + counter
\end{lstlisting}
  Questa prende in input un numero, ha il side effect di incrementare un contatore globale e poi ne legge il valore per aggiungerlo all'input. Sebbene il tipo di \lstinline{f} sia \lstinline{Int => Int}, a causa dei suoi side effect questa funzione non può essere modellata come una funzione matematica $f: \mathbb{Z} \rightarrow \mathbb{Z}$: a parità di input non darà sempre lo stesso output.

  Dato che la funzione necessita di leggere una variabile definita esternamente, questa dovrà essergli passata in come input esplicitamente:
  \begin{lstlisting}[language=scala3]
def wrongF(x: Int, counter: Int): Int =
  // Come modificare lo stato di current?
  val newCounterState = counter + 1
  x + newCounterState
\end{lstlisting}
  Rimane il problema di come poter modellare la modifica dello stato globale in modo che tale effetto si ripercuota anche su chiamate successive. Infatti, il side effect di aumentare il valore del contatore fa parte della logica applicativa di \lstinline{f} e rimuoverlo comporterebbe un cambiamento nella sua semantica.

  La soluzione consiste nel trasformare la funzione in modo tale che restituisca il valore del nuovo stato così che possa essere utilizzato per chiamate successive:
  \begin{lstlisting}[language=scala3]
def betterF(x: Int, counter: Int): (Int, Int) =
  val newCounter = counter + 1
  val result = x + newCounter
  (result, newCounter)
\end{lstlisting}
  Quindi lo stato dovrà essere passato in maniera esplicita da una chiamata a funzione alla successiva:
  \begin{lstlisting}[language=scala3, label=lst:use-better-f]
def useBetterF: (Int, Int) =
  val startingCounter = 1
  val (result1, counter1) = betterF(1, startingCounter)
  // Il nuovo stato counter1 viene passato alla seconda
  // chiamata di betterF
  val (result2, counter2) = betterF(1, counter1)
  // Il nuovo stato counter2 viene passato alla terza
  // chiamata di betterF
  val (result3, finalCounter) = betterF(1, counter2)
  (result1 + result2 + result3, finalCounter)
\end{lstlisting}
  La funzione non solo prende in input lo stato a cui deve accedere ma restituisce in output la nuova versione dello stato modificato\footnote{In un linguaggio con strutture dati mutabili come Scala la funzione potrebbe anche non restituire la nuova versione dello stato ma modificare lo stato ricevuto come argomento per riferimento. Tuttavia, come descritto in precedenza anche la modifica degli argomenti è un side effect; questo approccio non risolverebbe il problema del poter tracciare esplicitamente gli effetti di una funzione.}, a questo punto è nuovamente possibile modellarla come una funzione matematica $f : \mathbb{Z}\times\mathbb{Z} \rightarrow \mathbb{Z}\times\mathbb{Z}$ che, per ogni coppia di valori ricevuti in input, darà sempre lo stesso risultato.

In questo modo diventa più facile testare il comportamento della funzione:
\begin{lstlisting}[language=scala3]
def testBetterF(): Unit =
  val counter = 0
  val (result, newCounter) = betterF(1, counter)
  result shouldBe 2
  newCounter shouldBe 1
\end{lstlisting}
Il test è completamente autonomo, non sono più necessarie operazioni preliminari di preparazione dello stato: l'output di \lstinline{betterF}, infatti, dipende unicamente dai valori passati in input e non da una qualche variabile globale che potrebbe essere utilizzata e modificata anche da altri test.

\subsection{Svantaggi della modellazione esplicita}
\label{svantaggi-della-modellazione-esplicita}
Con le trasformazioni elencate in precedenza è possibile mantenere la trasparenza referenziale delle funzioni modellandone i side effect in maniera esplicita. Questo ha il vantaggio di aiutare il programmatore nella composizione delle funzioni rendendo chiaro, a tempo di compilazione, quale potrebbe essere il loro comportamento.

Nonostante ciò, come forse si è già potuto intuire da alcuni degli esempi riportati, questo approccio rende necessario lo scrivere codice spesso molto più verboso.
In seguito sono riportati due esempi di questo problema e delle soluzioni ad hoc che possono essere adottate per porvi rimedio.

\subsubsection{Gestione del fallimento di una funzione}
Consideriamo la seguente funzione:
\begin{lstlisting}[language=scala3]
def halve(n: Int): Option[Int] =
  n % 2 match
    case 0 => Some(n / 2)
    case _ => None 
\end{lstlisting}
Nel caso in cui il numero passato come input sia pari ne restituirà la metà, altrimenti fallirà (il side effect del fallimento è reso esplicito tramite l'uso del tipo \lstinline{Option}). Per realizzare una funzione analoga che, se l'input è divisibile per 8 ne restituisca il risultato della divisione, un programmatore potrebbe comporre insieme   più chiamate a \lstinline{halve}:
\begin{lstlisting}[language=scala3]
def eighth(n: Int): Option[Int] =
  halve(n) match
    case None       => None
    case Some(half) => halve(half) match
      case None         => None
      case Some(fourth) => halve(fourth)
\end{lstlisting}
Nonostante la semplicità della funzione, si può osservare come il dover gestire in maniera esplicita il fallimento di ogni chiamata ad \lstinline{halve} renda il codice più verboso e meno leggibile. Nella funzione si mescolano la logica applicativa -- il dover dividere ripetutamente per due -- e la gestione del possibile fallimento della divisione intera -- il \emph{pattern matching} sul risultato.

Il problema può essere risolto osservando una struttura comune a tutti i passaggi: se uno qualunque dei risultati intermedi restituisce \lstinline{None} allora la funzione fallisce immediatamente restituendo \lstinline{None}, come se si fosse verificata un'eccezione; altrimenti, si prosegue continuando a dividere il valore ottenuto.
Questo comportamento può essere fattorizzato in un'apposita funzione\footnote{In Scala per poter utilizzare \lstinline{andThen} come una funzione infissa questa deve essere dichiarata come \emph{extension method}~\cite{cit:scala-extension-methods} degli oggetti di tipo \lstinline{Option}.}:
\begin{lstlisting}[language=scala3]
extension [A](a: Option[A])
	def andThen[B](f: A => Option[B]): Option[B] =
		a match
			case None         => None
			case Some(result) => f(result)
\end{lstlisting}
che permetterà di scrivere la funzione \lstinline{eighth} in maniera molto più chiara:
\begin{lstlisting}[language=scala3]
def eighth(n: Int): Option[Int] =
  halve(n).andThen(halve).andThen(halve)
\end{lstlisting}
La funzione è solo interessata dalla logica applicativa: dividere tre volte il valore in input. La gestione del fallimento è delegata ad \lstinline{andThen}\footnote{In Scala è già definito per \lstinline{Option} il metodo \lstinline{flatMap} con lo stesso comportamento di \lstinline{andThen} mostrato nell'esempio. Per questi esempi si preferisce utilizzare \lstinline{andThen} in quanto rende più chiaro il ruolo della funzione.} che, in caso di fallimento, restituisce immediatamente \lstinline{None}.

\nicetohave{Per completezza si potrebbbe anche mostrare come alcuni linguaggi (come Kotlin, Rust) adottano un approccio analogo ma con un supporto diretto del compilatore e annesso zucchero sintattico? In questo caso la soluzione è ad hoc e non è generica come per le monadi ma comunque interessante}

\subsubsection{Gestione della lettura e modifica di uno stato globale}
\label{gestione-della-lettura-e-modifica-di-uno-stato-globale}

Come già discusso, è possibile modificare una funzione che accede e modifica uno stato globale in una funzione pura; in modo che riceva lo stato globale come input e ne restituisca la nuova versione in output.
Lo svantaggio di tale approccio sta nel fatto che il programmatore dovrà gestire manualmente il passaggio dello stato fra le diverse chiamate; ciò, oltre ad aumentare il \emph{boilerplate,} rende più probabile compiere errori come dimenticare di passare lo stato aggiornato a una chiamata successiva:
\begin{lstlisting}[language=scala3]
...
val (res1, state1) = statefulFunction1(initialState)
val (res2, state2) = statefulFunction2(res1, state1)
val (res3, state3) = statefulFunction3(res2, state2)
statefulFunction4(res3, state2)
// bug: statefulFunction4 ha ricevuto come input lo
// stato precedente a quello aggiornato da 
// statefulFunction3!
...
\end{lstlisting}

Anche in questo caso si può ottenere una soluzione che permetta di separare la logica applicativa dalla gestione manuale del passaggio dello stato fra un chiamata e la successiva:
\begin{lstlisting}[language=scala3]
extension [A, S](f: S => (A, S))
  def andThen[B](g: A => S => (B, S)): S => (B, S) =
    state0 => 
      val (result, state1) = f(state0)
      g(result)(state1)
\end{lstlisting}
Grazie a questa funzione è possibile riscrivere la funzione dell'esempio precedente in maniera più chiara:
\begin{lstlisting}[language=scala3]
...
statefulFunction1
  .andThen(statefulFunction2)
  .andThen(statefulFunction3)
  .andThen(statefulFunction4)
  .apply(initialState)
...
\end{lstlisting}
La gestione del \emph{threading} dello stato fra una chiamata e la successiva è delegato ad \lstinline{andThen}; in questo modo è possibile indicare la sequenza di operazioni che si vogliono compiere e lasciare che il boilerplate relativo al passaggio dello stato fra una chiamata e la successiva venga gestito automaticamente.

\section{Modellazione degli effetti tramite monadi}
Osservando le soluzioni adottate nella Sezione precedente si può osservare come a queste sottende un meccanismo comune. Infatti in entrambi i casi è stata definita una funzione \lstinline{andThen} per permettere di combinare in sequenza più operazioni con side effect. Il risultato ottenuto è una descrizione dichiarativa della sequenza di operazioni da svolgere; il particolare meccanismo con cui le operazioni vengono concatenate -- gestione del fallimento prematuro, o passaggio implicito dello stato ad ogni passaggio -- viene delegato alla funzione \lstinline{andThen}.

Questo meccanismo può essere catturato dall'astrazione delle monadi.

\subsection{Cos'è una monade?}
\label{cos-e-una-monade}
Il concetto di monade, nato nell'ambito della teoria delle categorie~\cite{cit:categories-for-the-working-mathematician}, venne utilizzato da Eugenio Moggi come mezzo per strutturare la semantica denotazionale di aspetti di un programma come lo stato mutabile, la gestione delle eccezioni e delle continuazioni~\cite{cit:an-abstract-view-of-programming-languages}.
Fu poi Philip Wadler, ispirato dal lavoro di Moggi e Michael Spivey~\cite{cit:a-functional-theory-of-exceptions}, a intuire che questa stessa tecnica potesse essere sfruttata direttamente per \emph{strutturare} un programma funzionale -- non solo per descriverne la semantica come fatto da Moggi~\cite{cit:comprehending-monads,cit:the-essence-of-functional-programming}.

Una monade è una tripla \lstinline{(M, pure, flatMap)} dove:
\begin{itemize}
  \item \lstinline{M} è un costruttore di tipi; ovvero prende in input un tipo \lstinline{A} e restituisce un tipo \lstinline{M[A]}. Un valore di tipo \lstinline{M[A]} può essere interpretato come una computazione che restituisce un valore di tipo \lstinline{A} e può avere un qualche side effect
  \item \lstinline{pure} è una funzione polimorfa con tipo \lstinline{A => M[A]}\footnote{\lstinline{pure} è spesso anche indicato come \lstinline{return}.}
  \item \lstinline{flatMap} è una funzione polimorfa con tipo \lstinline{(M[A], A => M[B]) => M[B]}\footnote{\lstinline{flatMap} è anche indicato come \lstinline{bind} o \lstinline{>>=}. Nella sua versione \lstinline{>>=} viene generalmente utilizzato come operatore binario infisso: vale a dire che \lstinline{m >>= f} è equivalente a indicare \lstinline{flatMap(m, f)}.}; rappresenta la combinazione in sequenza di due computazioni che possono presentare side effect
\end{itemize}
Inoltre è richiesto che valgano le seguenti \emph{leggi monadiche}:
\begin{itemize}
  \item (Identità sinistra) \lstinline{pure(a) >>= f = f(a)}
  \item (Identità destra) \lstinline{m >>= pure = m}
  \item (Associatività) \lstinline{(m >>= f) >>= g = m >>= (x => f(x) >>= g)}
\end{itemize}
Le prime due leggi servono a garantire che \lstinline{pure} sia l'elemento neutro per l'operazione di concatenazione \lstinline{flatMap}: \lstinline{pure} può essere quindi visto come l'operazione che trasforma un valore di tipo \lstinline{A} in un valore di tipo \lstinline{M[A]} senza compiere alcun side effect.
La terza legge, garantisce che \lstinline{flatMap} sia associativa, dunque scrivere \lstinline{m >>= f >>= g} è equivalente a scrivere \lstinline{(m >>= f) >>= g} o \lstinline{m >>= (x => f(x) >>= g)}.

\subsection{Encoding di una monade}
È possibile esprimere le soluzioni ad hoc adottate alla \Cref{svantaggi-della-modellazione-esplicita} in termini del concetto di monade, garantendo un'interfaccia uniforme per diversi side effect.
Tuttavia, prima di poter generalizzare gli esempi precedenti è necessario capire come il concetto di monade possa essere implementato in un linguaggio di programmazione.

\subsubsection{Encoding in Haskell}
Come descritto in precedenza una monade è composta da tre elementi fondamentali: un costruttore di tipo e due funzioni \lstinline{return} e \lstinline{>>=}. In Haskell è possibile esprimere direttamente tale concetto tramite l'uso di una \emph{type class}, un meccanismo utilizzato per supportare il \emph{polimorfismo ad hoc}~\cite{cit:type-classes-in-haskell}:
\begin{lstlisting}[language=haskell]
class Monad m where
  return :: a -> m a
  (>>=)  :: m a -> (a -> m b) -> m b
\end{lstlisting}

La dichiarazione di una \emph{type class} può essere interpretata come un predicato su un tipo o, come in questo caso, su un costruttore di tipi.
Quindi, la precedente definizione può essere letta come: \emph{``Un generico costruttore di tipi \lstinline{m} è una monade se esistono due funzioni \lstinline{return} con tipo \lstinline{a -> m a} e \lstinline{>>=} con tipo \lstinline{m a -> (a -> m b)   -> m b}''}.
Dato un concreto costruttore di tipi si può istanziare la \emph{type class} \lstinline{Monad} fornendo l'implementazione delle funzioni che questa dichiara:
\begin{lstlisting}[language=haskell]
instance Monad Maybe where
  return = Just
  Nothing >>= f = Nothing
  Just x  >>= f = f x
\end{lstlisting}

Interpretando la \emph{type class} come un predicato, definire un'istanza consiste nel provare che lo specifico tipo -- in questo caso \lstinline{Maybe} -- soddisfa tale predicato e viene fornita come dimostrazione l'implementazione delle funzioni \lstinline{return} e \lstinline{>>=}.

\subsubsection{Encoding in Scala}
In Scala è possibile codificare una \emph{type class} sfruttando il passaggio implicito di parametri~\cite{cit:type-classes-as-objects-and-implicits}. La definizione di una \emph{type class} si riduce quindi a indicare un'interfaccia:
\scalaFromFile{8}{10}{monads/Monad.scala}

Potrà essere istanziata per uno specifico costruttore di tipi fornendo un'implementazione come istanza implicita~\cite{cit:scala-book-type-classes}:
\scalaFromFile{15}{21}{monads/Monad.scala}

\subsection{Esempi di monadi}
Gli esempi mostrati alla \Cref{svantaggi-della-modellazione-esplicita} presentano una struttura comune: si definisce una struttura dati che descrive il risultato di un'operazione con il possibile side effect e si definisce poi una funzione -- negli esempi chiamata \lstinline{andThen} -- che permette di combinare in sequenza passaggi intermedi.

Si può osservare come in entrambi i casi la funzione \lstinline{andThen} abbia la stessa firma descritta per \lstinline{flatMap}: infatti, i casi mostrati in precedenza non sono altro che esempi di monadi. In seguito viene formalizzata la definizione di monade per entrambi gli esempi riportando in aggiunta l'implementazione di \lstinline{pure}. Inoltre viene mostrata la definizione di una monade banale che non esegue alcun side effect.

\subsubsection{La monade identità}
\label{la-monade-identita}
La più semplice monade possibile è quella che non applica alcun side effect. Tale monade può essere definita come segue:
\scalaFromFile{3}{10}{monads/Identity.scala}

\begin{itemize}
  \item \lstinline{Identity} è il costruttore di tipi: preso un tipo \lstinline{A} restituisce un tipo \lstinline{Identity[A]} che rappresenta una computazione che produce un valore di tipo \lstinline{A} senza attuare alcun side effect
  \item \lstinline{pure} permette di trasformare un valore di tipo  \lstinline{A} in uno di tipo \lstinline{Identity[A]}. In questo caso corrisponde alla funzione identità che non modifica il valore di tipo \lstinline{A}
  \item \lstinline{flatMap} permette di mettere in sequenza valori di tipo \lstinline{Identity}; dato che la monade non introduce alcun side effect corrisponde all'applicazione di funzione
\end{itemize}

Una dimostrazione del rispetto delle leggi monadiche è riportata in \Cref{dimostrazione-per-la-monade-identita}.

L'utilità di una monade che non introduce alcun side effect sarà resa evidente nel \Cref{chapter:stack-di-monadi}.

\subsubsection{La monade Option}
\label{la-monade-optional}
Per gestire il fallimento prematuro di una funzione era stato sfruttato il costruttore di tipi \lstinline{Option} e la funzione \lstinline{andThen} per concatenare in sequenza passaggi intermedi e propagare il fallimento.
Si può mostrare come \lstinline{Option} sia una monade implementando l'operazione \lstinline{>>=} come \lstinline{andThen}:
\begin{lstlisting}[language=scala3]
enum Option[+A]:
  case Some(a: A)
  case None

given Monad[Option] with
	def pure[A](a: A): Option[A] = Some(a)
	extension [A](m: Option[A]) 
		def flatMap[B](f: A => Option[B]): Option[B] =
			m match
        case None    => None
				case Some(a) => f(a)
\end{lstlisting}

\begin{itemize}
  \item \lstinline{Option} è il costruttore di tipi: preso un tipo \lstinline{A}, restituisce un tipo \lstinline{Option[A]} che rappresenta una computazione che può fallire o produrre un valore di tipo \lstinline{A}
  \item \lstinline{pure} permette di trasformare un valore di tipo  \lstinline{A} in uno di tipo \lstinline{Option[A]} senza avere side effect. In questo caso il side effect sarebbe fallire restituendo \lstinline{None}; quindi \lstinline{pure(a)} restituisce \lstinline{Some(a)}
  \item \lstinline{flatMap} permette di concatenare in sequenza passaggi intermedi propagando il fallimento
\end{itemize}

Come descritto in precedenza, perché \lstinline{Option} sia effettivamente una monade deve rispettare le tre leggi monadiche; una dimostrazione è riportata all'\Cref{dimostrazione-per-la-monade-optional}.

\subsubsection{La monade State}
\label{la-monade-state}

È possibile definire un'istanza di monade anche per l'esempio dello stato globale mutabile mostrato alla \Cref{gestione-della-lettura-e-modifica-di-uno-stato-globale}; per fare ciò può tornare utile definire prima un'apposita struttura che incapsula una funzione che prende in input lo stato e restituisce il risultato e lo stato aggiornato:
\scalaFromFile{5}{5}{monads/State.scala}

Un problema nel definire l'istanza di monade per \lstinline{State} sta nel fatto che questo è un costruttore di tipi che accetta in input \emph{due tipi} \lstinline{S} e \lstinline{A} -- quello dello stato manipolato e quello del risultato -- e restituisce un tipo \lstinline{State[S, A]}. Per poter essere un costruttore valido secondo la definizione di monade deve prendere in input un solo tipo; per ovviare a tale problema si può fissare uno dei due tipi e lasciare l'altro libero: in questo caso si è scelto di fissare il tipo dello stato \lstinline{S} e lasciare libero di variare il tipo del risultato\footnote{In Scala 3 si può ottenere questa applicazione parziale del costruttore di tipi utilizzando il meccanismo delle \emph{type lambda}~\cite{cit:scala-reference-type-lambdas}; quindi per fissare il tipo \lstinline{S} e lasciare libero \lstinline{A} in \lstinline{State} si può utilizzare la seguente sintassi: \lstinline{[A] =>> State[S, A]}. Nel codice riportato nell'esempio viene utilizzata una sintassi abbreviata che utilizza il carattere \lstinline{_} come segnaposto nella lambda a livello di tipi in maniera analoga a come viene utilizzato per le lambda a livello di termini~\cite{cit:scala-reference-wildcard-arguments-in-types}. Tale sintassi non è ancora stata adottata come default in Scala 3 ma verrà introdotta in futuro, al momento è disponibile utilizzando l'estensione del compilatore \lstinline{"-Ykind-projector:underscores"}~\cite{cit:scala-reference-kind-projector-migration}.}:
\scalaFromFile{11}{18}{monads/State.scala}

\begin{itemize}
  \item \lstinline{State[S, _]} è il costruttore di tipi: preso un tipo \lstinline{A} restituisce un tipo \lstinline{State[S, A]} che rappresenta una computazione che può modificare uno stato globale di tipo \lstinline{S} e restituisce un valore di tipo \lstinline{A}
  \item \lstinline{pure} permette di trasformare un valore di tipo  \lstinline{A} in uno di tipo \lstinline{State[S, A]} senza avere side effect. Il side effect sarebbe modificare lo stato globale, quindi in questo caso lo stato globale viene restituito inalterato e il risultato è il valore passato in input
  \item \lstinline{flatMap} permette di concatenare in sequenza operazioni che operano su uno stato globale mutabile di tipo \lstinline{S} e passa in automatico la sua versione aggiornata da un una chiamata alla successiva
\end{itemize}

Come mostrato nell'\Cref{dimostrazione-per-la-monade-state} l'istanza di monade per \lstinline{State} rispetta le tre leggi monadiche.

Nel caso della monade \lstinline{State} alcune operazioni di base sono quelle che permettono di leggere o modificare il valore dello stato mutabile:
\scalaFromFile{8}{9}{monads/State.scala}
Queste funzioni possono essere utilizzate come base per ottenere operazioni più complesse. Un esempio più articolato che combina queste operazioni è dato dalla seguente funzione \lstinline{incrementCounter}: questa ha come tipo \lstinline{State[Int, String]} dunque può accedere a uno stato mutabile di tipo \lstinline{Int} e produce come risultato una stringa:
\scalaFromFile{21}{28}{monads/State.scala}
Come prima azione accede allo stato globale con \lstinline{get}, ne aumenta il valore con \lstinline{set} e infine legge nuovamente il valore dopo averlo modificato con un'ultimo \lstinline{get}. Il risultato è una stringa contenente il nuovo valore appena letto.
Si noti come, per mettere in sequenza le operazioni di lettura e scrittura sia necessario utilizzare \lstinline{flatMap}. Per poter eseguire la computazione con uno specifico stato iniziale sarà sufficiente utilizzare il metodo \lstinline{runState}:
\begin{lstlisting}[language=scala3]
incrementCounter.runState(0) // -> ("counter is: 1", 1)
incrementCounter.runState(2) // -> ("counter is: 3", 3)
\end{lstlisting}

\subsection{Zucchero sintattico per codice monadico}
Osservando l'esempio precedente è possibile notare come la messa in sequenza delle operazioni tramite l'uso di \lstinline{flatMap} può rendere il codice più complesso da leggere: infatti, ad ogni concatenazione successiva aumenta il livello di annidamento del codice portando alla cosiddetta \emph{pyramid of doom}.
Scrivere codice di questo genere diventerebbe impraticabile molto rapidamente anche per brevi sequenze di funzioni concatenate con \lstinline{flatMap}.

Per questo motivo, linguaggi come Haskell e Scala forniscono dello zucchero sintattico che permette di scrivere codice monadico in modo più leggibile e con un aspetto più ``imperativo''.

\subsubsection{\emph{For comprehension} in Scala}
\label{sec:for-comprehension-in-scala}
Per risolvere questo problema, Scala fornisce la \emph{for comprehension}~\cite{cit:scala-book-control-structures}; per esempio la funzione \lstinline{incrementCounter} mostrata in precedenza può essere scritta in modo equivalente come segue:
\scalaFromFile{30}{35}{monads/State.scala}

Il codice appare come una sequenza di operazioni imperative. In realtà, il compilatore Scala traduce il codice in una sequenza di chiamate a \lstinline{flatMap} e \lstinline{pure}.
Le regole adottate per il \emph{desugaring} possono essere descritte ricorsivamente come segue\footnote{In realtà nella \emph{for comprehension} sarebbe possibile utilizzare anche altre espressioni che non siano nella forma \lstinline{name <- expr} ma, per semplicità, non verranno considerate in questa sezione.}:

\begin{lstlisting}
for {name <- expr} yield res = expr.map(name => res)

for {name <- expr; exprs} yield res =
  expr.flatMap(name => for {exprs} yield res)
\end{lstlisting}

La funzione \lstinline{map} utilizzata nel \emph{desugaring} di una singola espressione è semanticamente equivalente, per una monade, alla seguente composizione di di \lstinline{flatMap} e \lstinline{pure}: \lstinline{m.map(f) = m.flatMap(x => pure(f(x)))}.

\subsubsection{\emph{Do notation} in Haskell}
Haskell adotta una soluzione analoga fornendo la cosiddetta \emph{do notation}; il precedente codice Scala preso ad esempio potrebbe essere implementato analogamente in Haskell come segue:
\begin{lstlisting}[language=haskell]
incrementCounter = do
  counter <- get
  set (counter + 1)
  newCounter <- get
  return ("counter is:" ++ show newCounter)
\end{lstlisting}
Anche in questo caso, il compilatore traduce il codice in una serie di chiamate a \lstinline{>>=} e \lstinline{return}. In particolare, le regole per la traduzione sono le seguenti:
\begin{lstlisting}
do { expr } = expr

do { name <- expr; exprs } =
  expr >>= \name -> do { exprs }

do { expr; exprs } = expr >>= \_ -> do { exprs }
\end{lstlisting}

Quindi, la forma senza zucchero sintattico di \lstinline{incrementCounter} sarebbe:
\begin{lstlisting}[language=haskell]
incrementCounter =
  get >>= (\counter ->
    set (counter + 1) >>= (\_ ->
      get >>= (\newCounter ->
        return ("counter is:" ++ show newCounter))))
\end{lstlisting}

\subsection{Vantaggi nell'uso delle monadi}
\subsubsection{Separazione del codice impuro dal codice puro}
Un primo importante vantaggio sta nella possibilità di esprimere a livello di \emph{type system} quali funzioni presentano side effect e quali no. Questo è un aiuto fondamentale per il programmatore: infatti, per capire se una funzione è pura è sufficiente analizzarne il tipo, senza dover cercare di capirlo dal suo nome -- che potrebbe non rispecchiare l'effettivo comportamento della funzione -- o ispezionandone il corpo.
Consideriamo come esempio le seguenti funzioni:
\begin{lstlisting}[language=scala3]
def incrementCounter: State[Int, ()] = ...
def first[A](xs: List[A]): Option[A] = ...
def double(n: Int): Int = ...
\end{lstlisting}
Senza bisogno di conoscerne le implementazioni si può capire che la prima funzione può modificare uno stato mutabile di tipo intero e che la seconda funzione può fallire non producendo alcun valore. Inoltre è possibile capire che la terza funzione non ha side effect: non potrà fallire, modificare uno stato globale o effettuare operazioni di input o output; il suo output sarà determinato unicamente dal valore dei suoi parametri\footnote{In realtà, dato che in Scala è comunque possibile scrivere codice impuro la funzione \lstinline{double} potrebbe avere side effect; quanto detto vale sotto l'assunzione che il programmatore stia modellando esplicitamente i side effect tramite l'approccio descritto. In altri linguaggi come Haskell, invece, è il compilatore stesso a fare in modo che questa regola venga rispettata rendendo impossibile lo scrivere funzioni che non siano pure. Perciò, si avrebbe la certezza che una funzione come \lstinline{double} non possa avere side effect.}.

\subsubsection{Side effect come cittadini di prima classe del linguaggio}
Utilizzare il concetto di monade come astrazione unificante delle diverse tipologie di side effect ha un ulteriore vantaggio: le azioni con side effect sono valori di prima classe che possono essere combinati in maniera modulare e astraendo dallo specifico tipo di effetti.

È possibile definire l'equivalente di strutture di controllo imperative tramite funzioni generiche; per poter ripetere più volte gli effetti di un'azione è possibile implementare una funzione come segue:
\begin{lstlisting}[language=scala3]
extension [A, M[_]: Monad](m: M[A])
  def >>[B](other: M[B]) = m.flatMap(_ => other)

def repeat[M[_]: Monad, A]
  (times: Int)
  (action: M[A]): M[Unit] =
	times match
		case 0 => pure(())
		case n => action >> repeat(n-1)(action)
\end{lstlisting}

Se invece si volesse implementare una versione del ciclo \lstinline{for} che permetta di ripetere un qualunque side effect per tutti gli elementi di una lista si potrebbe implementare la seguente funzione:
\begin{lstlisting}[language=scala3]
def forLoop[A, M[_]: Monad]
  (as: List[A])
  (f: A => M[Unit]): M[Unit] = 
  as match 
    case Nil     => pure(())	
    case a :: as => f(a) >> forLoop(as)(f)
\end{lstlisting}
\lstinline{forLoop} prende in input una lista e una funzione che, dato un elemento della lista, determina l'azione da intraprendere; il risultato sarà un'unica computazione che esegue tutti i side effect dati dall'applicazione della funzione a ciascun elemento della lista. Poiché la funzione è generica sul tipo della monade è possibile utilizzarla per qualsiasi tipo di side effect!
Questa è una tecnica molto potente che lascia al programmatore la libertà di inventare le proprie strutture di controllo senza essere doversi limitare a quelle predefinite dal linguaggio~\cite{cit:tackling-the-awkward-squad}.

\section{Input e output puri}
\label{section:input-e-output-puri}

Nelle precedenti sezioni è stato mostrato come sia possibile ``simulare'' la presenza di side effect -- come eccezioni e modifica di uno stato mutabile -- con opportune modifiche al tipo delle funzioni in modo da rendere esplicito il fatto che queste possano avere side effect.
Inoltre, è stato evidenziato come il concetto di monade permetta di fornire un meccanismo comune per la modellazione e messa in sequenza di tali side effect.

Tuttavia, fino ad ora è stato tralasciato un side effect fondamentale: l'esecuzione di input e output.
Chiaramente una funzione Scala potrebbe effettuare input e output semplicemente utilizzando le funzioni standard fornite dal linguaggio:
\begin{lstlisting}[language=scala3]
def addTo(x: Int): Int =
	val y = scala.io.StdIn.readInt() // side effect!
	x + y
\end{lstlisting}\label{code:addToScalaImpure}
Tuttavia, dalla sola analisi del tipo della funzione \lstinline{sum : Int => Int} non è possibile capire se questa interagirà con il mondo esterno o meno.

Per comprendere come sia possibile tracciare tale side effect a livello di tipi verrà preso come esempio paradigmatico il linguaggio Haskell; successivamente verrà mostrato come le stesse intuizioni possano essere applicate in Scala.

\subsection{Modello di valutazione \emph{lazy}}
Haskell è un linguaggio funzionale con strategia di valutazione \emph{lazy} (anche detta \emph{call-by-need}): ciò significa che gli argomenti delle funzioni vengono valutati solo se strettamente necessario e non sono valutati prima di essere passati alla funzione. Si consideri per esempio la seguente funzione:
\begin{lstlisting}[language=haskell]
lazy :: Int -> Int -> Int
lazy x y = x * 3
\end{lstlisting}
Quando la funzione viene chiamata, anziché valutare i suoi argomenti prima di eseguire il corpo della funzione, vengono allocati due \emph{thunk} che rappresentano le espressioni da valutare. Sarà poi il corpo della funzione, in base al bisogno, a stabilire di quali \emph{thunk} forzare la valutazione. Nell'esempio specifico valutato solo il \emph{thunk} del primo argomento. Si considerino le possibili chiamate alla funzione \lstinline{lazy}:
\begin{lstlisting}[language=haskell]
lazy (2 + 3) (expensiveFunction 2)
lazy (2 + 3) (1 + undefined)
\end{lstlisting}
Nel primo esempio il \emph{thunk} che rappresenta l'espressione \lstinline{expensiveFunction 2} non verrà mai valutato; allo stesso modo nella seconda chiamata il \emph{thunk} dell'espressione \lstinline{1 + undefined}\footnote{In Haskell il valore \lstinline{undefined} è un valore speciale che comporta il crash dell'applicazione nel momento in cui viene valutato. In questo caso, dato che si trova in un \emph{thunk} che verrà scartato senza essere valutato non verrà sollevata alcuna eccezione.} non sarà valutato; il risultato sarà 15 in entrambi i casi.

Grazie a questa strategia di valutazione è possibile definire direttamente operatori con \emph{short-circuiting} come \lstinline{&&} e \lstinline{||}:
\begin{lstlisting}[language=haskell]
(&&) :: Bool -> Bool -> Bool
x && y = case x of
  True  -> y
  False -> False 

(||) :: Bool -> Bool -> Bool
x || y = case x of
  True  -> True
  False -> y
\end{lstlisting}
Le due funzioni hanno una struttura simile: inizialmente viene forzata la valutazione del primo argomento tramite \emph{pattern matching} sul suo valore. In seguito viene restituito il secondo argomento o un valore predefinito in base al risultato del pattern matching. In entrambi i casi non viene forzata la valutazione del secondo argomento attuando la logica di \emph{short-circuiting} che ci si potrebbe aspettare dagli operatori logici \lstinline{&&} e \lstinline{||}: il risultato di \lstinline{False && undefined} sarà \lstinline{False} e il programma non terminerà con un'eccezione dato che l'espressione \lstinline{undefined} non sarà valutata.

\subsubsection{Incompatibilità di \emph{laziness} e side effect}
Un aspetto negativo della strategia \emph{lazy} è che può diventare estremamente complesso capire l'ordine con il quale le espressioni vengono valutate. Infatti, come mostrato negli esempi precedenti la valutazione dei \emph{thunk} viene forzata solo quando strettamente necessario. Per questo motivo sarebbe pressoché impossibile riuscire a compiere in una sequenza prevedibile i side effect delle funzioni~\cite{cit:tackling-the-awkward-squad}.\nicetohave{Forse si potrebbe aggiungere un esempio più complesso in un listato a parte che mostri la difficoltà di avere side effect in un contesto lazy? Per esempio con lettura e chiusura di file handle}

È proprio questa caratteristica che ha fatto sì che Haskell rimanesse un linguaggio puro e ha portato all'invenzione dell'input e output monadico: ``forse il più grande beneficio della \emph{laziness} non è la \emph{laziness} in sè, quanto il fatto che ci abbia forzato a rimanere puri, motivando così una grande quantità di lavoro sulle monadi''\footnote{Traduzione dal testo originale: ``[...] perhaps the biggest single benefit of laziness is not laziness per se, but rather that laziness kept us pure, and thereby motivated a great deal of productive work on monads [...]''~\cite{cit:a-history-of-haskell-being-lazy-with-class}}.

\subsection{I/O monadico in Haskell}
\label{sub:io-monadico-haskell}
La soluzione adottata da Haskell per permettere di effettuare operazioni di I/O è quello di fornire un nuovo costruttore di tipi chiamato \lstinline{IO} che sia una monade. Un valore di tipo \lstinline{IO a} modella una computazione che produce un valore di tipo \lstinline{a} e può avere il side effect di interagire con il sistema, per esempio effettuando operazioni di input o output.

Il programmatore potrà definire nuove operazioni combinando le funzioni di libreria fornite dal linguaggio sfruttando l'interfaccia delle monadi:
\begin{lstlisting}[language=haskell]
echo :: IO ()
echo = do
  line <- getLine
  putStrLn line
\end{lstlisting}
Partendo dal tipo della funzione si può comprendere come questa rappresenti una computazione che una volta eseguita restituisce un valore di tipo \lstinline{()} e che può effettuare I/O. La funzione è implementata mettendo in sequenza due operazioni più semplici: prima viene letta una riga dallo \emph{standard input} e il contenuto letto viene stampato sullo \emph{standard output} tale e quale.
È interessante osservare come la \emph{do notation} nasconda l'applicazione di \lstinline{>>=} e \lstinline{pure} dando al codice un tipico aspetto imperativo. Nonostante l'apparente ``imperatività'' è fondamentale ricordare che i valori di tipo \lstinline{IO} -- come \lstinline{getLine} e \lstinline{putStrLn} -- non sono funzioni che provocano side effect ma descrivono i side effect che devono avere luogo. L'equivalente versione senza zucchero sintattico è:
\begin{lstlisting}[language=haskell]
echo :: IO ()
echo = getLine >>= putStrLn
\end{lstlisting}

Il punto d'ingresso di ogni programma Haskell è la funzione \lstinline{main}:
\begin{lstlisting}[language=haskell]
main :: IO ()
main = echo
\end{lstlisting}
Quindi un programma non è altro che una struttura dati immutabile che \emph{descrive} la sequenza di operazioni che il \emph{runtime system} del linguaggio deve eseguire a tempo d'esecuzione.

\subsubsection{Separazione di codice puro e impuro}
Grazie all'approccio appena mostrato non è possibile mescolare inavvertitamente codice puro e codice con side effect -- proprio come per i casi di eccezioni e stato mutabile mostrati in precedenza. Per esempio, consideriamo come potrebbe essere riscritta la funzione impura mostrata all'inizio della sezione; semplicemente leggere un valore intero da standard input non è possibile:
\begin{lstlisting}[language=haskell]
readInt :: IO Int
readInt = fmap read getLine

addTo :: Int -> Int
addTo x = let y = readInt in x + y
\end{lstlisting}
Il codice mostrato non compilerebbe in quanto \lstinline{x} ha tipo \lstinline{Int} mentre \lstinline{readInt} è un valore di tipo \lstinline{IO Int}: non è un valore intero bensì una computazione che produrrà un intero. Per far sì che \lstinline{addTo} possa utilizzare una funzione impura come \lstinline{readInt}, è necessario rendere esplicito a livello di tipi il fatto che anche \lstinline{addTo} sia impura:
\begin{lstlisting}[language=haskell]
addTo :: Int -> IO Int
addTo x = do
  y <- getInt
  pure (x + y)
\end{lstlisting}

\subsubsection{Programmi come valori di prima classe}
Come già descritto, un valore di tipo \lstinline{IO a} non è altro che una struttura dati che descrive una sequenza di computazioni per produrre un valore di tipo \lstinline{a}. Ciò permette di passare programmi come valori di prima classe e costruire una ricca serie di funzioni generiche che operano su programmi e producono nuovi programmi in output. Per esempio la funzione
\begin{lstlisting}[language=haskell]
forever :: IO a -> IO b
forever action = action >> forever action
\end{lstlisting}
prende in input un programma e restituisce un programma che lo esegue in loop all'infinito.

Un ulteriore esempio può essere la funzione \lstinline{retry} definita come segue:
\begin{lstlisting}[language=haskell]
retry :: Int -> (a -> Bool) -> IO a -> IO (Maybe a)
retry 0 _ _ = pure Nothing
retry times shouldRetry action = do
  result <- action
  if shouldRetry result
    then retry (times - 1) shouldRetry action
    else pure (Just result)
\end{lstlisting}
Questa restituisce in output un programma che ripete fino a un massimo numero di volte un programma passato in input secondo una certa logica di ripetizione definita dal predicato \lstinline{shouldRetry}.
Queste funzioni possono essere combinate in programmi più complessi\footnote{Nell'esempio per effettuare le richieste a un server viene utilizzata la libreria \emph{http-conduit}~\cite{cit:http-conduit}}:
\begin{lstlisting}[language=haskell]
URLToResource :: String -> IO (Response ByteString)
URLToResource url = httpLBS (fromString url)

shouldRetry :: Response a -> Bool
shouldRetry response =
  let statusCode = getResponseStatusCode response
   in statusCode `elem' [500, 503]

main :: IO ()
main = forever $ do
  let times = 10
  putStrLn "URL of the resource: "
  url <- getLine  
  result <- retry times shouldRetry (URLToResource url)
  case result of
    Nothing -> putStrLn "Failed after 10 retries"
    Just _  -> putStrLn "Got a response"
\end{lstlisting}
Sfruttando la funzione \lstinline{retry} è possibile definire un programma che ripete una richiesta HTTP fino a un massimo di 10 volte in caso di errore 500 o 503\footnote{Ai fini dell'esempio la funzione \lstinline{retry} è piuttosto semplice e non implementa logiche complesse per attendere prima di ripetere una richiesta evitando di oberare il server. Il package \emph{retry}~\cite{cit:retry} implementa una funzione analoga a quella mostrata con la possibilità di specificare delle \emph{policy} per stabilire la strategia con cui riprovare l'azione.}. Utilizzando la funzione \lstinline{forever} è possibile fare in modo che il programma continui a chiedere input al programmatore all'infinito.

Un ulteriore vantaggio dato dal fatto che \lstinline{IO} è una monade sta nella possibilità di sfruttare codice generico sul tipo di monade:

\begin{lstlisting}[language=haskell]
void :: Monad m => m a -> m ()
void = m >>= (\_ -> pure ())

sequence :: Monad m => [m a] -> m [a]
sequence [] = pure []
sequence (m:ms) = do
  a  <- m
  as <- sequence ms
  pure (a:as)

main :: IO ()
main =
  let messages = ["message1", "message2", "message3"]
  in void (sequence (map putStrLn messages))
\end{lstlisting}

In realtà le stesse funzioni \lstinline{forever} e \lstinline{retry} possono essere definite in maniera generica rispetto al tipo di effetto in considerazione:
\begin{lstlisting}[language=haskell]
forever :: Monad m => m a -> m b
retry :: Monad m
  => Int
  -> (a -> Bool)
  -> m a
  -> m (Maybe a)
\end{lstlisting}
L'implementazione è tralasciata in quanto identica a quanto riportato nell'esempio di codice mostrato in precedenza: l'unico cambiamento necessario per rendere la funzione più generica è stato quello di sostituire nel tipo \lstinline{IO} con una generica monade \lstinline{m}.

\subsection{I/O monadico in Scala}
In Haskell la monade \lstinline{IO} deve essere implementata come un tipo di dato opaco con un supporto speciale del compilatore. Haskell infatti è un linguaggio puro con una modalità di valutazione \emph{lazy} che, come mostrato in precedenza, è incompatibile con la presenza di side effect.
In Scala non è necessario un supporto diretto del compilatore in quanto è già possibile definire computazioni che svolgono side effect; perciò una semplice implementazione\footnote{L'implementazione vuole unicamente mostrare come si possa immaginare l'implementazione della monade \lstinline{IO} in un linguaggio come Scala. Tuttavia, un'implementazione simile non è \emph{stack safe:} interpretare un'azione \lstinline{IO} ottenuta componendo molti blocchi di base può portare a un errore di \emph{stack overflow}. Questo problema può essere risolto complicando l'implementazione della monade sfruttando una tecnica nota come \emph{trampolining}~\cite{cit:stackless-scala-with-free-monads} ed è l'approccio adottato da librerie come Cats Effect~\cite{cit:cats-effect-stack-safety}.} della monade \lstinline{IO} potrebbe essere la seguente:
\scalaFromFile{7}{14}{monads/IO.scala}
\begin{itemize}
  \item \lstinline{IO} è un costruttore di tipi che preso in input un valore di tipo \lstinline{A} restituisce un valore di tipo \lstinline{IO[A]}. Un valore con questo tipo rappresenta una computazione che, quando eseguita, può effettuare input o output e restituisce un valore di tipo \lstinline{A}
  \item \lstinline{pure} permette di trasformare un valore di tipo \lstinline{A} in uno di tipo \lstinline{IO[A]}. Non introduce alcun side effect e restituisce semplicemente il valore passato in input
  \item \lstinline{flatMap} permette di concatenare in sequenza due operazioni che possono svolgere I/O. Il risultato sarà una singola computazione che esegue i side effect di ciascuna in sequenza
\end{itemize}

Le funzioni impure della libreria standard di Scala possono quindi essere espresse in termini di \lstinline{IO}:
\scalaFromFile{16}{17}{monads/IO.scala}
\lstinline{putStrLn} e \lstinline{getLine} sono valori di tipo \lstinline{IO}; vale a dire strutture dati immutabili che descrivono come eseguire un side effect nel momento in cui la computazione verrà interpretata. Si noti la differenza fra una computazione \lstinline{IO} e l'equivalente versione impura:
\begin{lstlisting}[language=scala3]
val res1 = putStrLn("Hello, World!")
// res1 : IO[Unit]

val res2 = println("Hello, World!")
// res2 : Unit
// -> "Hello, World!"
\end{lstlisting}
La valutazione di \lstinline{putStrLn} produce un valore di tipo \lstinline{IO[Unit]} senza alcun side effect; d'altro canto, la valutazione di \lstinline{println} produce un valore di tipo \lstinline{Unit} e ha il side effect di stampare in output la stringa.

L'unico modo per poter estrarre un valore dalla monade \lstinline{IO} interpretandone il contenuto è tramite l'uso di \lstinline{unsafeRun}:
\begin{lstlisting}[language=scala3]
val program = for
  _ <- putStrLn("Line 1")
  _ <- putStrLn("Line 2")
yield ()

program.unsafeRun()
// -> Line 1
// -> Line 2
\end{lstlisting}
Il metodo è stato chiamato \lstinline{unsafeRun} a suggerire l'eccezionalità nella sua invocazione. Infatti, l'intero programma dovrebbe essere descritto all'interno della monade \lstinline{IO} per poter poi essere eseguito nel main del programma con un'unica chiamata a \lstinline{unsafeRun}.

I vantaggi ottenuti grazie a questo approccio sono gli stessi già descritti nella \Cref{sub:io-monadico-haskell}: è possibile separare chiaramente il codice impuro dal codice puro, i programmi diventano cittadini di prima classe che possono essere presi come input e restituiti in output e diventa possibile sfruttare tutte le funzioni generiche sul tipo di monade per comporre programmi complessi. Per esempio, come mostrato al \Cref{lst:io-http}, il codice mostrato in precedenza per effettuare delle richieste a un server può essere scritto in Scala in maniera molto simile\footnote{Nell'esempio per effettuare le richieste a un server viene utilizzata la libreria \emph{Requests-Scala}~\cite{cit:requests-scala}}.

\begin{figure}[htp]
  \begin{lstlisting}[language=scala3, caption={Esempio di codice monadico che incapsula i side effect all'interno della monade IO per implementare una politica di \emph{retry} per le richieste HTTP.}, label={lst:io-http}]
    extension[A](m: IO[A])
      def forever[B]: IO[B] = m.flatMap(_ => m.forever)
      def retry(times: Int, shouldRetry: A => Boolean):
        IO[Option[A]] =
        times match
          case 0 => IO.pure(None)
          case n => m.flatMap{ result => 
            if shouldRetry(result)
            then m.retry(n-1, shouldRetry)
            else IO.pure(Some(result))
          }

    def urlToResource(url: String): IO[Try[Response]] =
      IO(() => Try(requests.get(url)))

    def shouldRetry(response: Try[Response]): Boolean =
      response match
        case Failure(exception: RequestFailedException) =>
          val statusCode = exception.response.statusCode
          List(500, 503).contains(statusCode)
        case _ => false

    @main def main: Unit = 
      val times = 10
      val step = for 
        _ <- putStrLn("URL of the resource: ")
        url <- getLine
        result <- URLToResource(url)
                    .retry(times, shouldRetry)
        _ <- result match
          case None => putStrLn("Failed after 10 retries")
          case Some(_) => putStrLn("Got a response")
      yield ()
      step.forever.unsafeRun()
  \end{lstlisting}
\end{figure}

\chapter{Stack di monadi}

Nel capitolo precedente si è mostrato come sia possibile implementare semplici monadi per poter modellare la presenza di diversi side effect.
Tuttavia, seguendo tale approccio non è evidente come sia possibile gestire contemporaneamente più side effect. Infatti ogni monade descritta permette di modellare un singolo side effect per volta: uno stato mutabile per \lstinline{State}, la possibilità di fallimenti con \lstinline{Option} e la capacità di effettuare input e output con \lstinline{IO}.

In questo capitolo verrà introdotto il concetto di \term{monad transformer}: un meccanismo che permette di unire monadi elementari combinandone le caratteristiche.





\appendix
\mustfix{Aggiungere un'appendice in cui si parla da un'introduzione della sintassi haskell (magari come confronto con scala): il minimo è mostrare applicazione di funzione, associatività e precedenza degli operatori, costruttori di tipi e currying dei tipi delle funzioni per poter capire gli esempi}
\chapter{Dimostrazioni delle leggi monadiche}
\label{dimostrazioni-delle-leggi-monadiche}

\section{Dimostrazione per la monade identità}
\label{dimostrazione-per-la-monade-identita}

Alla \Cref{la-monade-identita} è stata data una definizione di monade per \lstinline{Identity}. In seguito è riportata una dimostrazione del rispetto delle leggi monadiche elencate nella \Cref{cos-e-una-monade} per la definizione fornita\footnote{Nei passaggi della dimostrazione, così come in quelle successive, sono utilizzati i nomi canonici utilizzati per l'implementazione Scala; quindi, si utilizza \lstinline{flatMap} anziché \lstinline{>>=} e \lstinline{pure} anziché \lstinline{return}}.

Dimostrazione dell'identità sinistra, ovvero che \lstinline{pure(a).flatMap(f) = f(a)}:

\begin{tabularx}{\textwidth}{ll}
  \lstinline{pure(a).flatMap(f) =} & \emph{Definizione di \lstinline{pure}} \\
  \\
  \lstinline{a.flatMap(f) =} & \emph{Definizione di \lstinline{flatMap}}    \\
  \\
  \lstinline{f(a)}$\qed$ &
\end{tabularx}

Dimostrazione dell'identità destra, ovvero che \lstinline{m.flatMap(pure) = m}:

\begin{tabularx}{\textwidth}{ll}
  \lstinline{m.flatMap(pure) =} & \emph{Definizione di \lstinline{flatMap}} \\
  \\
  \lstinline{pure(m) = m}$\qed$ & \emph{Definizione di \lstinline{pure}}    \\
\end{tabularx}

Dimostrazione dell'associatività, ovvero che \lstinline{(m.flatMap(f)).flatMap(g) = m.flatMap(x => f(x).flatMap(g))}:

\begin{tabularx}{\textwidth}{ll}
  \lstinline{(m.flatMap(f)).flatMap(g) =} & \emph{Definizione di \lstinline{flatMap}} \\
  \\
  \lstinline{f(m).flatMap(g) =} & \emph{Definizione di \lstinline{flatMap}}           \\
  \\
  \lstinline{g(f(m)) =} & \emph{Composizione di funzione}                             \\
  \\
  \lstinline{(x => g(f(x)))(m)} & \emph{Definizione di \lstinline{flatMap}}           \\
  \\
  \lstinline{m.flatMap(x => g(f(x))) =} & \emph{Definizione di \lstinline{flatMap}}   \\
  \\
  \lstinline{m.flatMap(x => f(x).flatMap(g))}$\qed$ &
\end{tabularx}


\section{Dimostrazione per la monade Optional}
\label{dimostrazione-per-la-monade-optional}

Alla \Cref{la-monade-optional} è stata data una definizione di monade per \lstinline{Option}. In seguito è riportata una dimostrazione del rispetto delle leggi monadiche elencate nella \Cref{cos-e-una-monade} per la definizione fornita\footnote{Nei passaggi della dimostrazione, così come in quelle successive, sono utilizzati i nomi canonici utilizzati per l'implementazione Scala; quindi, si utilizza \lstinline{flatMap} anziché \lstinline{>>=} e \lstinline{pure} anziché \lstinline{return}}.

Dimostrazione dell'identità sinistra, ovvero che \lstinline{pure(a).flatMap(f) = f(a)}:

\begin{tabularx}{\textwidth}{ll}
\lstinline{pure(a).flatMap(f) =} & \emph{Definizione di \lstinline{pure}}\\
\\
\lstinline{Some(a).flatMap(f) =} & \emph{Definizione di \lstinline{flatMap}}\\
\\
\lstinline{f(a)}$\qed$ &
\end{tabularx}

Dimostrazione dell'identità destra, ovvero che \lstinline{m.flatMap(pure) = m}:

\begin{tabularx}{\textwidth}{ll}
  Procedo per casi su \lstinline{m}: & \\
  & \\
  \emph{Se \lstinline{m = None}} & \\
  \\
  \lstinline{\ \ m.flatMap(pure) =} & \emph{Per ipotesi \lstinline{m = None}}\\
  \\
  \lstinline{\ \ None.flatMap(pure) =} & \emph{Definizione di \lstinline{flatMap}}\\
  \\
  \lstinline{\ \ None = m} & \emph{Per ipotesi \lstinline{None = m}} \\
  \\
  \emph{Se \lstinline{m = Some(a)}} & \\
  \\
  \lstinline{\ \ m.flatMap(pure) =} & \emph{Per ipotesi \lstinline{m = Some(a)}}\\
  \\
  \lstinline{\ \ Some(a).flatMap(pure) =} & \emph{Definizione di \lstinline{flatMap}}\\
  \\
  \lstinline{\ \ pure(a) =} & \emph{Definizione di \lstinline{pure}}\\
  \\
  \lstinline{\ \ Some(a) = m}$\qed$ & \emph{Per ipotesi \lstinline{Some(a) = m}}
\end{tabularx}

Dimostrazione dell'associatività, ovvero che \lstinline{(m.flatMap(f)).flatMap(g) = m.flatMap(x => f(x).flatMap(g))}:

\begin{tabularx}{\textwidth}{ll}
  Procedo per casi su \lstinline{m}: & \\
  & \\
  \emph{Se \lstinline{m = None}} & \\
  \\
  \lstinline{\ \ (m.flatMap(f)).flatMap(g) =} & \emph{Per ipotesi \lstinline{m = None}}\\
  \\
  \lstinline{\ \ (None.flatMap(f)).flatMap(g) =} & \emph{Definizione di \lstinline{flatMap}}\\
  \\
  \lstinline{\ \ None.flatMap(g) =} & \emph{Definizione di \lstinline{flatMap}}\\
  \\
  \lstinline{\ \ None =} & \emph{Definizione di \lstinline{flatMap}}\\
  \\
  \lstinline{\ \ None.flatMap(x => f(x).flatMap(g)) =} & \emph{Per ipotesi \lstinline{None = m}}\\
  \\
  \lstinline{\ \ m.flatMap(x => f(x).flatMap(g))} &  \\
  \\
  \emph{Se \lstinline{m = Some(a)}} & \\
  \\
  \lstinline{\ \ (m.flatMap(f)).flatMap(g) =} & \emph{Per ipotesi \lstinline{m = Some(a)}}\\
  \\
  \lstinline{\ \ (Some(a).flatMap(f)).flatMap(g) =} & \emph{Definizione di \lstinline{flatMap}}\\
  \\
  \lstinline{\ \ f(a).flatMap(g) =} & \emph{Applicazione di funzione}\\
  \\
  \lstinline{\ \ (x => f(x).flatMap(g))(a) =} & \emph{Definizione di \lstinline{flatMap}} \\
  \\
  \lstinline{\ \ Some(a).flatMap(x => f(x).flatMap(g)) =} & \emph{Per ipotesi \lstinline{Some(a) = m}} \\
  \\
  \lstinline{\ \ m.flatMap(x => f(x).flatMap(g))}$\qed$ & 
\end{tabularx}


\section{Dimostrazione per la monade State}
\label{dimostrazione-per-la-monade-state}

Alla \Cref{la-monade-state} è stata data una definizione di monade per \lstinline{State}. In seguito è riportata una dimostrazione del rispetto delle leggi monadiche per la definizione fornita.

Dimostrazione dell'identità sinistra, ovvero che \lstinline{pure(a).flatMap(f) = f(a)}:

\begin{tabularx}{\textwidth}{ll}
\lstinline{pure(a).flatMap(f) =}               & \emph{Definizione di \lstinline{pure}}\\
\\
\lstinline{State(s => (a, s)).flatMap(f) =}    & \emph{Sia \lstinline{m = State(s => (a, s))}}\\
\\
\lstinline{m.flatMap(f) =}                     & \emph{Definizione di 
\lstinline{flatMap}}\\
\\
\lstinline{State(s0 =>} \\
\lstinline{\ \ val (res1, s1) = m.runState(s0)}\\
\lstinline{\ \ f(res1).runState(s1)) =}      & \emph{Definizione di \lstinline{runState}}\\
\\
\lstinline{State(s0 =>} \\
\lstinline{\ \ val (res1, s1) = (s => (a, s))(s0)}\\
\lstinline{\ \ f(res1).runState(s1)) =}      & \emph{Applicazione di funzione}\\
\\
\lstinline{State(s0 =>} \\
\lstinline{\ \ val (res1, s1) = (a, s0)}\\
\lstinline{\ \ f(res1).runState(s1)) =}      & \emph{Pattern matching su una tupla}\\
\\
\lstinline{State(s0 => f(a).runState(s0)) =} & \emph{Sia \lstinline{f(a) = State(g)}}\\
\\
\lstinline{State(s0 => State(g).runState(s0)) =} & \emph{Definizione di \lstinline{runState}}\\
\\
\lstinline{State(s0 => g(s0)) =}                   & \emph{$\eta$-riduzione}\\
\\
\lstinline{State(g) =}                             & \emph{Per definizione di \lstinline{f(a)}}\\
\\
\lstinline{f(a)}$\qed$ &
\end{tabularx}

Dimostrazione dell'identità destra, ovvero che \lstinline{m.flatMap(pure) = m}:

\begin{tabularx}{\textwidth}{ll}
\lstinline{m.flatMap(pure) =} & \emph{Definizione di \lstinline{flatMap}}\\
\\
\lstinline{State(s0 =>} \\
\lstinline{\ \ val (res1, s1) = m.runState(s0)}\\
\lstinline{\ \ pure(res1).runState(s1)) =} & \emph{Definizione di \lstinline{pure}}\\
\\
\lstinline{State(s0 =>} \\
\lstinline{\ \ val (res1, s1) = m.runState(s0)}\\
\lstinline{\ \ State(s => (res1, s))} \\
\lstinline{\ \ \ \ .runState(s1)) =}& \emph{Definizione di \lstinline{runState}}\\
\\
\lstinline{State(s0 =>} \\
\lstinline{\ \ val (res1, s1) = m.runState(s0)}\\
\lstinline{\ \ (s => (res1, s))(s1)) =} & \emph{Applicazione di funzione}\\
\\
\lstinline{State(s0 =>} \\
\lstinline{\ \ val (res1, s1) = m.runState(s0)}\\
\lstinline{\ \ (res1, s1)) =} & \emph{Eliminazione pattern matching}\\
\\
\lstinline{State(s0 => m.runState(s0)) =} & \emph{Sia \lstinline{m = State(g)}} \\
\\
\lstinline{State(s0 => State(g).runState(s0)) =} & \emph{Definizione di \lstinline{runState}} \\
\\
\lstinline{State(s0 => g(s0)) =} & \emph{$\eta$-riduzione} \\
\\
\lstinline{State(g) =} & \emph{Per definizione di \lstinline{m}} \\
\\
\lstinline{m} $\qed$ 
\end{tabularx}

Dimostrazione dell'associatività, ovvero che \lstinline{(m.flatMap(f)).flatMap(g) = m.flatMap(x => f(x).flatMap(g))}:

\begin{tabularx}{\textwidth}{ll}
\lstinline{(m.flatMap(f)).flatMap(g) =} & \emph{Definizione di \lstinline{flatMap}}\\
\\
\lstinline{State(s0 =>} \\
\lstinline{\ \ val (res2, s2) =}\\
\lstinline{\ \ \ \ (m.flatMap(f)).runState(s0)} \\
\lstinline{\ \ g(res2).runState(s2)) =} & \emph{Definizione di \lstinline{flatMap}}\\
\\
\lstinline{State(s0 =>} \\
\lstinline{\ \ val (res2, s2) = (State(s =>} \\
\lstinline{\ \ \ \ val (res1, s1) = m.runState(s)}\\
\lstinline{\ \ \ \ f(res1).runState(s1)}\\
\lstinline{\ \ \ \ .runState(s0)} \\
\lstinline{\ \ g(res2).runState(s2)) =} & \emph{Definizione di \lstinline{runState}}\\
\\
\lstinline{State(s0 =>} \\
\lstinline{\ \ val (res2, s2) =} \\
\lstinline{\ \ \ \ val (res1, s1) = m.runState(s0)} \\
\lstinline{\ \ \ \ f(res1).runState(s1)))} \\
\lstinline{\ \ g(res2).runState(s2)) =} & \emph{Estrazione dichiarazioni locali} \\
\\
\lstinline{State(s0 =>} \\
\lstinline{\ \ val (res1, s1) = m.runState(s0)} \\
\lstinline{\ \ val (res2, s2) = f(res1).runState(s1)} \\
\lstinline{\ \ g(res2).runState(s2)) =} & \emph{Definizione di \lstinline{State}} \\
\\
\lstinline{State(s0 =>} \\
\lstinline{\ \ val (res1, s1) = m.runState(s0)} \\
\lstinline{\ \ State(s =>} \\
\lstinline{\ \ \ \ val (res2, s2) = f(res1).runState(s)} \\
\lstinline{\ \ \ \ g(res2).runState(s2))} \\
\lstinline{\ \ \ \ .runState(s1)) =} & \emph{Definizione di \lstinline{flatMap}} \\
\\
\lstinline{State(s0 =>} \\
\lstinline{\ \ val (res1, s1) = m.runState(s0)} \\
\lstinline{\ \ (f(res1).flatMap(g))} \\
\lstinline{\ \ \ \ .runState(s1)) =} & \emph{Applicazione di funzione} \\
\\
\lstinline{State(s0 =>} \\
\lstinline{\ \ val (res1, s1) = m.runState(s0)} \\
\lstinline{\ \ (x => f(x).flatMap(g))(res1)} \\
\lstinline{\ \ \ \ .runState(s1)) =} & \emph{Definizione di \lstinline{flatMap}} \\
\\
\lstinline{m.flatMap(x => f(x).flatMap(g))} $\qed$\\
\end{tabularx}



%----------------------------------------------------------------------------------------
% BIBLIOGRAPHY
%----------------------------------------------------------------------------------------
\nocite{*} % Show all elements in bibliography
\printbibliography[nottype=online, title={Riferimenti Bibliografici}]
\printbibliography[type=online, title={Riferimenti Sitografici}]

\end{document}