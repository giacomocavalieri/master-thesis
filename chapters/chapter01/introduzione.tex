La ragion d'essere di un qualunque programma è quella di produrre un qualche effetto tangibile sul mondo esterno: che sia scrivere dei dati su disco, inviare un messaggio sulla rete, stampare dei caratteri a schermo ecc.
Linguaggi dalla natura imperativa come C, Java o Scala forniscono delle funzioni apposite per ottenere tali effetti; in Scala, per esempio, si potrebbe implementare la seguente funzione per apporre una stringa al contenuto di un file su disco\footnote{In questa funzione così come in tutte le altre funzioni Scala a seguire viene adottata la sintassi introdotta da Scala 3~\cite{cit:new-in-scala-3}.}:

\begin{lstlisting}[language=scala3]
def appendToFile(file: File, line: String): Unit =
  println(f"Appending $line to $file")
  val writer = FileWriter(file, true)
  try writer.write(f"$line\n")
  finally writer.close
\end{lstlisting}
Lo scopo di funzioni come \lstinline{appendToFile}, \lstinline{write} e \lstinline{println} non è quello di produrre un risultato -- il loro valore di ritorno è sempre \lstinline{Unit} -- ma di mettere in atto dei \emph{side effect}.

Più in generale, con side effect si intende una qualunque interazione della funzione con un ambiente diverso da quello locale.
Possono essere quindi considerati side effect il modificare una variabile globale o un parametro passato per riferimento, lanciare un'eccezione, l'effettuare operazioni di input e output -- come mostrato precedentemente in \lstinline{appendToFile} -- o il chiamare una funzione che a sua volta presenta dei side effect~\cite{cit:on-the-prevalence-of-function-side-effects-in-general-purpose-open-source-software-systems}.

Le funzioni che presentano side effect sono spesso anche dette \emph{impure} mentre funzioni che non hanno side effect sono anche dette \emph{pure} o caratterizzate da \emph{trasparenza referenziale}.