Date le criticità evidenziate nella sezione precedente, diverse pratiche di buona programmazione suggeriscono di ridurre al minimo le funzioni che presentano side effect~\cite[p.~44]{cit:clean-code-a-handbook-of-agile-software-craftsmanship} e, quando inevitabili, di renderli espliciti nel nome della funzione~\cite[p.~313]{cit:clean-code-a-handbook-of-agile-software-craftsmanship}.

Tuttavia, si possono individuare altre tecniche più sofisticate per tracciare i side effect delle funzioni.
Queste nascono nel contesto dei linguaggi funzionali puri, perciò è necessario comprendere come questa classe di linguaggi possa conciliare la presenza di side effect con la loro natura puramente funzionale.
