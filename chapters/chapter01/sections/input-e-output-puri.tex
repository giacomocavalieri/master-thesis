\section{Input e output puri}
\label{section:input-e-output-puri}

Nelle precedenti sezioni è stato mostrato come sia possibile ``simulare'' la presenza di side effect -- come eccezioni e modifica di uno stato mutabile -- con opportune modifiche al tipo delle funzioni in modo da rendere esplicito il fatto che queste possano avere side effect.
Inoltre, è stato evidenziato come il concetto di monade permetta di fornire un meccanismo comune per la modellazione e messa in sequenza di tali side effect.

Tuttavia, fino ad ora è stato tralasciato un side effect fondamentale: l'esecuzione di input e output.
Chiaramente una funzione Scala potrebbe effettuare input e output semplicemente utilizzando le funzioni standard fornite dal linguaggio:
\begin{lstlisting}[language=scala3]
def addTo(x: Int): Int =
	val y = scala.io.StdIn.readInt() // side effect!
	x + y
\end{lstlisting}\label{code:addToScalaImpure}
Tuttavia, dalla sola analisi del tipo della funzione \lstinline{sum : Int => Int} non è possibile capire se questa interagirà con il mondo esterno o meno.

Per comprendere come sia possibile tracciare tale side effect a livello di tipi verrà preso come esempio paradigmatico il linguaggio Haskell; successivamente verrà mostrato come le stesse intuizioni possano essere applicate in Scala.

\subsection{Modello di valutazione \emph{lazy}}
Haskell è un linguaggio funzionale con strategia di valutazione \emph{lazy} (anche detta \emph{call-by-need}): ciò significa che gli argomenti delle funzioni vengono valutati solo se strettamente necessario e non sono valutati prima di essere passati alla funzione. Si consideri per esempio la seguente funzione:
\begin{lstlisting}[language=haskell]
lazy :: Int -> Int -> Int
lazy x y = x * 3
\end{lstlisting}
Quando la funzione viene chiamata, anziché valutare i suoi argomenti prima di eseguire il corpo della funzione, vengono allocati due \emph{thunk} che rappresentano le espressioni da valutare. Sarà poi il corpo della funzione, in base al bisogno, a stabilire di quali \emph{thunk} forzare la valutazione. Nell'esempio specifico valutato solo il \emph{thunk} del primo argomento. Si considerino le possibili chiamate alla funzione \lstinline{lazy}:
\begin{lstlisting}[language=haskell]
lazy (2 + 3) (expensiveFunction 2)
lazy (2 + 3) (1 + undefined)
\end{lstlisting}
Nel primo esempio il \emph{thunk} che rappresenta l'espressione \lstinline{expensiveFunction 2} non verrà mai valutato; allo stesso modo nella seconda chiamata il \emph{thunk} dell'espressione \lstinline{1 + undefined}\footnote{In Haskell il valore \lstinline{undefined} è un valore speciale che comporta il crash dell'applicazione nel momento in cui viene valutato. In questo caso, dato che si trova in un \emph{thunk} che verrà scartato senza essere valutato non verrà sollevata alcuna eccezione.} non sarà valutato; il risultato sarà 15 in entrambi i casi.

Grazie a questa strategia di valutazione è possibile definire direttamente operatori con \emph{short-circuiting} come \lstinline{&&} e \lstinline{||}:
\begin{lstlisting}[language=haskell]
(&&) :: Bool -> Bool -> Bool
x && y = case x of
  True  -> y
  False -> False 

(||) :: Bool -> Bool -> Bool
x || y = case x of
  True  -> True
  False -> y
\end{lstlisting}
Le due funzioni hanno una struttura simile: inizialmente viene forzata la valutazione del primo argomento tramite \emph{pattern matching} sul suo valore. In seguito viene restituito il secondo argomento o un valore predefinito in base al risultato del pattern matching. In entrambi i casi non viene forzata la valutazione del secondo argomento attuando la logica di \emph{short-circuiting} che ci si potrebbe aspettare dagli operatori logici \lstinline{&&} e \lstinline{||}: il risultato di \lstinline{False && undefined} sarà \lstinline{False} e il programma non terminerà con un'eccezione dato che l'espressione \lstinline{undefined} non sarà valutata.

\subsubsection{Incompatibilità di \emph{laziness} e side effect}
Un aspetto negativo della strategia \emph{lazy} è che può diventare estremamente complesso capire l'ordine con il quale le espressioni vengono valutate. Infatti, come mostrato negli esempi precedenti la valutazione dei \emph{thunk} viene forzata solo quando strettamente necessario. Per questo motivo sarebbe pressoché impossibile riuscire a compiere in una sequenza prevedibile i side effect delle funzioni~\cite{cit:tackling-the-awkward-squad}.

È proprio questa caratteristica che ha fatto sì che Haskell rimanesse un linguaggio puro e ha portato all'invenzione dell'input e output monadico: ``forse il più grande beneficio della \emph{laziness} non è la \emph{laziness} in sè, quanto il fatto che ci abbia forzato a rimanere puri, motivando così una grande quantità di lavoro sulle monadi''\footnote{Traduzione dal testo originale: ``[...] perhaps the biggest single benefit of laziness is not laziness per se, but rather that laziness kept us pure, and thereby motivated a great deal of productive work on monads [...]''~\cite{cit:a-history-of-haskell-being-lazy-with-class}}.

\subsection{I/O monadico in Haskell}
\label{sub:io-monadico-haskell}
La soluzione adottata da Haskell per permettere di effettuare operazioni di I/O è quello di fornire un nuovo costruttore di tipi chiamato \lstinline{IO} che sia una monade. Un valore di tipo \lstinline{IO a} modella una computazione che produce un valore di tipo \lstinline{a} e può avere il side effect di interagire con il sistema, per esempio effettuando operazioni di input o output.

Il programmatore potrà definire nuove operazioni combinando le funzioni di libreria fornite dal linguaggio sfruttando l'interfaccia delle monadi:
\begin{lstlisting}[language=haskell]
echo :: IO ()
echo = do
  line <- getLine
  putStrLn line
\end{lstlisting}
Partendo dal tipo della funzione si può comprendere come questa rappresenti una computazione che una volta eseguita restituisce un valore di tipo \lstinline{()} e che può effettuare I/O. La funzione è implementata mettendo in sequenza due operazioni più semplici: prima viene letta una riga dallo \emph{standard input} e il contenuto letto viene stampato sullo \emph{standard output} tale e quale.
È interessante osservare come la \emph{do notation} nasconda l'applicazione di \lstinline{>>=} e \lstinline{pure} dando al codice un tipico aspetto imperativo. Nonostante l'apparente ``imperatività'' è fondamentale ricordare che i valori di tipo \lstinline{IO} -- come \lstinline{getLine} e \lstinline{putStrLn} -- non sono funzioni che provocano side effect ma descrivono i side effect che devono avere luogo. L'equivalente versione senza zucchero sintattico è:
\begin{lstlisting}[language=haskell]
echo :: IO ()
echo = getLine >>= putStrLn
\end{lstlisting}

Il punto d'ingresso di ogni programma Haskell è la funzione \lstinline{main}:
\begin{lstlisting}[language=haskell]
main :: IO ()
main = echo
\end{lstlisting}
Quindi un programma non è altro che una struttura dati immutabile che \emph{descrive} la sequenza di operazioni che il \emph{runtime system} del linguaggio deve eseguire a tempo d'esecuzione.

\subsubsection{Separazione di codice puro e impuro}
Grazie all'approccio appena mostrato non è possibile mescolare inavvertitamente codice puro e codice con side effect -- proprio come per i casi di eccezioni e stato mutabile mostrati in precedenza. Per esempio, consideriamo come potrebbe essere riscritta la funzione impura mostrata all'inizio della sezione; semplicemente leggere un valore intero da standard input non è possibile:
\begin{lstlisting}[language=haskell]
readInt :: IO Int
readInt = fmap read getLine

addTo :: Int -> Int
addTo x = let y = readInt in x + y
\end{lstlisting}
Il codice mostrato non compilerebbe in quanto \lstinline{x} ha tipo \lstinline{Int} mentre \lstinline{readInt} è un valore di tipo \lstinline{IO Int}: non è un valore intero bensì una computazione che produrrà un intero. Per far sì che \lstinline{addTo} possa utilizzare una funzione impura come \lstinline{readInt}, è necessario rendere esplicito a livello di tipi il fatto che anche \lstinline{addTo} sia impura:
\begin{lstlisting}[language=haskell]
addTo :: Int -> IO Int
addTo x = do
  y <- getInt
  pure (x + y)
\end{lstlisting}

\subsubsection{Programmi come valori di prima classe}
Come già descritto, un valore di tipo \lstinline{IO a} non è altro che una struttura dati che descrive una sequenza di computazioni per produrre un valore di tipo \lstinline{a}. Ciò permette di passare programmi come valori di prima classe e costruire una ricca serie di funzioni generiche che operano su programmi e producono nuovi programmi in output. Per esempio la funzione
\begin{lstlisting}[language=haskell]
forever :: IO a -> IO b
forever action = action >> forever action
\end{lstlisting}
prende in input un programma e restituisce un programma che lo esegue in loop all'infinito.

Un ulteriore esempio può essere la funzione \lstinline{retry} definita come segue:
\begin{lstlisting}[language=haskell]
retry :: Int -> (a -> Bool) -> IO a -> IO (Maybe a)
retry 0 _ _ = pure Nothing
retry times shouldRetry action = do
  result <- action
  if shouldRetry result
    then retry (times - 1) shouldRetry action
    else pure (Just result)
\end{lstlisting}
Questa restituisce in output un programma che ripete fino a un massimo numero di volte un programma passato in input secondo una certa logica di ripetizione definita dal predicato \lstinline{shouldRetry}.
Queste funzioni possono essere combinate in programmi più complessi\footnote{Nell'esempio per effettuare le richieste a un server viene utilizzata la libreria \emph{http-conduit}~\cite{cit:http-conduit}}:
\begin{lstlisting}[language=haskell]
URLToResource :: String -> IO (Response ByteString)
URLToResource url = httpLBS (fromString url)

shouldRetry :: Response a -> Bool
shouldRetry response =
  let statusCode = getResponseStatusCode response
   in statusCode `elem' [500, 503]

main :: IO ()
main = forever $ do
  let times = 10
  putStrLn "URL of the resource: "
  url <- getLine  
  result <- retry times shouldRetry (URLToResource url)
  case result of
    Nothing -> putStrLn "Failed after 10 retries"
    Just _  -> putStrLn "Got a response"
\end{lstlisting}
Sfruttando la funzione \lstinline{retry} è possibile definire un programma che ripete una richiesta HTTP fino a un massimo di 10 volte in caso di errore 500 o 503\footnote{Ai fini dell'esempio la funzione \lstinline{retry} è piuttosto semplice e non implementa logiche complesse per attendere prima di ripetere una richiesta evitando di oberare il server. Il package \emph{retry}~\cite{cit:retry} implementa una funzione analoga a quella mostrata con la possibilità di specificare delle \emph{policy} per stabilire la strategia con cui riprovare l'azione.}. Utilizzando la funzione \lstinline{forever} è possibile fare in modo che il programma continui a chiedere input al programmatore all'infinito.

Un ulteriore vantaggio dato dal fatto che \lstinline{IO} è una monade sta nella possibilità di sfruttare codice generico sul tipo di monade:

\begin{lstlisting}[language=haskell]
void :: Monad m => m a -> m ()
void = m >>= (\_ -> pure ())

sequence :: Monad m => [m a] -> m [a]
sequence [] = pure []
sequence (m:ms) = do
  a  <- m
  as <- sequence ms
  pure (a:as)

main :: IO ()
main =
  let messages = ["message1", "message2", "message3"]
  in void (sequence (map putStrLn messages))
\end{lstlisting}

In realtà le stesse funzioni \lstinline{forever} e \lstinline{retry} possono essere definite in maniera generica rispetto al tipo di effetto in considerazione:
\begin{lstlisting}[language=haskell]
forever :: Monad m => m a -> m b
retry :: Monad m
  => Int
  -> (a -> Bool)
  -> m a
  -> m (Maybe a)
\end{lstlisting}
L'implementazione è tralasciata in quanto identica a quanto riportato nell'esempio di codice mostrato in precedenza: l'unico cambiamento necessario per rendere la funzione più generica è stato quello di sostituire nel tipo \lstinline{IO} con una generica monade \lstinline{m}.

\subsection{I/O monadico in Scala}
In Haskell la monade \lstinline{IO} deve essere implementata come un tipo di dato opaco con un supporto speciale del compilatore. Haskell infatti è un linguaggio puro con una modalità di valutazione \emph{lazy} che, come mostrato in precedenza, è incompatibile con la presenza di side effect.
In Scala non è necessario un supporto diretto del compilatore in quanto è già possibile definire computazioni che svolgono side effect; perciò una semplice implementazione\footnote{L'implementazione vuole unicamente mostrare come si possa immaginare l'implementazione della monade \lstinline{IO} in un linguaggio come Scala. Tuttavia, un'implementazione simile non è \emph{stack safe:} interpretare un'azione \lstinline{IO} ottenuta componendo molti blocchi di base può portare a un errore di \emph{stack overflow}. Questo problema può essere risolto complicando l'implementazione della monade sfruttando una tecnica nota come \emph{trampolining}~\cite{cit:stackless-scala-with-free-monads} ed è l'approccio adottato da librerie come Cats Effect~\cite{cit:cats-effect-stack-safety}.} della monade \lstinline{IO} potrebbe essere la seguente:
\scalaFromFile{7}{14}{monads/IO.scala}
\begin{itemize}
  \item \lstinline{IO} è un costruttore di tipi che preso in input un valore di tipo \lstinline{A} restituisce un valore di tipo \lstinline{IO[A]}. Un valore con questo tipo rappresenta una computazione che, quando eseguita, può effettuare input o output e restituisce un valore di tipo \lstinline{A}
  \item \lstinline{pure} permette di trasformare un valore di tipo \lstinline{A} in uno di tipo \lstinline{IO[A]}. Non introduce alcun side effect e restituisce semplicemente il valore passato in input
  \item \lstinline{flatMap} permette di concatenare in sequenza due operazioni che possono svolgere I/O. Il risultato sarà una singola computazione che esegue i side effect di ciascuna in sequenza
\end{itemize}

Le funzioni impure della libreria standard di Scala possono quindi essere espresse in termini di \lstinline{IO}:
\scalaFromFile{16}{17}{monads/IO.scala}
\lstinline{putStrLn} e \lstinline{getLine} sono valori di tipo \lstinline{IO}; vale a dire strutture dati immutabili che descrivono come eseguire un side effect nel momento in cui la computazione verrà interpretata. Si noti la differenza fra una computazione \lstinline{IO} e l'equivalente versione impura:
\begin{lstlisting}[language=scala3]
val res1 = putStrLn("Hello, World!")
// res1 : IO[Unit]

val res2 = println("Hello, World!")
// res2 : Unit
// -> "Hello, World!"
\end{lstlisting}
La valutazione di \lstinline{putStrLn} produce un valore di tipo \lstinline{IO[Unit]} senza alcun side effect; d'altro canto, la valutazione di \lstinline{println} produce un valore di tipo \lstinline{Unit} e ha il side effect di stampare in output la stringa.

L'unico modo per poter estrarre un valore dalla monade \lstinline{IO} interpretandone il contenuto è tramite l'uso di \lstinline{unsafeRun}:
\begin{lstlisting}[language=scala3]
val program = for
  _ <- putStrLn("Line 1")
  _ <- putStrLn("Line 2")
yield ()

program.unsafeRun()
// -> Line 1
// -> Line 2
\end{lstlisting}
Il metodo è stato chiamato \lstinline{unsafeRun} a suggerire l'eccezionalità nella sua invocazione. Infatti, l'intero programma dovrebbe essere descritto all'interno della monade \lstinline{IO} per poter poi essere eseguito nel main del programma con un'unica chiamata a \lstinline{unsafeRun}.

I vantaggi ottenuti grazie a questo approccio sono gli stessi già descritti nella \Cref{sub:io-monadico-haskell}: è possibile separare chiaramente il codice impuro dal codice puro, i programmi diventano cittadini di prima classe che possono essere presi come input e restituiti in output e diventa possibile sfruttare tutte le funzioni generiche sul tipo di monade per comporre programmi complessi. Per esempio, come mostrato al \Cref{lst:io-http}, il codice mostrato in precedenza per effettuare delle richieste a un server può essere scritto in Scala in maniera molto simile\footnote{Nell'esempio per effettuare le richieste a un server viene utilizzata la libreria \emph{Requests-Scala}~\cite{cit:requests-scala}}.

\begin{figure}[htp]
  \begin{lstlisting}[language=scala3, caption={Esempio di codice monadico che incapsula i side effect all'interno della monade IO per implementare una politica di \emph{retry} per le richieste HTTP.}, label={lst:io-http}]
    extension[A](m: IO[A])
      def forever[B]: IO[B] = m.flatMap(_ => m.forever)
      def retry(times: Int, shouldRetry: A => Boolean):
        IO[Option[A]] =
        times match
          case 0 => IO.pure(None)
          case n => m.flatMap{ result => 
            if shouldRetry(result)
            then m.retry(n-1, shouldRetry)
            else IO.pure(Some(result))
          }

    def urlToResource(url: String): IO[Try[Response]] =
      IO(() => Try(requests.get(url)))

    def shouldRetry(response: Try[Response]): Boolean =
      response match
        case Failure(exception: RequestFailedException) =>
          val statusCode = exception.response.statusCode
          List(500, 503).contains(statusCode)
        case _ => false

    @main def main: Unit = 
      val times = 10
      val step = for 
        _ <- putStrLn("URL of the resource: ")
        url <- getLine
        result <- URLToResource(url)
                    .retry(times, shouldRetry)
        _ <- result match
          case None => putStrLn("Failed after 10 retries")
          case Some(_) => putStrLn("Got a response")
      yield ()
      step.forever.unsafeRun()
  \end{lstlisting}
\end{figure}
