\section{Modellazione esplicita dei side effect}
\label{modellazione-esplicita-dei-side-effect}

Date le criticità evidenziate nella sezione precedente, diverse pratiche di buona programmazione suggeriscono di ridurre al minimo le funzioni che presentano side effect~\cite[p.~44]{cit:clean-code-a-handbook-of-agile-software-craftsmanship} e, quando inevitabili, di renderli espliciti nel nome della funzione~\cite[p.~313]{cit:clean-code-a-handbook-of-agile-software-craftsmanship}.

Tuttavia, si possono individuare altre tecniche più sofisticate per tracciare i side effect delle funzioni.
Queste nascono nel contesto dei linguaggi funzionali puri, perciò è necessario comprendere come questa classe di linguaggi possa conciliare la presenza di side effect con la loro natura puramente funzionale.

\subsection{Linguaggi funzionali puri e side effect}
\label{linguaggi-funzionali-puri-e-side-effect}
Un linguaggio funzionale si dice \emph{puro} se le sue funzioni sono referenzialmente trasparenti.
Questa totale assenza di side effect sembra in netto contrasto con la possibilità di scrivere codice di una qualche utilità: come può un programma puro interagire con il mondo esterno, leggere il valore di uno stato globale o lanciare eccezioni?

La soluzione consiste nella possibilità di ``simulare'' la presenza di side effect tramite opportune modifiche ai tipi delle funzioni mantenendole pure. 
In seguito sono riportati alcuni esempi di questo approccio in Scala\footnote{Sebbene Scala sia un linguaggio impuro può essere utilizzato come se fosse un linguaggio puro evitando di ricorrere ai meccanismi che fornisce per produrre side effects.}. 

\subsubsection{Eccezioni}
\label{eccezioni}
Le eccezioni sono un meccanismo di controllo del flusso che permette di interrompere l'esecuzione di una funzione e di ritornare il controllo al chiamante. 
Tuttavia, una funzione che ne fa uso non avrà trasparenza referenziale:
\begin{lstlisting}[language=scala3]
def h(n: Int): Int =
  n match
    case 0 => throw Exception("n = 0")
    case _ => n * 2
\end{lstlisting}
\lstinline|h| non può essere assimilata a una funzione $h : \mathbb{Z} \rightarrow \mathbb{Z}$ in termini matematici; infatti, per certi input -- in questo caso \lstinline|0| -- la funzione non restituisce un valore appartenente al proprio codominio ma lancia un'eccezione.

Per rendere esplicito il fatto che la funzione possa terminare in maniera anomala per determinati input si può estenderne il codominio: ritornando all'analogia matematica si può pensare ad \lstinline{h} come a una funzione $h : \mathbb{Z} \rightarrow \mathbb{Z} \cup \{ \texttt{Error} \}$.
L'equivalente implementazione Scala sarebbe:
\begin{lstlisting}[language=scala3]
enum Result:
  case Ok(value: Int)
  case Error

def betterH(n: Int): Result =
  n match
    case 0 => Error
    case _ => Ok(n * 2)
\end{lstlisting}
In questo caso è evidente dal tipo di ritorno di \lstinline|betterH| che questa può fallire restituendo un valore di tipo \lstinline{Result.Error}. 
La funzione non nasconde questo comportamento tramite il meccanismo delle eccezioni, lo rende invece evidente nel proprio tipo \lstinline{Int => Result}. 
Il grande vantaggio di rendere esplicita la possibilità di fallimento nel tipo di ritorno sta nel fatto che il programmatore dovrà obbligatoriamente gestire anche il caso in cui la computazione fallisca o il compilatore solleverà un errore a tempo di compilazione:
\begin{lstlisting}[language=scala3]
def useBetterH(n: Int): Int =
  // betterH(n) + betterH(n + 1) darebbe un errore
  // a tempo di compilazione dato che betterH(x)
  // ha come tipo Result e non Int
  betterH(n) match
    case Error      => 0
    case Ok(value1) => betterH(n + 1) match
      case Error      => 0
      case Ok(value2) => value1 + value2
\end{lstlisting}

\subsubsection{Lettura e modifica di uno stato globale}
\label{lettura-e-modifica-di-uno-stato-globale}

Dal momento in cui una funzione legge una variabile globale perde la propria trasparenza referenziale: il suo risultato non dipende più dai soli valori passati in input ma da un nuovo input \emph{nascosto}, lo stato a cui accede. 
La soluzione per rendere esplicita questa dipendenza è passare come parametri di tutti gli elementi necessari al funzionamento della funzione, senza affidarsi alla definizione in uno scope esterno di variabili globali a cui accedere implicitamente.

Questa semplice trasformazione permette di rimuovere il side effect che consiste nella lettura di uno stato globale. 
Sfortunatamente, non è sufficiente per modellare anche la \emph{modifica} di una variabile globale. In questo caso può essere utile tornare all'analogia con le funzioni matematiche.
Prendiamo come esempio la funzione \lstinline{f} definita in precedenza:
\begin{lstlisting}[language=scala3]
var counter = 0
def f(x: Int): Int =
  counter = counter + 1 
  x + counter
\end{lstlisting}
Questa prende in input un numero, ha il side effect di incrementare un contatore globale e poi ne legge il valore per aggiungerlo all'input. Sebbene il tipo di \lstinline{f} sia \lstinline{Int => Int}, a causa dei suoi side effect questa funzione non può essere modellata come una funzione matematica $f: \mathbb{Z} \rightarrow \mathbb{Z}$: a parità di input non darà sempre lo stesso output.

Dato che la funzione necessita di leggere una variabile definita esternamente, questa dovrà essergli passata in come input esplicitamente:
\begin{lstlisting}[language=scala3]
def wrongF(x: Int, counter: Int): Int =
  // Come modificare lo stato di current?
  val newCounterState = counter + 1
  x + newCounterState
\end{lstlisting}
Rimane il problema di come poter modellare la modifica dello stato globale in modo che tale effetto si ripercuota anche su chiamate successive. Infatti, il side effect di aumentare il valore del contatore fa parte della logica applicativa di \lstinline{f} e rimuoverlo comporterebbe un cambiamento nella sua semantica.

La soluzione consiste nel trasformare la funzione in modo tale che restituisca il valore del nuovo stato così che possa essere utilizzato per chiamate successive:
\begin{lstlisting}[language=scala3]
def betterF(x: Int, counter: Int): (Int, Int) =
  val newCounter = counter + 1
  val result = x + newCounter
  (result, newCounter)
\end{lstlisting}
Quindi lo stato dovrà essere passato in maniera esplicita da una chiamata a funzione alla successiva:
\begin{lstlisting}[language=scala3, label=lst:use-better-f]
def useBetterF: (Int, Int) =
  val startingCounter = 1
  val (result1, counter1) = betterF(1, startingCounter)
  // Il nuovo stato counter1 viene passato alla seconda
  // chiamata di betterF
  val (result2, counter2) = betterF(1, counter1)
  // Il nuovo stato counter2 viene passato alla terza
  // chiamata di betterF
  val (result3, finalCounter) = betterF(1, counter2)
  (result1 + result2 + result3, finalCounter)
\end{lstlisting}
La funzione non solo prende in input lo stato a cui deve accedere ma restituisce in output la nuova versione dello stato modificato\footnote{In un linguaggio con strutture dati mutabili come Scala la funzione potrebbe anche non restituire la nuova versione dello stato ma modificare lo stato ricevuto come argomento per riferimento. Tuttavia, come descritto in precedenza anche la modifica degli argomenti è un side effect; questo approccio non risolverebbe il problema del poter tracciare esplicitamente gli effetti di una funzione.}, a questo punto è nuovamente possibile modellarla come una funzione matematica $f : \mathbb{Z}\times\mathbb{Z} \rightarrow \mathbb{Z}\times\mathbb{Z}$ che, per ogni coppia di valori ricevuti in input, darà sempre lo stesso risultato.

In questo modo diventa più facile testare il comportamento della funzione:
\begin{lstlisting}[language=scala3]
def testBetterF(): Unit =
  val counter = 0
  val (result, newCounter) = betterF(1, counter)
  result shouldBe 2
  newCounter shouldBe 1
\end{lstlisting}
Il test è completamente autonomo, non sono più necessarie operazioni preliminari di preparazione dello stato: l'output di \lstinline{betterF}, infatti, dipende unicamente dai valori passati in input e non da una qualche variabile globale che potrebbe essere utilizzata e modificata anche da altri test.

\subsection{Svantaggi della modellazione esplicita}
\label{svantaggi-della-modellazione-esplicita}
Con le trasformazioni elencate in precedenza è possibile mantenere la trasparenza referenziale delle funzioni modellandone i side effect in maniera esplicita. Questo ha il vantaggio di aiutare il programmatore nella composizione delle funzioni rendendo chiaro, a tempo di compilazione, quale potrebbe essere il loro comportamento.

Nonostante ciò, come forse si è già potuto intuire da alcuni degli esempi riportati, questo approccio rende necessario lo scrivere codice spesso molto più verboso. 
In seguito sono riportati due esempi di questo problema e delle soluzioni ad hoc che possono essere adottate per porvi rimedio.

\subsubsection{Gestione del fallimento di una funzione}
Consideriamo la seguente funzione:
\begin{lstlisting}[language=scala3]
def halve(n: Int): Option[Int] =
  n % 2 match
    case 0 => Some(n / 2)
    case _ => None 
\end{lstlisting}
Nel caso in cui il numero passato come input sia pari ne restituirà la metà, altrimenti fallirà (il side effect del fallimento è reso esplicito tramite l'uso del tipo \lstinline{Option}). Per realizzare una funzione analoga che, se l'input è divisibile per 8 ne restituisca il risultato della divisione, un programmatore potrebbe comporre insieme   più chiamate a \lstinline{halve}:
\begin{lstlisting}[language=scala3]
def eighth(n: Int): Option[Int] =
  halve(n) match
    case None       => None
    case Some(half) => halve(half) match
      case None         => None
      case Some(fourth) => halve(fourth)
\end{lstlisting}
Nonostante la semplicità della funzione, si può osservare come il dover gestire in maniera esplicita il fallimento di ogni chiamata ad \lstinline{halve} renda il codice più verboso e meno leggibile. Nella funzione si mescolano la logica applicativa -- il dover dividere ripetutamente per due -- e la gestione del possibile fallimento della divisione intera -- il \emph{pattern matching} sul risultato.

Il problema può essere risolto osservando una struttura comune a tutti i passaggi: se uno qualunque dei risultati intermedi restituisce \lstinline{None} allora la funzione fallisce immediatamente restituendo \lstinline{None}, come se si fosse verificata un'eccezione; altrimenti, si prosegue continuando a dividere il valore ottenuto.
Questo comportamento può essere fattorizzato in un'apposita funzione\footnote{In Scala per poter utilizzare \lstinline{andThen} come una funzione infissa questa deve essere dichiarata come \emph{extension method}~\cite{cit:scala-extension-methods} degli oggetti di tipo \lstinline{Option}.}:
\begin{lstlisting}[language=scala3]
extension [A](a: Option[A])
	def andThen[B](f: A => Option[B]): Option[B] =
		a match
			case None         => None
			case Some(result) => f(result)
\end{lstlisting}
che permetterà di scrivere la funzione \lstinline{eighth} in maniera molto più chiara:\todo{2. Sarebbe meglio utilizzare la sintassi "classica" OOP oppure sfruttare lo zucchero sintattico di Scala e scrivere il codice come nel commento?}
\begin{lstlisting}[language=scala3]
def eighth(n: Int): Option[Int] =
  halve(n).andThen(halve).andThen(halve)
  // halve(n) andThen halve andThen halve
\end{lstlisting}
La funzione è solo interessata dalla logica applicativa: dividere tre volte il valore in input. La gestione del fallimento è delegata ad \lstinline{andThen}\footnote{In Scala è già definito per \lstinline{Option} il metodo \lstinline{flatMap} con lo stesso comportamento di \lstinline{andThen} mostrato nell'esempio. Per questi esempi si preferisce utilizzare \lstinline{andThen} in quanto rende più chiaro il ruolo della funzione.} che, in caso di fallimento, restituisce immediatamente \lstinline{None}.

\todo{3. Per completezza si potrebbbe anche mostrare come alcuni linguaggi (come Kotlin, Rust) adottano un approccio analogo ma con un supporto diretto del compilatore e annesso zucchero sintattico? In questo caso la soluzione è ad hoc e non è generica come per le monadi ma comunque interessante}

\subsubsection{Gestione della lettura e modifica di uno stato globale}
\label{gestione-della-lettura-e-modifica-di-uno-stato-globale}

Come già discusso, è possibile modificare una funzione che accede e modifica uno stato globale in una funzione pura; in modo che riceva lo stato globale come input e ne restituisca la nuova versione in output.
Lo svantaggio di tale approccio sta nel fatto che il programmatore dovrà gestire manualmente il passaggio dello stato fra le diverse chiamate; ciò, oltre ad aumentare il \emph{boilerplate,} rende più probabile compiere errori come dimenticare di passare lo stato aggiornato a una chiamata successiva:
\begin{lstlisting}[language=scala3]
...
val (res1, state1) = statefulFunction1(initialState)
val (res2, state2) = statefulFunction2(res1, state1)
val (res3, state3) = statefulFunction3(res2, state2)
statefulFunction4(res3, state2)
// bug: statefulFunction4 ha ricevuto come input lo
// stato precedente a quello aggiornato da 
// statefulFunction3!
...
\end{lstlisting}

Anche in questo caso si può ottenere una soluzione che permetta di separare la logica applicativa dalla gestione manuale del passaggio dello stato fra un chiamata e la successiva:
\begin{lstlisting}[language=scala3]
extension [A, S](f: S => (A, S))
  def andThen[B](g: A => S => (B, S)): S => (B, S) =
    state0 => 
      val (result, state1) = f(state0)
      g(result)(state1)
\end{lstlisting}
Grazie a questa funzione è possibile riscrivere la funzione dell'esempio precedente in maniera più chiara:
\begin{lstlisting}[language=scala3]
...
statefulFunction1
  .andThen(statefulFunction2)
  .andThen(statefulFunction3)
  .andThen(statefulFunction4)
  .apply(initialState)
...
\end{lstlisting}
La gestione del \emph{threading} dello stato fra una chiamata e la successiva è delegato ad \lstinline{andThen}; in questo modo è possibile indicare la sequenza di operazioni che si vogliono compiere e lasciare che il boilerplate relativo al passaggio dello stato fra una chiamata e la successiva venga gestito automaticamente. 
