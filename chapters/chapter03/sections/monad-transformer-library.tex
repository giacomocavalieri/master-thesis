\section{L'approccio MTL}

\acf{MTL} è una libreria Haskell nata con l'obiettivo di semplificare la gestione di \term{stack} di monad transformer~\cite{cit:mtl} permettendo di comporre in maniera modulare computazioni che modellano la presenza di side effect.
Questa libreria è diventata nel tempo uno standard \emph{de facto} per la realizzazione di sistemi complessi nell'ecosistema di Haskell: circa il 30\% dei pacchetti presenti su Hackage\footnote{Hackage è l'archivio di software open source della comunità Haskell utilizzato per pubblicare librerie e programmi.} ne dipendono direttamente~\cite{cit:which-monads-haskell-developers-use-an-exploratory-study} ed è il sesto pacchetto più utilizzato su Stackage\footnote{Stackage è un'infrastruttura utilizzata per mantenere degli \term{snapshot} di pacchetti Hackage compatibili fra loro per ottenere delle \term{build} stabili.}~\cite{cit:evolution-of-a-haskell-repository-and-its-use-of-monads-an-exploratory-study-of-stackage}.

L'approccio di \ac{MTL} è stato poi adottato anche in altri linguaggi; per esempio in Scala nell'ecosistema di Typelevel è stata realizzata la libreria \emph{Cats MTL}~\cite{cit:cats-mtl}.

\subsection{Idea alla base di MTL}
L'obiettivo di \ac{MTL} è di permettere la descrizione di computazioni con side effect in maniera flessibile, modulare e facilmente componibile, con la possibilità di cambiare senza difficoltà l'implementazione sottostante utilizzata per gestire i side effect.
Questo obiettivo può essere raggiunto sfruttando nuovamente il meccanismo delle \term{type class} per definire \emph{famiglie} di effetti, ciascuna delle quali è caratterizzata da una serie di primitive che permettono di esprimere le operazioni che possono essere effettuate~\cite{cit:functional-programming-with-overloading-and-higher-order-polymorphism}.

\subsubsection{Modifica di uno stato mutabile}
Nello stile \ac{MTL} il side effect del cambiamento di uno stato mutabile può essere catturato dalla seguente interfaccia:
\scalaFromFile{7}{9}{monads/mtl/State.scala}
L'interfaccia descrive, per un generico costruttore di tipi \lstinline{M[_]}, le operazioni che questo deve fornire per permettere di gestire uno stato mutabile.
Sfruttando il meccanismo dei parametri impliciti di Scala è possibile definire una funzione che accetta come parametro il contesto \lstinline{M[_]} che implementa l'interfaccia \lstinline{State}:
\scalaFromFile{30}{36}{monads/mtl/State.scala}

Inoltre, è possibile modificare la funzione per rendere in maniera più ``dichiarativa'' i side effect che \lstinline{M[_]} deve avere. Innanzitutto, è necessario definire il seguente \term{type alias}:
\scalaFromFile{6}{6}{monads/mtl/State.scala}
In questo modo, sfruttando lo zucchero sintattico che Scala offre per definire \term{context bound}, è possibile modificare la firma della funzione precedente:
\begin{lstlisting}[language=scala3]
def update[S, M[_]: Monad: HasState[S]](f: S => S): M[Unit] = ???
\end{lstlisting}
Nella definizione del generico \lstinline{M[_]} è presente una lista di \term{type class} che ne descrivono le funzionalità; la definizione può essere letta come: \emph{``Dato un qualunque \lstinline{M} che sia una monade e abbia uno stato di tipo \lstinline{S}''}.
Infine, si può aggiungere al \term{companion object} di \lstinline{State} un metodo che permetta di ottenere il parametro passato implicitamente:
\scalaFromFile{11}{13}{monads/mtl/State.scala}
In questo modo il corpo della funzione potrà essere implementato come:
\scalaFromFile{38}{42}{monads/mtl/State.scala}

Dunque, \lstinline{M[_]} può essere interpretato come il generico ``contesto'' all'interno del quale si svolge la computazione descritta; i \term{context bound} permettono di vincolare il tipo \lstinline{M} indicando i side effect che dev'essere possibile svolgere all'interno di tale contesto.

\subsubsection{Fallimento di una computazione}
In maniera analoga a quanto mostrato per lo stato mutabile, è possibile definire in astratto il side effect del fallimento di una computazione:
\scalaFromFile{8}{9}{monads/mtl/Fail.scala}
Anche in questo l'interfaccia definisce per un generico contesto \lstinline{M[_]} le operazioni rilevanti per permettere di descrivere il fallimento di una computazione.

Un'operazione che richiede tale side effect può ricevere implicitamente in input una generica monade \lstinline{M[_]} che implementi tale interfaccia:
\scalaFromFile{29}{33}{monads/mtl/Fail.scala}
Anche in questo caso è possibile sfruttare lo zucchero sintattico di Scala per definire in maniera concisa il \term{context bound}:
\scalaFromFile{11}{12}{monads/mtl/Fail.scala}
\scalaFromFile{35}{40}{monads/mtl/Fail.scala}

\subsection{Composizione di più effetti}
\label{sub:composizione-di-piu-effetti}
Il grande vantaggio offerto da MTL sta nella possibilità di combinare fra loro side effect differenti con grande semplicità. Si consideri la seguente computazione:
\scalaFromFile{68}{73}{monads/mtl/MTL.scala}
Si possono osservare alcuni aspetti interessanti:
\begin{itemize}
  \item La computazione sfrutta due diversi effetti che sono resi espliciti tramite l'uso dei \term{context bound}: la definizione del tipo generico \lstinline{M[_]} può essere letta come \emph{``Dato un qualunque \lstinline{M} che permetta di modificare uno stato di tipo \lstinline{S}, che permetta di far fallire la computazione e che sia una monade''}. Mentre i primi due requisiti sono fondamentali per poter utilizzare i side effect, richiedere che \lstinline{M} sia una monade ha lo scopo di poter accedere all'operazione di \lstinline{flatMap} per mettere in sequenza più operazioni
  \item Non viene concretizzato il tipo che permetterà di interpretare i diversi effetti, tutta la definizione è basata su un generico \lstinline{M[_]}; la semantica delle operazioni potrà essere stabilita in un secondo momento in maniera indipendente. Perciò, a differenza dei \term{transformer}, la descrizione della computazione e di come questa verrà eseguita sono disaccoppiate
\end{itemize}

Inoltre viene eliminata la rigidità data dai \term{monad transformer} nel comporre operazioni con tipi differenti: in un contesto \lstinline{M[_]} è possibile eseguire operazioni che richiedono entrambi gli effetti, solo uno di essi o nessuno di essi. Nell'esempio precedente viene utilizzata la funzione \lstinline{divide} che richiede solo la capacità di fallire; ciò è possibile dato che è uno degli effetti forniti nel contesto della funzione \lstinline{effect}. Non è necessario applicare operazioni di \term{lifting} nè modificare artificialmente il tipo di ritorno delle funzioni per semplificarne la composizione come veniva fatto nel caso dei \term{transformer}.

\subsection{Interpretazione della computazione}
Una computazione descritta secondo lo stile \ac{MTL} definisce in astratto quali sono le operazioni che dovranno essere eseguite; quindi, per poterla interpretare è necessario fornire un tipo concreto che implementi tutte le operazioni richieste.
L'implementazione viene sempre fornita come istanza di una \term{type class} in maniera analoga a quanto fatto per le monadi.
Per esempio il \term{transformer} \lstinline{OptionT} può essere usato come meccanismo concreto per interpretare computazioni che possono fallire:
\scalaFromFile{14}{15}{monads/mtl/Fail.scala}
Può essere utile riprendere l'interpretazione delle \term{type class} come predicati: fornire un'istanza equivale a dimostrare che \lstinline{OptionT} può fallire e la dimostrazione è data dall'implementazione del metodo \lstinline{fail}. Tuttavia, come già descritto in precedenza, \lstinline{OptionT} non è l'unica monade che permette di modellare una computazione che fallisca; il fallimento può essere modellato anche nella monade \lstinline{IO} tramite l'uso di eccezioni:
\scalaFromFile{17}{18}{monads/mtl/Fail.scala}

Una volta definite tali istanze sarà possibile sfruttarle per interpretare una qualunque computazione che richieda il side effect del fallimento:
\scalaFromFile{42}{45}{monads/mtl/Fail.scala}
Utilizzare un diverso interprete per la computazione si riduce a indicare quale monade utilizzare come concreta implementazione del contesto \lstinline{M[_]}. Nel primo caso viene utilizzata la monade \lstinline{IO} e, se la computazione dovesse fallire, verrà sollevata un'eccezione; nel secondo caso viene utilizzato \lstinline{OptionT} e in caso di fallimento il risultato sarà \lstinline{None}.

L'istanza per l'effetto della modifica di uno stato è simile a quella del fallimento; in questo caso si mostra come il \term{transformer} \lstinline{StateT} permetta di ottenere tale effetto:
\scalaFromFile{15}{18}{monads/mtl/State.scala}

\subsubsection{Integrazione di stack di monadi e MTL}
Le istanze fornite fino a questo momento permettono di ottenere semplici effetti ma non sono sufficienti per interpretare computazioni che necessitino di più effetti contemporaneamente: per esempio il semplice \lstinline{OptionT} non può essere utilizzato per interpretare computazioni che richiedono uno stato mutabile.
Una computazione simile richiede necessariamente uno \term{stack} di monadi che abbia al proprio interno una monade in grado di gestire il cambiamento di stato (come per esempio \lstinline{StateT}), in quel caso l'intero \term{stack} potrà essere sfruttato per interpretare la computazione:
\scalaFromFile{21}{25}{monads/mtl/State.scala}
In questo caso la definizione dell'istanza è più complessa:
\begin{itemize}
  \item L'istanza viene definita per un generico \term{transformer} \lstinline{T} e una generica monade \lstinline{M}; se \lstinline{M} può gestire uno stato mutabile di tipo \lstinline{S}, allora anche \lstinline{T} con \lstinline{M} al proprio interno potrà farlo
  \item Per poter ottenere lo stato globale, \lstinline{T} delega alla monade \lstinline{M} l'operazione di \lstinline{get} che viene poi inserita all'interno del \term{transformer} tramite \lstinline{lift}. L'operazione di \lstinline{set} è implementata in maniera analoga
\end{itemize}

Le due istanze \lstinline{stateHasState} e \lstinline{transformerHasState} appena mostrate permettono di ottenere un'istanza di \lstinline{State} per un qualunque \term{stack} di monadi che abbia \lstinline{StateT} al proprio interno: \lstinline{stateHasState} funge da caso base per la ricorsione mentre \lstinline{transformerHasState} permette di applicare tutti i \lstinline{lift} necessari automaticamente indipendentemente dalla composizione dello \term{stack} di monadi.

Si consideri il seguente \term{stack}: \lstinline{Transformer1[Transformer2[StateT[Int, IO, _], _], _]} (si supponga che \lstinline{Transformer1} e \lstinline{Transformer2} siano due \term{monad transformer}). Il compilatore Scala potrà derivare in automatico un'istanza di \lstinline{HasState[Int]} per l'intero \term{stack} secondo il seguente procedimento:
\begin{enumerate}
  \item \lstinline{StateT[Int, IO, _]} ha un'istanza di \lstinline{State} che può essere ottenuta tramite \lstinline{stateHasState}
  \item Poiché \lstinline{Transformer2} è un \term{transformer} e \lstinline{StateT[Int, IO, _]} è una monade con un'istanza di \lstinline{State} allora anche \lstinline{Transformer2[StateT[Int, IO, _], _]} ha un'istanza di \lstinline{State} che può essere sintetizzata grazie a \lstinline{transformerHasState}
  \item Poiché \lstinline{Transformer1} è un \term{transformer} e \lstinline{Transformer2[StateT[Int, IO, _], _]} è una monade con un'istanza di \lstinline{State} -- generata al passaggio precedente -- allora anche \lstinline{Transformer1[Transformer2[StateT[Int, IO, _], _], _]} ha un'istanza di \lstinline{State} che può essere sintetizzata grazie a \lstinline{transformerHasState}
\end{enumerate}

Nel concreto una chiamata a \lstinline{get} in questo \term{stack} sarà interpretata come \lstinline{StateT.get.lift[Transformer2].lift[Transformer1]}; tutti i \lstinline{lift} saranno applicati automaticamente grazie alle istanze generate dal compilatore.

Il caso di fallimento è analogo a quanto osservato per \lstinline{State}; se un \term{transformer} ha al proprio interno una monade che permette di ottenere tale side effect allora può a sua volta fornire un'istanza di \lstinline{Fail}:
\scalaFromFile{21}{24}{monads/mtl/Fail.scala}

Una volta definite tali istanze una computazione complessa come \lstinline{effects} mostrata nella \Cref{sub:composizione-di-piu-effetti} può essere interpretata specificando uno stack contenente i transformer necessari:
\scalaFromFile{77}{88}{monads/mtl/MTL.scala}
Il compilatore fornirà implicitamente le istanze necessarie a interpretare la computazione. A seconda dello \term{stack} utilizzato si potrà dare una diversa interpretazione degli effetti specificati:
\begin{itemize}
  \item Nel primo risultato ogni modifica allo stato prima del fallimento della computazione sarebbe comunque valida: infatti, il tipo di ritorno non incapsula anche lo stato \lstinline{Int} all'interno di \lstinline{Option}
  \item Nel secondo caso lo stato è gestito in maniera ``transazionale'': se la computazione fallisce allora lo stato non viene modificato
  \item Nel terzo caso l'interruzione della computazione viene ottenuta tramite le eccezioni e interpretarla potrebbe comportare il fallimento del programma a \term{runtime}
\end{itemize}
