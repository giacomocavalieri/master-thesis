\section{Free monad come effect system}
Il meccanismo delle free monad introdotto in questo capitolo è sicuramente un esempio di effect system in quanto permette di descrivere e interpretare effetti arbitrati. Inoltre, è stato mostrato come sia possibile modellare separatamente e comporre fra loro DSL che descrivono effetti pertinenti a domini differenti.

I programmi realizzati tramite l'uso di questo meccanismo non sono altro che strutture dati che descrivono in maniera astratta quali effetti devono avere luogo. L'interpretazione degli effetti avviene in un secondo momento e può assegnare un significato agli effetti in base alle necessità del contesto in cui si sta lavorando.

\subsection{Definizione di effetti arbitrari}
Anche in questo caso, così come per l'approccio MTL, è possibile definire qualunque tipo di effetto sia ritenuto rilevante ai fini della computazione; la scelta del livello di dettaglio con cui questi vengono descritti è lasciato a discrezione dell'autore del DSL. \mustfix{Inserire citazione all'articolo che parla di questo aspetto}

In seguito viene ripreso l'esempio mostrato per MTL nella \Cref{sec:mtl-effetti-arbitrari} reimplementandolo in maniera analoga nello stile delle free monad. Gli effetti da modellare sono il recupero di un utente, il salvataggio e la cancellazione:
\scalaFromFile{9}{12}{monads/free/UserStore.scala}
Il DSL è analogo a quanto definito per MTL; una differenza importante sta nella necessità, nel caso delle free monad, di dover definire dei costruttori per ogni operazione del DSL:
\scalaFromFile{14}{22}{monads/free/UserStore.scala}

Come già mostrato in precedenza, è possibile codificare programmi complessi in termini delle operazioni di base. La logica applicativa sarà basata sull'interfaccia astratta definita dall'insieme di operazioni che compongono il DSL:
\scalaFromFile{33}{37}{monads/free/UserStore.scala}
L'implementazione è essenzialmente identica a quella mostrata per l'esempio di MTL; l'unica sostanziale differenza sta nel tipo di ritorno delle funzioni. Da un lato MTL richiede di definire una generica monade \lstinline{M} all'interno della quale sarà incapsulato il valore di ritorno; l'approccio delle free monad invece restituisce un valore di tipo \lstinline{Program} parametrizzato su un generico DSL \lstinline{I}.
L'implementazione di \lstinline{Program} fa si che questo sia automaticamente una monade; la firma del metodo non dovrà quindi occuparsi di specificare alcun vincolo di questo tipo.

Sebbene l'implementazione di una free monad operazionale abbia richiesto un certo sforzo, il risultato finale per chi deve utilizzare il tipo \lstinline{Program} è piuttosto intuitivo:
\begin{itemize}
  \item Vengono definiti più DSL relativi ai diversi effetti del proprio sistema
  \item Si definiscono gli \emph{smart constructor} per ciascuna operazione
  \item Le operazioni di base possono essere composte in maniera naturale e mantenendo una sintassi ``imperativa'' grazie alla \emph{for comprehension}
  \item Il programma finale può essere interpretato passo a passo definendo la semantica delle operazioni in un secondo momento
\end{itemize}

Nell'approccio MTL è necessario indicare esplicitamente il rispetto dell'interfaccia di monade per poter mettere in sequenza le diverse operazioni tramite l'uso di \lstinline{flatMap}. Al contrario, in questo caso, una volta che un'operazione è stata incapsulata all'interno di \lstinline{Program} il programmatore guadagna automaticamente la possibilità di combinarla in sequenza con altri programmi.

\subsubsection{Interpretazione e testing dei programmi}
Potendo definire interpreti arbitrari per le istruzioni di un programma è piuttosto facile realizzare dei \emph{mock} che permettono di testare effetti complessi senza dover complicare significativamente il processo di testing della logica applicativa.

A differenza dell'approccio MTL, l'interpretazione può essere definita come un'analisi della sequenza di istruzioni del programma.
In questo caso, quindi, non è necessario introdurre esplicitamente il concetto di monade che rimane un dettaglio implementativo sfruttato da \lstinline{Program}.

Nel caso di MTL, poiché il vincolo di monade è reso esplicito all'interno della firma del metodo, per implementare un interprete è necessario avere un tipo di dato che rispetti l'istanza di monade introducendo la complessità di dover lavorare esplicitamente con i monad transformer.
In questo caso, invece, l'interpretazione può essere implementata come una semplice funzione ricorsiva che di passo in passo assegna semantica all'istruzione corrente e interpreta la continuazione.

Si consideri per esempio, riprendendo la funzione \lstinline{updateAge} mostrata nel caso di MTL:
\scalaFromFile{47}{52}{monads/free/UserStore.scala}
Al \Cref{lst:mock-free} è possibile osservare l'implementazione di un interprete che implementa il \emph{mock} di un database in memoria.

\begin{figure}
  \begin{lstlisting}[language=scala3, label={lst:mock-free}, caption={Esempio di un interprete di test per un programma che usa lo \lstinline{UserStoreDSL}. In questo caso l'uso di una mappa degli utenti permette di testare semplicemente la logica applicativa delle operazioni}]
extension [A](program: Program[UserStoreDSL, A])
  @tailrec
  def runMocked(users: Map[UserId, User]): Map[UserId, User] =
    program.next match
      case ProgramView.Return(value) => users
      case ProgramView.Then(instruction, continuation) =>
        instruction match
          case Get(userId) =>
            continuation(users.get(userId)).runMocked(users)
          case Save(user) =>
            val updatedUsers = users + ((user.id, user))
            continuation(()).runMocked(updatedUsers)
          case Delete(userId) =>
            continuation(()).runMocked(users - userId)

def testUpdateAgeDeletesUnderageUsers: Unit =
  val user = User(UserId(1), "Giacomo", 12)
  val users = Map(user.id -> user)
  val finalUsers = updateAge(user.id).runMocked(users)
  assert(finalUsers.isEmpty)
  \end{lstlisting}
\end{figure}
