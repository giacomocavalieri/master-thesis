\section{Implementazione di una Free Monad}

L'AST descritto tramite una free monad ha l'obiettivo di catturare la struttura sintattica di una computazione monadica~\cite{cit:programming-monads-operationally-with-unimo}.
Come già descritto in precedenza, l'interfaccia delle monadi permette di descrivere una computazione come una sequenza di passi successivi che, una volta terminati, restituiscono un valore.

In seguito verrà formalizzata tale definizione arrivando in maniera graduale alla definizione di una free monad.
L'encoding adottato è quello delle cosiddette \emph{Free Operational Monads}~\cite{cit:programming-monads-operationally-with-unimo}; tuttavia esistono svariate tecniche per ottenere risultati analoghi~\cite{cit:data-types-a-la-carte,cit:freer-monads-more-extensible-effects,cit:fusion-for-free}.

\subsection{Descrizione astratta di un programma monadico}
\subsubsection{Istruzioni di un programma monadico}
In generale si può considerare un programma monadico come definito da una serie di istruzioni appartenenti ad un insieme $I$ messe in sequenza fra loro.
Ogni istruzione, una volta eseguita, può restituire un valore differente; si consideri per esempio il seguente programma definito nella monade \lstinline{State}:
\begin{lstlisting}[language=scala3]
def program[S](f: S => S): State[S, String] =
  for
    state <- get
    newState = f(state)
    _ <- set(newState)
  yield "end"
\end{lstlisting}
Le operazioni di base di cui si compone sono \lstinline{get} e \lstinline{set}; mentre la prima restituisce un valore di tipo \lstinline{S}, la seconda restituisce \lstinline{Unit}.
Quindi, in questo caso, l'insieme delle istruzioni sarebbe $I = \{get, set\}$; questo insieme può essere codificato in Scala con una semplice enumerazione:
\scalaFromFile{7}{9}{monads/free/State.scala}
Un valore di tipo \lstinline{StateDSL[S, A]} rappresenta una singola istruzione che opera su uno stato di tipo \lstinline{S} e restituisce un valore di tipo \lstinline{A}:
\begin{itemize}
  \item \lstinline{Get} opera su uno stato di tipo \lstinline{S} e lo restituisce, quindi ha tipo \lstinline{StateDSL[S, S]}
  \item \lstinline{Set} ha tipo \lstinline{StateDSL[S, Unit]} in quanto, come descritto in precedenza, ha l'effetto di modificare lo stato \lstinline{S} ma non restituisce alcun valore d'interesse
\end{itemize}

\subsubsection{Esecuzione di istruzioni in un programma monadico}
Come descritto all'inizio del capitolo una free monad cattura in una struttura dati la struttura sintattica di un programma monadico.
Un tipo di dato che rappresenti l'AST di un programma monadico deve quindi permettere di rappresentare l'esecuzione di una singola istruzione.
Per disaccoppiare l'AST dallo specifico tipo di istruzioni eseguite un programma verrà definito in maniera generica rispetto alle istruzioni che può eseguire:
\begin{lstlisting}[language=scala3]
enum Program[I[_], A]: ...
\end{lstlisting}
Il tipo generico \lstinline{I} rappresenta il set di istruzioni che possono essere eseguite dal programma monadico, mentre \lstinline{A} rappresenta il tipo del valore di ritorno ottenuto dall'esecuzione del programma. Per esempio un valore di tipo \lstinline{Program[StateDSL[Int, _], String]} rappresenterà un programma monadico che può eseguire operazioni su uno stato di tipo \lstinline{Int} e restituisce un valore di tipo \lstinline{String}.

Per poter costruire un programma che esegue una singola istruzione può essere definito un caso ad hoc dell'enumerazione e uno \emph{smart constructor}:
\begin{lstlisting}[language=scala3]
enum Program[I[_], A]:
  case Instruction[I[_], A](instruction: I[A])
    extends Program[I, A]

object Program:
  def fromInstruction[I[_], A](instruction: I[A]): Program[I, A] =
    Instruction(instruction)
\end{lstlisting}

Questa definizione permette di creare semplici programmi che eseguono una singola istruzione:
\begin{lstlisting}[language=scala3]
  def get[S] = Program.fromInstruction(StateDSL.Get())
  def set[S](state: S) =
    Program.fromInstruction(StateDSL.Set(state))
\end{lstlisting}

Chiaramente un programma che esegue una singola istruzione non è particolarmente interessante; per poter combinare semplici istruzioni e realizzare programmi complessi è necessario espandere la definizione di \lstinline{Program} fornita.

\subsubsection{Terminazione di un programma monadico}
Un'ulteriore caratteristica di ogni computazione monadica -- espressa dal metodo \lstinline{pure} dell'interfaccia delle monadi -- sta nella possibilità di poter inserire un qualunque valore all'interno della sequenza delle computazioni senza aggiungervi ulteriori side effect.
Tuttavia, fino a questo momento un programma monadico si limita a permettere l'esecuzione di una singola istruzione appartenente al set \lstinline{I} specificato. L'AST di \lstinline{Program} può essere esteso per permettere questa operazione:
\begin{lstlisting}[language=scala3]
enum Program[I[_], A]:
  case Instruction[I[_], A](instruction: I[A])
    extends Program[I, A]
  case Return[I[_], A](value: A) extends Program[I, A]

object Program:
  def fromValue[I[_], A](value: A): Program[I, A] =
    Return(value)
\end{lstlisting}

Questo ulteriore costruttore permette di definire programmi che restituiscono valori senza alcun side effect, ovvero senza eseguire nessuna delle operazioni dell'insieme \lstinline{I}. Per esempio
\begin{lstlisting}[language=scala3]
val program: Program[StateDSL[Int, _], String] =
  Program.fromValue("result")
\end{lstlisting}
Restituisce il valore \lstinline{"result"} senza modificare lo stato mutabile tramite le operazioni definite nello \lstinline{StateDSL}.

\subsubsection{Messa in sequenza di programmi}
L'aspetto più importante di una monade sta nella possibilità di mettere in sequenza operazioni tramite l'uso di \lstinline{flatMap}.
Anche l'AST di \lstinline{Program} deve esprimere questo concetto per poter catturare la struttura di un programma monadico. Per questo motivo viene definito un ultimo costruttore:
\scalaFromFile{9}{16}{monads/free/lib/Free.scala}
Il costruttore \lstinline{Then} permette di catturare la messa in sequenza di operazioni monadiche: il primo argomento è il primo programma da eseguire che produrrà un valore di tipo \lstinline{A}; il secondo argomento è una continuazione che, preso il valore prodotto dal primo programma, restituisce un secondo programma da eseguire.

È possibile definire un \emph{extension method} per rendere più facile la costruzione di programmi complessi:
\begin{lstlisting}[language=scala3]
extension [I[_], A](program: Program[I, A])
  def andThen[B](continuation: A => Program[I, B]) =
    Then(program, continuation)
\end{lstlisting}

Per esempio il programma con stato mostrato alla sezione precedente può essere ora espresso come:
\begin{lstlisting}[language=scala3]
def program[S](f: S => S): Program[StateDSL[S, _], String] =
  get.andThen { state => 
    val newState = f(state)
    set(newState).andThen { _ => 
      Program.fromValue("end")
    }
  }
\end{lstlisting}

È possibile osservare come \lstinline{Program} rispetti per costruzione l'interfaccia delle monadi, indipendentemente dal tipo di istruzioni che utilizza: i costruttori \lstinline{Result} e \lstinline{Then} equivalgono rispettivamente alle operazioni \lstinline{pure} e \lstinline{flatMap}:
\scalaFromFile{36}{41}{monads/free/lib/Free.scala}
In questo modo è possibile definire programmi nella monade \lstinline{Program} sfruttando lo zucchero sintattico della \emph{for comprehension}; l'esempio precedente può essere riscritto in maniera più chiara come:
\begin{lstlisting}[language=scala3]
def program[S](f: S => S): Program[StateDSL[S, _], String] =
  for
    state <- get
    newState = f(state)
    _ <- set(newState)
  yield "end"
\end{lstlisting}
