Fino ad ora si è osservato come sia possibile modellare i side effect tramite l'uso di monadi, sia ricorrendo ai \term{monad transformers} che all'approccio \ac{MTL}. In quest'ultimo caso la computazione viene descritta tramite l'uso di metodi astratti definiti in un'interfaccia; l'interpretazione concreta di tali chiamate a metodo può essere definita in un secondo momento assegnando la semantica desiderata a ciascuna di esse.

L'approccio adottato dalle \term{free monad} consiste nel descrivere la computazione tramite un \ac{AST} che permette di comporre in sequenza più operazioni definite in maniera astratta. In questo modo è possibile dare semantica alle operazioni definendo interpreti che, attraversando l'\ac{AST}, possono tradurre le operazioni astratte in una versione ``eseguibile''.
