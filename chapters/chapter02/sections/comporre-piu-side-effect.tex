\section{Problemi nel comporre più side effect}

Per rendere evidenti le problematiche dell'approccio mostrato nel capitolo precedente verrà preso ad esempio il \term{parsing} monadico di stringhe.
Nell'implementare un'istanza di monade per un \term{parser} si potrà osservare come il codice sia essenzialmente una ripetizione di quello di alcune monadi di base già descritte.

\subsection{Parsing monadico}
\label{sec:parsing-monadico}
Un \term{parser} può essere modellato come una funzione che consuma una stringa (o una sua porzione) producendo un risultato come un \ac{AST}. Nel caso in cui la stringa non sia conforme al linguaggio descritto dal \term{parser} questo processo potrebbe fallire senza produrre alcun valore.
Chiaramente la modellazione riportata è una semplificazione che cattura solamente l'essenza di un \term{parser} tralasciando altri aspetti come la possibilità di restituire più risultati -- nel caso in cui la grammatica del \term{parser} sia ambigua -- o di indicare la posizione nella stringa dove si è verificato un errore.

Nella definizione riportata, per quanto semplice, è possibile individuare due tipologie di side effect già incontrate in precedenza: la stringa su cui opera il \term{parser} è uno stato mutabile e l'operazione di \term{parsing} potrebbe fallire.
Il modello di \term{parser} così descritto può essere rappresentato tramite un'apposita classe:
\scalaFromFile{5}{5}{monads/Parser.scala}
Un \term{parser} è modellato come una funzione che prende in input la stringa da consumare e restituisce il risultato del \term{parsing} che consiste nella coppia composta dalla porzione non consumata della stringa e dal risultato vero e proprio. Inoltre, dato che il processo di \term{parsing} potrebbe fallire, il risultato viene modellato come un valore opzionale.

Per esempio un \term{parser} che fallisce sempre può essere definito come:
\scalaFromFile{19}{19}{monads/Parser.scala}
Un \term{parser} che fallisce in caso di stringhe vuote e altrimenti ne consuma il primo carattere restituendolo come risultato può essere implementato come:
\scalaFromFile{21}{22}{monads/Parser.scala}

È possibile definire per la classe \lstinline{Parser} un'istanza di monade che rispetta le leggi monadiche:
\scalaFromFile{8}{17}{monads/Parser.scala}

\begin{itemize}
  \item \lstinline{pure} è l'operazione di costruzione di un \term{parser} che non attua alcun side effect: restituisce il valore \lstinline{a} senza consumare la stringa e senza fallire
  \item \lstinline{flatMap} permette di concatenare fra loro dei parser ottenendo il seguente comportamento: se uno dei \term{parser} fallisce allora fallisce l'intera sequenza, altrimenti verranno usati i due \term{parser} in sequenza e il secondo consumerà la stringa rimanente non consumata dal primo
\end{itemize}

Grazie a quest'interfaccia monadica è possibile definire \term{parser} complessi in maniera modulare, combinando fra loro blocchi elementari tramite l'uso di \lstinline{flatMap}~\cite{cit:monadic-parsing-in-haskell}. Per esempio un \term{parser} che consuma esattamente due caratteri dalla testa di una stringa può essere definito come:
\scalaFromFile{28}{32}{monads/Parser.scala}
Per combinare fra loro \term{parser} non è necessario conoscere i dettagli implementativi della classe \lstinline{Parser} ma è sufficiente sapere che questo è una monade. In questo modo se l'implementazione di \lstinline{Parser} dovesse cambiare -- come verrà mostrato in seguito -- non sarà necessario riscrivere i parser ottenuti combinando elementi di base in questo modo.

Inoltre, poiché \lstinline{Parser} è una monade, è possibile sfruttare tutte le funzioni generiche definite in precedenza per le monadi. L'esempio precedente può essere generalizzato ad un \term{parser} che consuma esattamente $n$ caratteri dalla testa di una stringa:
\scalaFromFile{24}{25}{monads/Parser.scala}

\subsection{Duplicazione di codice}
Osservando la definizione di \lstinline{Parser} e la sua istanza di monade si può notare che questa ha una struttura simile a quella di \lstinline{State} e \lstinline{Option} viste al capitolo precedente:
\begin{lstlisting}[language=scala3]
final case class Parser[A](
  parse: String => Option[(A, String)])
final case class State[S, A](
  run:   S      =>        (A, S))
\end{lstlisting}
Ciò non dovrebbe stupire: come descritto in precedenza, \lstinline{Parser} gestisce uno stato mutabile e può fallire nel processo di \term{parsing}. Dovendo ottenere questi effetti la struttura della monade \lstinline{Parser} presenterà delle similarità con quella di \lstinline{State} e \lstinline{Option}. Ciò è ancora più evidente se si osservano le implementazioni di \lstinline{pure} e \lstinline{flatMap}. La funzione \lstinline{pure} non fa altro che applicare l'effetto neutro sia di \lstinline{State} che di \lstinline{Option}:
\begin{lstlisting}[language=scala3]
def pure[A](a: A): Option[A] = Some(a)
def pure[A](a: A): State[S, A] = State(s => (a, s))
def pure[A](a: A): Parser[A] = Parser(s => Some((a, s)))
\end{lstlisting}
Anche nel caso di \lstinline{flatMap} è presente una ripetizione della logica di base che contraddistingue \lstinline{State} e \lstinline{Option}:
\scalaFromFile{11}{17}{monads/Parser.scala}
Il \term{pattern matching} più esterno attua la logica di \term{short-circuiting} della monade \lstinline{Option}, restituendo il valore \lstinline{None} nel caso in cui il \term{parsing} fallisca. Mentre la gestione dello stato -- in questo caso il passaggio della nuova stringa \lstinline{string1} al secondo \term{parser} -- è analoga al \term{threading} effettuato dalla monade \lstinline{State}.

L'implementazione di \lstinline{Parser} presenta quindi una significativa duplicazione di codice: l'istanza di monade è essenzialmente ottenuta combinando le due implementazioni di \lstinline{State} e \lstinline{Option}. Il problema scaturisce dalla necessità di dover combinare più side effect in un'unica monade andando a replicare la struttura delle monadi di base che forniscono gli effetti desiderati.

Per rendere ancora più evidente il problema si immagini di dover estendere il \term{parser} aggiungendo la possibilità di effettuare \term{logging} su \term{standard output}. In questo caso la monade \lstinline{Parser} dovrebbe fornire anche le funzionalità della monade \lstinline{IO}:
\begin{lstlisting}[language=scala3]
final case class Parser[A](
  parse: String => (() => Option[(A, String)]))

def flatMap[B](f: A => Parser[B]): Parser[B] = 
  Parser(string0 => 
    val unsafeRun1 = p.parse(string0)
    unsafeRun1() match
      case None => None
      case Some((result, string1)) =>
        val unsafeRun2 = f(result).parse(string1)
        unsafeRun2())
\end{lstlisting}
L'implementazione di \lstinline{flatMap} è ulteriormente complicata dalla necessità di dover gestire l'esecuzione delle computazioni con side effect; inoltre, si può nuovamente notare come l'implementazione consiste nella combinazione delle implementazioni delle tre monadi viste in precedenza: \lstinline{State}, \lstinline{IO} e \lstinline{Option}.

Una soluzione ideale consentirebbe di definire \lstinline{Parser} come una combinazione di monadi elementari, ciascuna delle quali fornisce la possibilità di gestire uno specifico effetto senza doverne implementare nuovamente la logica. Per risolvere tale problema è possibile sfruttare il concetto di \term{monad transformer} che permette di comporre in maniera modulare più monadi per gestire molteplici effetti contemporaneamente.
