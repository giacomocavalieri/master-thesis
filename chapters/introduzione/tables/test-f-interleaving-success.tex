\begin{table}[!ht]
  \centering
  \begin{tabular}{ll}
    \toprule
    \lstinline|test1|                & \lstinline|test2|                \\
    \midrule
    \lstinline|oldCounter = counter| &                                  \\
    \lstinline|counter = 0|          &                                  \\
    \lstinline|f(1) shouldBe 2|      &                                  \\
                                     & \lstinline|oldCounter = counter| \\
                                     & \lstinline|counter = 1|          \\
                                     & \lstinline|f(1) shouldBe 3|      \\
    \lstinline|counter = oldCounter| &                                  \\
                                     & \lstinline|counter = oldCounter| \\
    \bottomrule
  \end{tabular}
  \caption{Esempio di \term{interleaving} delle operazioni dei test \lstinline|test1| e \lstinline|test2| che porterebbe al successo di entrambi i test.}
  \label{table:test-f-interleaving-success}
\end{table}