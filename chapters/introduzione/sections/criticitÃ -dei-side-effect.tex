\section*{Criticità dei side effect}
\label{section:criticita-dei-side-effect}

Il fatto che una funzione possa avere interazioni con il mondo esterno presenta però alcune criticità.
Prendiamo ad esempio una funzione Scala \lstinline{f : Int => Int}\footnote{La notazione \lstinline{f : Int => Int} indica una funzione chiamata \lstinline{f} che prende come argomento un valore di tipo \lstinline{Int} e restituisce un valore di tipo \lstinline{Int}.}; a una prima osservazione potrebbe risultare sorprendente, ma in generale non varrà l'uguaglianza per cui \lstinline{f(1) + f(1) = 2 * f(1)}.
Si consideri la seguente implementazione di \lstinline{f}:
\begin{lstlisting}[language=scala3]
var counter = 0
def f(x: Int): Int =
  counter = counter + 1
  x + counter
\end{lstlisting}
In questo caso \lstinline{2 * f(1) = 4} ma \lstinline{f(1) + f(1) = 5}.
Infatti \lstinline{f} legge e modifica una variabile globale dalla quale dipende il suo valore di ritorno; per via di questo side effect ogni chiamata successiva alla funzione, a parità di argomento, produrrà un valore differente.
Dall'esempio è possibile intuire come, in presenza di side effect, sia più difficile ragionare sul comportamento delle funzioni: per poterlo comprendere non è sufficiente osservare il corpo della funzione in analisi ma è necessario conoscere il contesto nel quale questa viene invocata.

Al contrario, quando si ha a che fare con una funzione pura è possibile manipolarla ed effettuare \term{refactoring} del codice in maniera \emph{equazionale}: è sempre possibile sostituire alla chiamata di funzione il suo valore di ritorno senza il rischio di alterare la semantica del programma. Se \lstinline{f} è una funzione pura allora \lstinline{f(1) + f(1) = 2 * f(1)}.

\subsection*{Complessità del refactoring}
\label{subsection:complessita-del-refactoring}
\emph{L'ordine} e il \emph{numero delle invocazioni} di funzioni con side effect ha importanza nel determinare il comportamento complessivo di un programma e -- come mostrato nell'esempio della funzione \lstinline{f} -- modificare uno di questi due fattori può comportarne uno stravolgimento.
Questa mancanza di trasparenza referenziale porta il codice ad avere una maggiore resistenza al cambiamento: è più difficile per il programmatore -- così come per strumenti automatici -- effettuare del \term{refactoring} avendo la certezza di mantenere inalterata la semantica del programma originale~\cite{cit:towards-purity-guided-refactoring-in-java}.

Per utilizzare le efficaci parole di Robert Martin ``I side effect sono bugie. La tua funzione promette di fare una cosa, ma in realtà fa anche qualcos'altro \emph{di nascosto}. [...] Sono falsità che spesso risultano in strani accoppiamenti temporali e comportano dipendenze nell'ordine delle funzioni''\footnote{Traduzione dal testo originale: ``Side effects are lies. Your function promises to do one thing, but it also does other \emph{hidden} things. [...] They are [...] mistruths that often result in strange temporal couplings and order dependencies''~\cite[p.~44]{cit:clean-code-a-handbook-of-agile-software-craftsmanship}.}.

\subsection*{Difficoltà nel testare funzioni impure}
\label{subsection:difficolta-nel-testare-funzioni-impure}
Le stesse motivazioni che rendono difficile ragionare e compiere \term{refactoring} di codice con side effect ne complicano anche la fase di \term{unit testing}.
Per testare una funzione impura in isolamento è necessario che vengano effettuate apposite operazioni prima e dopo lo svolgimento di ciascun test per garantire che il risultato non possa cambiare a seconda del loro ordine di esecuzione.

Immaginando di dover testare il comportamento di \lstinline{f}:
\begin{lstlisting}[language=scala3]
def test1: Unit =
  val oldCounter = counter
  counter = 0 // set up
  f(1) shouldBe 2
  counter = oldCounter // tear down

def test2: Unit =
  val oldCounter = counter
  counter = 1 // set up
  f(1) shouldBe 3
  counter = oldCounter // tear down
\end{lstlisting}
Si può notare come la dipendenza implicita introdotta dal side effect di \lstinline{f} renda necessario dover modificare opportunamente lo stato globale \lstinline{counter} prima e dopo l'esecuzione di ogni test.
Non solo, tali test non potranno nemmeno essere mandati in esecuzione simultaneamente\footnote{Si potrebbe ovviare a questo problema introducendo un meccanismo di \term{locking} della risorsa condivisa per gestire l'esecuzione parallela dei test. Tuttavia, questa strategia aggiungerebbe una complessità accidentale non indifferente per testare una funzione semplice come \lstinline{f}!} in quanto l'\term{interleaving} delle loro operazioni potrebbe portare a fallimenti imprevedibili.
Due esempi di \term{interleaving} che portano a risultati differenti dei test sono riportati alle \Cref{table:test-f-interleaving-fail,table:test-f-interleaving-success}.

\begin{table}[!ht]
  \centering
  \begin{tabular}{ll}
    \toprule
    \lstinline|test1|                & \lstinline|test2|                \\
    \midrule
    \lstinline|oldCounter = counter| &                                  \\
    \lstinline|counter = 0|          &                                  \\
    \lstinline|f(1) shouldBe 2|      &                                  \\
                                     & \lstinline|oldCounter = counter| \\
                                     & \lstinline|counter = 1|          \\
                                     & \lstinline|f(1) shouldBe 3|      \\
    \lstinline|counter = oldCounter| &                                  \\
                                     & \lstinline|counter = oldCounter| \\
    \bottomrule
  \end{tabular}
  \caption{Esempio di \term{interleaving} delle operazioni dei test \lstinline|test1| e \lstinline|test2| che porterebbe al successo di entrambi i test.}
  \label{table:test-f-interleaving-success}
\end{table}
\begin{table}[!ht]
  \centering
  \begin{tabular}{ll}
    \toprule
    \lstinline|test1|                & \lstinline|test2|                \\
    \midrule
    \lstinline|oldCounter = counter| &                                  \\
                                     & \lstinline|oldCounter = counter| \\
    \lstinline|counter = 0|          &                                  \\
                                     & \lstinline|counter = 1|          \\
    \lstinline|f(1) shouldBe 2|      &                                  \\
    \lstinline|counter = oldCounter| &                                  \\
                                     & \lstinline|f(1) shouldBe 3|      \\
                                     & \lstinline|counter = oldCounter| \\
    \bottomrule
  \end{tabular}
  \caption{Esempio di interleaving delle operazioni dei test \lstinline|test1| e \lstinline|test2| che porterebbe al fallimento dell'asserzione fatta da \lstinline|test1|.}
  \label{table:test-f-interleaving-fail}
\end{table}

Quest'ultimo esempio fornisce anche un ottimo spunto per riflettere sull'ulteriore difficoltà nella possibilità di parallelizzare codice che presenta side effect.
Poiché i side effect introducono dipendenze temporali nascoste, non è possibile garantire che l'esecuzione di funzioni impure in parallelo -- e quindi con ordini di esecuzione che potrebbero variare a seconda dell'\term{interleaving} dei processi -- dia sempre lo stesso risultato.
Quindi non potranno essere parallelizzate in maniera automatica; sarà invece necessario ricorrere a euristiche~\cite{cit:safe-automated-refactoring-for-intelligent-parallelization-of-java-8-streams} o estensioni del linguaggio~\cite{cit:pure-functions-in-c-a-small-keyword-for-automatic-parallelization} per poter capire se una funzione è pura o meno e poter sfruttare automaticamente i vantaggi offerti dai moderni processori multi-core senza dover intervenire sulla struttura del programma originale.

\section*{Obiettivi della tesi}
Nonostante buone pratiche di programmazione possano limitare le problematiche associate alla presenza di side effect, è interessante osservare come gli stessi linguaggi di programmazione possano aiutare il programmatore a rendere espliciti gli effetti delle funzioni.

L'obiettivo di questa tesi è effettuare una rassegna dei più diffusi approcci che possono essere adottati per modellare e tracciare la presenza di side effect nelle funzioni.
Partendo da uno degli approcci più noti, quello delle monadi, se ne evidenziano le limitazioni per arrivare ad analizzare meccanismi più sofisticati: MTL e \term{free monad}.
Inoltre, verrà mostrato come queste tecniche possano essere implementate utilizzando Scala come linguaggio di riferimento.

Infine, verrà mostrato un ultimo approccio basato sugli effetti algebrici; questa tecnica, seppur meno diffusa rispetto all'uso delle monadi, rappresenta un'interessante evoluzione nell'ambito dei linguaggi di programmazione puri con l'obiettivo di semplificare la gestione degli effetti delle funzioni.
In questo caso il linguaggio di riferimento utilizzato sarà Koka.

Ogni approccio sarà introdotto in maniera graduale, modellando inizialmente semplici side effect -- come il fallimento di una computazione -- fino a mostrare come ciascun meccanismo possa essere utilizzato per modellare classi di effetti arbitrari.
