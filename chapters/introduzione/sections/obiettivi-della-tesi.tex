\section*{Obiettivi della tesi}
Seguire buone pratiche di programmazione può limitare le problematiche associate alla presenza di side effect. Tuttavia, è interessante osservare come gli stessi linguaggi di programmazione possano aiutare il programmatore a rendere espliciti gli effetti delle funzioni.

Questa rassegna è diretta al programmatore che voglia approfondire il tema della gestione esplicita dei side effect e non abbia particolari conoscenze pregresse; l'unico prerequisito necessario è una certa familiarità con il linguaggio Scala, necessaria a comprendere gli esempi riportati.

La seguente rassegna affronta i diversi approcci che possono essere adottati per modellare gli effetti delle funzioni; partendo da una descrizione del concetto di monade e dei suoi impieghi, sono poi analizzati meccanismi più sofisticati come \ac{MTL} e \term{free monad}.

Queste tecniche sono ormai ampiamente diffuse nell'ambito dei linguaggi di programmazione funzionali puri -- come Haskell -- e stanno iniziando a guadagnare terreno anche in linguaggi di programmazione più diffusi fra cui Python e TypeScript~\cite{cit:pyro,cit:ts-effect,cit:wasmfx}.
In particolare, uno dei campi applicativi di maggior rilievo in cui queste tecniche possono essere adottate con successo è quello della concorrenza strutturata: framework come \emph{ts-effect}~\cite{cit:ts-effect} e \emph{ZIO}~\cite{cit:zio} sono prominenti esempi di come un approccio funzionale puro possa permettere di affrontare la realizzazione di sistemi concorrenti complessi in maniera modulare, componibile ed espressiva.

La trattazione si conclude con l'analisi di un approccio emergente basato su effetti algebrici, mostrando come questo possa risolvere diversi problemi legati all'adozione delle tecniche monadiche e semplificare la diffusione della gestione esplicita degli effetti nel mainstream~\cite{cit:algebraic-effect-handlers-go-mainstream}.

Sicuramente l'adozione di un approccio basato sulla programmazione funzionale pura e la gestione esplicita dei side effect rappresenta un forte cambiamento rispetto alla più diffusa programmazione ad oggetti. Questo nuovo paradigma potrebbe richiedere tempo prima di essere recepito o influenzare il mainstream; rimane evidente il fatto che quello verso cui sembra puntare la ricerca da ormai diverso tempo è un futuro ``orientato agli effetti''. In conclusione, l'obiettivo del seguente elaborato è quello di fornire una visione d'insieme delle tecniche che, con grande probabilità, influenzeranno maggiormente l'evoluzione dei linguaggi nei prossimi anni.
