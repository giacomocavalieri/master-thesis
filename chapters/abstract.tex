\chapter*{Abstract}

Nel corso degli anni i linguaggi di programmazione imperativi ``mainstream'' -- come Java, Python e JavaScript -- sono stati fortemente influenzati dal paradigma di programmazione funzionale, adottando e riconoscendo l'utilità di numerose tecniche come funzioni anonime, \ac{ADT}, \term{pattern matching}, strutture dati immutabili, ecc.

Un approccio che ad oggi rimane prettamente confinato al mondo della programmazione funzionale pura è quello che consiste nel modellare esplicitamente gli effetti delle funzioni, separando in maniera netta -- grazie al supporto automatico del \term{type system} -- codice puro da quello che può presentare side effect.

In linguaggi di programmazione imperativi che non supportano nativamente questi meccanismi, è possibile ottenere una separazione analoga utilizzando architetture come quella ``esagonale''. Tuttavia, il rispetto di queste buone pratiche di programmazione è lasciato interamente all'autodisciplina del programmatore e non è imposto automaticamente dal compilatore.
La risultante presenza incontrollata di side effect ha l'effetto di creare reti di dipendenze invisibili che rendono complesso ragionare sulle proprietà del codice, testarlo o effettuare \term{refactoring}.

L'obiettivo di questa tesi è fornire una visione complessiva delle principali tecniche che possono essere adottate per modellare e rendere esplicita la presenza dei side effect nelle funzioni: \ac{MTL} e \term{free monad}.
Infine, viene mostrato un approccio emergente basato sull'utilizzo degli effetti algebrici.
Tutti i meccanismi analizzati sono introdotti in maniera graduale accompagnando la trattazione con diversi esempi e frammenti di codice che evidenziano i benefici che queste tecniche possono apportare.
