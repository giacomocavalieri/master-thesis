\chapter*{Abstract}

La presenza di side effect arbitrari nel codice complica notevolmente la capacità di ragionare sulle sue proprietà ed effettuare refactoring.
Per questo motivo, sono state sviluppate diverse tecniche che possono essere adottate per rendere esplicita la presenza dei side effect nei tipi delle funzioni.

L'approccio che ad oggi ha ottenuto maggior successo consiste nel modellare la presenza di side effect utilizzando le monadi.
Questo meccanismo, inizialmente impiegato da Haskell per gestire input e output, è stato adottato in diversi linguaggi.

Inoltre, è possibile individuare approcci più sofisticati, come MTL e free monad, per sopperire ad alcuni problemi legati all'uso delle semplici monadi.
Questi sistemi non solo permettono di tracciare la presenza di side effect ma, rendendo possibile definire cosa si intende per ``effetto'', diventano veri e propri strumenti di design e organizzazione del software.

L'approccio monadico non è l'unico che possa essere adottato per la modellazione degli effetti. In particolare, gli effetti algebrici stanno ricevendo sempre maggior attenzione come possibile alternativa.
Questi non solo permettono di modellare e tracciare effetti arbitrari, sono anche un meccanismo in grado di generalizzare diversi costrutti di controllo di flusso come eccezioni, generatori e concorrenza strutturata tramite \emph{async-await}.