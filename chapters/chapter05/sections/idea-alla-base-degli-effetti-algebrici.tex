\section{Idea alla base degli effetti algebrici}

Nella teoria elaborata da Plotkin e Power, un effetto viene modellato tramite un insieme di operazioni e una serie di equazioni che ne descrivono le proprietà~\cite{cit:handling-algebraic-effects,cit:computational-effects-and-operations-an-overview}.

Le operazioni di un effetto determinano le modalità con cui questo può verificarsi; possono quindi essere considerate come dei costruttori di un effetto~\cite{cit:algebraic-operations-and-generic-effects}.
Per poter interpretare un effetto e assegnarvi semantica vengono invece utilizzati degli appositi \term{handler}, o distruttori. Questi specificano come l'esecuzione del programma deve proseguire al verificarsi di ciascuna operazione di uno specifico effetto.

\subsection{Modellazione degli effetti}
\subsubsection{Fallimento di una computazione}
In Koka la parola chiave \lstinline{effect} definisce un effetto e i suoi costruttori sono definiti tramite la parola chiave \lstinline{ctl}.

Si consideri l'esempio già utilizzato del fallimento di una computazione: questo effetto può essere ottenuto con una sola operazione che sarà chiamata \lstinline{fail}:
\begin{lstlisting}[language=koka]
effect failure
  ctl fail(): a
\end{lstlisting}
La dichiarazione permette di definire un effetto chiamato \lstinline{failure} con una sola operazione \lstinline{fail()}. Questa operazione non prende in input alcun valore e restituisce un valore di un qualsiasi tipo. In automatico sarà generata una funzione \lstinline{fail} con tipo \lstinline{forall<a> () -> failure a}.

Grazie all'inferenza dell'\term{effect system}, una funzione che fa uso di \lstinline{fail} nel proprio corpo avrà automaticamente fra i propri effetti quello del fallimento:
\begin{lstlisting}[language=koka]
fun div(n, m)
  match m
    0 -> fail()
    _ -> n / m
\end{lstlisting}
Il tipo inferito per \lstinline{div} sarà \lstinline{(n : int, m : int) -> failure int}. Semplicemente osservando il tipo della funzione è possibile capire che questa potrebbe fallire; l'\term{effect system} fa in modo che non sia possibile ignorare questo aspetto.

\subsubsection{Lettura e scrittura di uno stato mutabile}
Seguendo la stessa strategia evidenziata per l'effetto del fallimento è possibile definire un effetto per modellare la presenza di uno stato mutabile:
\begin{lstlisting}[language=koka]
effect state<s>
  ctl get(): s
  ctl set(new-state: s): ()
\end{lstlisting}

In questo caso l'effetto viene definito rispetto a un generico tipo \lstinline{s} che rappresenta il tipo dello stato mutabile.
\begin{itemize}
  \item \lstinline{get} avrà come tipo \lstinline{() -> state<s> s}: una sua invocazione restituisce un valore di tipo \lstinline{s} e ha il side effect di accedere a uno stato mutabile
  \item \lstinline{set} avrà come tipo \lstinline{(new-state: s) -> state<s> ()}: prende in input il nuovo stato di tipo \lstinline{s} e non restituisce alcun valore, inoltre ha il side effect di accedere a uno stato mutabile \lstinline{s}
\end{itemize}

Queste operazioni di base possono essere combinate per realizzare funzioni più complesse, per esempio:
\begin{lstlisting}[language=koka]
fun update(f)
  val current-state = get()
  val new-state = f(current-state)
  set(new-state)
\end{lstlisting}
In questo caso la funzione prende in input una funzione \lstinline{f} che viene utilizzata per trasformare lo stato e aggiornarlo.

Poiché \lstinline{update} utilizza al proprio interno le operazioni \lstinline{get} e \lstinline{set} avrà nel proprio tipo l'effetto dello stato: il tipo inferito è \lstinline{(f : (s) -> e s) -> <state<s>|e> ()}.

Si noti come il tipo di \lstinline{f} è stato inferito come il più generico possibile: vale a dire che questa funzione potrebbe avere un qualunque side effect \lstinline{e}.
Per questo motivo il tipo di \lstinline{update} combina sia l'effetto dello stato che i possibili ulteriori effetti provenienti dall'invocazione di \lstinline{f}.

\subsection{Interpretazione degli effetti}
Fino a questo momento gli effetti sono stati descritti in maniera astratta definendo una serie di operazioni; queste definiscono un'interfaccia attraverso la quale è possibile generare l'effetto desiderato.

Per poter dare una semantica alle operazioni di un effetto è necessario definire un \term{handler}. Gli handler, detti anche distruttori, sono un meccanismo duale rispetto alle operazioni: mentre l'invocazione di quest'ultime arricchisce il tipo di una funzione aggiungendovi effetti, gli \term{handler} permettono di ``scaricare'' gli effetti dal tipo di una funzione.

In generale, un \term{handler} in grado di dare semantica alle operazioni dell'effetto \lstinline{e1} sarà una funzione che prende in input una computazione con tale effetto e lo rimuove dalla lista dei suoi effetti.

\subsubsection{Interpretazione degli effetti come eccezioni}
Per poter definire un \term{handler} in Koka viene utilizzata la parola chiave \lstinline{handle} a cui viene passata come argomento la computazione della quale devono essere gestiti gli effetti.

Il corpo di un \term{handler} viene definito tramite \term{pattern matching} sulle possibili operazioni dell'effetto che si sta interpretando.

Operativamente, quando viene invocata un'operazione il risultato è simile al sollevamento di un'eccezione in un linguaggio come Java o Scala: il normale flusso di controllo viene interrotto e il controllo passa all'\term{handler} più vicino in grado di gestire lo specifico effetto.

Per esempio, considerando la funzione \lstinline{div} mostrata in precedenza, una possibile gestione del fallimento potrebbe essere la seguente:
\begin{lstlisting}[language=koka]
fun main()
  handle(fn() div(10, 0).println)
    ctl fail() -> println("Exception: divided by zero")
\end{lstlisting}

Nel caso in cui venga chiamata l'operazione \lstinline{fail} il controllo della computazione sarà interrotto e l'\term{handler} eseguirà la porzione di codice associata all'operazione \lstinline{fail}.

In questo caso un blocco \lstinline|handle(...){...}| può essere assimilato al blocco \lstinline{try-catch} di linguaggi che supportano le eccezioni.
Per esempio, si potrebbe gestire il fallimento restituendo un valore di default in caso di eccezione:
\begin{lstlisting}[language=koka]
fun default(value, computation)
  handle(computation)
    ctl fail() -> value
\end{lstlisting}
Il tipo di \lstinline{default} è \lstinline{forall<a,e> (value : a, computation : () -> <failure|e> a) -> e a}:
\begin{itemize}
  \item La computazione restituisce un valore di tipo \lstinline{a} e potrebbe avere, fra gli altri effetti, anche quello del fallimento
  \item \lstinline{value} è un valore di default con lo stesso tipo del valore di ritorno di \lstinline{computation} che viene restituito in caso di fallimento
  \item Il risultato non ha più l'effetto del fallimento fra i propri effetti in quanto tutte le sue operazioni (in questo caso solo \lstinline{fail}) sono state gestite nell'\term{handler}. Rimangono comunque i possibili altri effetti \lstinline{e} della computazione in quanto l'handler gestisce solo le operazioni del fallimento
\end{itemize}

In caso si verifichi l'effetto del fallimento la computazione viene interrotta e l'\term{handler} restituisce immediatamente il valore di default; in caso contrario il risultato sarà quello restituito normalmente dalla computazione:
\begin{lstlisting}[language=koka]
fun main()
  default(-1, { div(10, 0) }).println // -1
  default(-1, { div(10, 2) }).println // 5
\end{lstlisting}

\subsubsection{Cattura del risultato di una computazione}
Un \term{handler} può anche gestire esplicitamente il caso in cui una computazione termini normalmente senza chiamare alcuna operazione.

In Koka questo viene fatto aggiungendo un ramo al \term{pattern matching}:
\begin{lstlisting}[language=koka]
fun to-option(computation)
  handle(computation)
    return(value) -> Some(value)
    ctl fail()    -> Empty
\end{lstlisting}
Il ramo \lstinline{return(value)} viene utilizzato se la computazione gestita dall'\term{handler} termina senza chiamare l'operazione di \lstinline{fail}; in questo caso alla variabile \lstinline{value} viene assegnato il valore di ritorno della computazione.

La funzione quindi permette di rimuovere l'effetto del fallimento di una funzione trasformando il suo valore di ritorno in un valore opzionale:
\begin{lstlisting}[language=koka]
fun main()
  to-option { div(10, 0) } // Empty
  to-option { div(10, 2) } // Some(5)
\end{lstlisting}

Se non viene specificato il caso \lstinline{return} -- come negli esempi mostrati in precedenza -- allora l'\term{handler} restituisce semplicemente il valore della computazione.

\subsubsection{Definizione di strutture di controllo}
Grazie a questo meccanismo è possibile definire funzioni più complicate.

Molti linguaggi di programmazione forniscono, tramite supporto diretto del compilatore, le eccezioni come meccanismo per far terminare prematuramente una computazione.
Inoltre, è possibile definire dei blocchi \lstinline{finally} per stabilire delle azioni che devono essere intraprese quando una computazione dentro ad un blocco \lstinline{try-catch} termina, con un'eccezione o normalmente.

Avendo a disposizione gli effetti è possibile definire blocchi \lstinline{finally} come semplici funzioni di libreria senza bisogno di aggiungere supporto speciale del compilatore o nuove parole chiave al linguaggio:
\begin{lstlisting}[language=koka]
fun finalizer(fin, computation)
  handle(computation)
    return(_)  -> ()
    ctl fail() -> ()
  fin()
\end{lstlisting}
La computazione viene gestita dall'\term{handler} il quale ignora il valore di ritorno e, in caso di eccezione, restituisce \lstinline{()}. In entrambi i casi, dopo la gestione della computazione viene eseguita la funzione di finalizzazione:
\begin{lstlisting}[language=koka]
fun main()
  with finalizer { print("Finalizer") }
  print("Start,")
  print(div(10, 0).show ++ ",")
  print("End")
\end{lstlisting}
In questo caso l'output sarà \lstinline{"Start,5,Finalizer"} in quanto l'operazione di divisione per 0 comporta il fallimento della computazione.

\subsection{Interpretazione degli effetti come eccezioni ripristinabili}
Si consideri adesso l'esempio dello stato: un \term{handler} per l'effetto \lstinline{state} dovrà stabilire come dare semantica alle operazioni di \lstinline{get} e \lstinline{set}.

In questo caso, pensare agli effetti come semplici eccezioni risulta essere limitante. Per capire il problema, si osservi la seguente computazione che ricorre all'effetto dello stato:
\begin{lstlisting}[language=koka]
fun increment-counter()
  val counter: int = get()
  println("counter is " ++ counter.show)
  set(counter + 1)
\end{lstlisting}
Se la chiamata all'operazione \lstinline{get} si limitasse a interrompere l'esecuzione della funzione, non sarebbe possibile eseguire i comandi successivi. Invece, il comportamento desiderato sarebbe quello di ottenere lo stato corrente, stampare un messaggio e modificare lo stato con un nuovo valore.

In realtà, gli effetti sono un meccanismo più generale delle semplici eccezioni. Infatti, non solo possono permettere di interrompere il normale flusso d'esecuzione del codice (come mostrato negli esempi per il side effect del fallimento) ma possono anche ripristinare il flusso d'esecuzione nel punto in cui è stata invocata un'operazione.

\subsubsection{Ripristino del flusso di controllo}
Quando viene chiamata l'operazione di un effetto e il flusso di controllo viene passato all'\term{handler}, questo ha la possibilità di far riprendere la normale esecuzione della funzione utilizzando la funzione \lstinline{resume}.

Il tipo della funzione \lstinline{resume} cambierà in base all'effetto invocato. Si consideri per esempio l'effetto di leggere valori di configurazione:
\begin{lstlisting}[language=koka]
effect configuration
  ctl get-name(): string
  ctl get-int(): int

fun print-config()
  val name = get-name()
  println("Name is: " ++ name)
  val n = get-int()
  println("Int is: " ++ n.show)
\end{lstlisting}

Quindi, quando si verifica l'effetto \lstinline{get-name}, per poter riprendere la normale computazione è necessario fornire in un qualche modo una stringa che sarà il valore di ritorno atteso dalla funzione \lstinline{get-name}. In maniera analoga, quando si verifica l'effetto \lstinline{get-int} sarà necessario fornire un valore intero per poter far riprendere la computazione.

Un \term{handler} per questo effetto potrebbe, per esempio, fornire sempre gli stessi valori:
\begin{lstlisting}[language=koka]
fun constant-config(computation)
  handle(computation)
    ctl get-name() -> resume("Koka")
    ctl get-int()  -> resume(11)
\end{lstlisting}
A seconda del ramo del \term{pattern matching} sarà presente in scope una funzione \lstinline{resume} con tipo differente: nel primo caso richiede una stringa mentre nel secondo un valore intero.
L'effetto in entrambi i casi è quello di riprendere il normale flusso di esecuzione dando come valore di ritorno dell'operazione il valore passato alla funzione \lstinline{resume}:
\begin{lstlisting}[language=koka]
fun main()
  with constant-config
  print-config()
  // Name is: Koka
  // Int is: 11
\end{lstlisting}

\subsubsection{Handler per lo stato mutabile}
Tornando all'esempio dello stato mutabile, è possibile osservare come, per riprendere la computazione in seguito ad una chiamata a \lstinline{get}, sia necessario fornire un valore del tipo dello stato.
Un possibile \term{handler} potrebbe essere il seguente:
\begin{lstlisting}[language=koka]
fun initial-state(initial-state, computation)
  var state := initial-state
  handle(computation) 
    return(x) -> (state, x)
    ctl get() -> resume(state)
    ctl set(new-state) ->
      state := new-state
      resume(())
\end{lstlisting}

La funzione utilizza internamente una variabile per tenere traccia dello stato attuale: quando viene chiamato \lstinline{get} l'esecuzione viene ripresa con il valore attuale dello stato; invece alla chiamata di \lstinline{set} si modifica lo stato attuale e poi viene fatta ripartire la computazione fornendo alla funzione \lstinline{resume} il valore \lstinline{()} -- vale a dire il tipo di ritorno atteso dalla funzione \lstinline{set}.

\subsubsection{Uso delle variabili in Koka}
È possibile osservare come, per poter implementare l'\term{handler} per lo stato mutabile, si sia fatto uso di una variabile.
Questo potrebbe apparire strano data la natura funzionale di Koka, tuttavia il linguaggio introduce la possibilità di utilizzare variabili con alcune limitazioni che permettono di mantenere il codice puro.

La scelta di Koka è quella di permettere la definizione di variabili all'interno del corpo di una funzione a condizione che queste non escano dallo \term{scope} lessicale della funzione all'interno della quale sono dichiarate. Per esempio il seguente codice risulterebbe in un errore da parte del compilatore:
\begin{lstlisting}[language=koka]
fun variable()
  var x:= 0
  fn() { x := 1; x }
  // error: a reference to a local variable 
  // escapes its scope
\end{lstlisting}

Le variabili non sono elementi di prima classe e non possono essere passate come argomenti alle funzioni o restituite come risultato di una funzione; ogni riferimento a una variabile equivale automaticamente a leggerne il contenuto:
\begin{lstlisting}[language=koka]
fun variables()
  var x := 0
  println(show(x)) // Read value referenced by x
  x := x + 1
  x // Returns the immutable value referenced by x
\end{lstlisting}
Se, come nell'esempio, viene restituita una variabile in realtà si restituisce il suo contenuto immutabile e non un riferimento mutabile.

Questa limitazione permette di usare le variabili in maniera opportunistica all'interno del corpo di una funzione per poterla implementare in maniera efficiente. Tuttavia, l'uso di variabili non è mai osservabile dall'esterno della funzione che continua a comportarsi come una funzione pura. Per esempio si potrebbe realizzare un'implementazione efficiente per calcolare l'n-esimo numero della sequenza di Fibonacci:
\begin{lstlisting}[language=koka]
fun fibonacci(n)
  var x := 0
  var y := 1
  repeat(n)
    val temp = y
    y := x + y
    x := temp
  x
\end{lstlisting}
I controlli del \term{type system} di Koka garantiscono che le variabili non possano sopravvivere oltre lo scope sintattico della funzione all'interno della quale sono dichiarate. Per questo la funzione \lstinline{fibonacci}, nonostante la mutazione di una variabile, sarà comunque considerata una funzione pura con tipo \lstinline{(int) -> <> int}.
