\section{Confronto con l'approccio monadico}
Gli effetti algebrici rappresentano un'alternativa interessante all'utilizzo delle monadi per la modellazione dei side effect.

Nella definizione di una monade l'attenzione viene posta sulle operazioni di \lstinline{pure} e \lstinline{flatMap} dovendo indicare come può avvenire la concatenazione delle computazioni e sono queste definizioni a dare semantica alle diverse istruzioni.

L'approccio degli effetti, al contrario, si concentra direttamente sulla definizione e interpretazione delle operazioni tramite l'uso di \term{handler}. L'attenzione è posta su come interpretare ciascun effetto e non su come devono essere concatenate le operazioni.
Sotto questo punto di vista, gli effetti algebrici presentano una forte similarità con le \term{free monad}: in entrambi i casi si definisce una serie di istruzioni che devono essere interpretate per dare semantica al programma.

Nel caso delle \term{free monad} il programmatore non deve definire esplicitamente il metodo \lstinline{flatMap} -- e quindi come concatenare le operazioni -- che viene fornito automaticamente dall'uso del tipo \lstinline{Program}.
Anche in quel caso l'interpretazione di un programma avviene interpretandone le singole istruzioni.

\subsection{Boilerplate ed inferenza}
Un aspetto sotto il quale le \term{free monad} e gli effetti algebrici differiscono sta nella quantità di \term{boilerplate} necessaria per poter utilizzare ciascun approccio.

Nel caso delle \term{free monad} è necessario definire, per ciascuna delle operazioni di uno specifico linguaggio degli \term{smart constructor}. Si consideri nuovamente l'esempio dello stato mutabile:
\scalaFromFile{7}{9}{monads/free/State.scala}
\scalaFromFile{15}{17}{monads/free/State.scala}

Inoltre, è stato necessario prestare particolare attenzione nella realizzazione di un meccanismo che permettesse di usare insiemi di istruzioni differenti in una stessa funzione.
Per rendere possibile questa combinazione è stato descritto il meccanismo delle iniezioni che, sfruttando il passaggio di istanze implicite, permettono di unire operazioni di più effetti in un linguaggio composto.

Nel caso di Koka la definizione di un effetto è minimale e non richiede alcun \term{boilerplate}:
\begin{lstlisting}[language=koka]
effect state<s>
  ctl get(): s
  ctl set(new-state: s): ()
\end{lstlisting}
Inoltre, l'\term{effect system} permette di tracciare automaticamente tutti i side effect delle funzioni e non è necessario alcun passo aggiuntivo per poter combinare funzioni che presentano side effect differenti:
\begin{lstlisting}[language=koka]
fun mixed-side-effects()
  println("Side effect!")
  fail()
\end{lstlisting}
Il tipo della funzione viene inferito come \lstinline{() -> <console, failure> ()}. Il meccanismo è estremamente semplice da spiegare: tutti i side effect delle singole funzioni invocate dentro al corpo di una funzione sono accumulati nel suo tipo.

Si faccia il confronto con l'equivalente versione Scala che fa uso di \term{free monad} dove le annotazioni di tipo sono necessarie:
\begin{lstlisting}[language=scala3]
def mixedSideEffects[I[_]: With[ConsoleDSL]: With[FailDSL]]() =
  for
    _ <- println("Side effect!")
    _ <- fail()
  yield ()
\end{lstlisting}
Per poter comprendere l'esempio un programmatore deve essere a conoscenza di meccanismi del linguaggio non banali come il passaggio di istanze implicite, i \term{context bound} e gli \term{higher-kinded types}.

Il problema sorge dal fatto che linguaggi come Scala non sono stati pensati per tracciare gli effetti delle funzioni come elementi di prima classe nel proprio \term{type system}.
È quindi necessario implementare un'infrastruttura -- come MTL e \term{free monad} -- in termini di funzionalità preesistenti del linguaggio che permetta di arricchirlo con le funzionalità desiderate.
Per esempio, durante la trattazione si è fatto uso estensivo di istanze implicite e passaggio di parametri implicito per poter codificare le monadi e vincolare gli effetti di una funzione.

Chiaramente, poiché il linguaggio non nasce con tale obiettivo, la semplicità di utilizzo non potrà essere paragonabile a quella di un linguaggio pensato appositamente per tracciare i side effect come Koka.

\subsection{Stile di programmazione}
Un'altra fondamentale differenza fra effetti algebrici e approccio monadico sta nello stile del codice: avendo modellato gli effetti tramite monadi l'unico modo per poterli concatenare consiste nell'usare \lstinline{flatMap}:
\begin{lstlisting}[language=scala3]
def flatMapExample: State[Int, Unit] =
  State.get.flatMap { state =>
    State.set(f(state)).flatMap { _ => 
      Monad.pure(())
    }
  }
\end{lstlisting}
Il codice risultante è generalmente poco leggibile e per questo motivo viene fornito lo zucchero sintattico della \term{for comprehension}:
\begin{lstlisting}[language=scala3]
def forComprehensionExample: State[Int, Unit] =
  for
    state <- State.get
    _     <- State.set(f(state))
  yield ()
\end{lstlisting}
Il codice assume nuovamente un aspetto ``imperativo''. Tuttavia è importante che questo non tragga in inganno: non è possibile scrivere
\lstinline{State.set(f(State.get))} anche se è quello che ci si potrebbe aspettare di poter fare.
Infatti \lstinline{get} non è un semplice valore ma una computazione all'interno della monade \lstinline{State}.

Si consideri invece l'approccio adottato nel caso degli effetti algebrici:
\begin{lstlisting}[language=koka]
fun direct-style(f)
  set(f(get()))
\end{lstlisting}

Le operazioni \lstinline{set} e \lstinline{get} non sono altro che normali funzioni arricchite da annotazioni relative ai side effect: il codice risultante è esattamente quello che potrebbe scrivere un programmatore ``imperativo'' che non è mai stato esposto al concetto degli effetti algebrici.

La differenza chiave sta nel fatto che Koka traccia automaticamente gli effetti delle funzioni. L'\term{effect system} può essere visto come una sorta di rete di sicurezza che guida il programmatore nella corretta gestione degli effetti.

Allo stesso tempo viene ridotto il carico cognitivo e non è aggiunto alcun overhead sintattico come nel caso della \term{for comprehension}.
Questo è un vantaggio da non sottovalutare in quanto rende il linguaggio più semplice da leggere e scrivere riducendone la barriera d'adozione.
